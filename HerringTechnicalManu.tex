\documentclass[]{article}
\usepackage{lmodern}
\usepackage{amssymb,amsmath}
\usepackage{ifxetex,ifluatex}
\usepackage{fixltx2e} % provides \textsubscript
\ifnum 0\ifxetex 1\fi\ifluatex 1\fi=0 % if pdftex
  \usepackage[T1]{fontenc}
  \usepackage[utf8]{inputenc}
\else % if luatex or xelatex
  \ifxetex
    \usepackage{mathspec}
  \else
    \usepackage{fontspec}
  \fi
  \defaultfontfeatures{Ligatures=TeX,Scale=MatchLowercase}
\fi
% use upquote if available, for straight quotes in verbatim environments
\IfFileExists{upquote.sty}{\usepackage{upquote}}{}
% use microtype if available
\IfFileExists{microtype.sty}{%
\usepackage{microtype}
\UseMicrotypeSet[protrusion]{basicmath} % disable protrusion for tt fonts
}{}
\usepackage[margin=1in]{geometry}
\usepackage{hyperref}
\hypersetup{unicode=true,
            pdftitle={The dream and the reality: meeting decision-making time frames while incorporating ecosystem and economic models into management strategy evaluation},
            pdfauthor={Jonathan Deroba, Sarah Gaichas, Min-Yang Lee, Rachel G. Feeney, Deirdre Boelke, Brian J. Irwin},
            pdfborder={0 0 0},
            breaklinks=true}
\urlstyle{same}  % don't use monospace font for urls
\usepackage{longtable,booktabs}
\usepackage{graphicx,grffile}
\makeatletter
\def\maxwidth{\ifdim\Gin@nat@width>\linewidth\linewidth\else\Gin@nat@width\fi}
\def\maxheight{\ifdim\Gin@nat@height>\textheight\textheight\else\Gin@nat@height\fi}
\makeatother
% Scale images if necessary, so that they will not overflow the page
% margins by default, and it is still possible to overwrite the defaults
% using explicit options in \includegraphics[width, height, ...]{}
\setkeys{Gin}{width=\maxwidth,height=\maxheight,keepaspectratio}
\IfFileExists{parskip.sty}{%
\usepackage{parskip}
}{% else
\setlength{\parindent}{0pt}
\setlength{\parskip}{6pt plus 2pt minus 1pt}
}
\setlength{\emergencystretch}{3em}  % prevent overfull lines
\providecommand{\tightlist}{%
  \setlength{\itemsep}{0pt}\setlength{\parskip}{0pt}}
\setcounter{secnumdepth}{0}
% Redefines (sub)paragraphs to behave more like sections
\ifx\paragraph\undefined\else
\let\oldparagraph\paragraph
\renewcommand{\paragraph}[1]{\oldparagraph{#1}\mbox{}}
\fi
\ifx\subparagraph\undefined\else
\let\oldsubparagraph\subparagraph
\renewcommand{\subparagraph}[1]{\oldsubparagraph{#1}\mbox{}}
\fi

%%% Use protect on footnotes to avoid problems with footnotes in titles
\let\rmarkdownfootnote\footnote%
\def\footnote{\protect\rmarkdownfootnote}

%%% Change title format to be more compact
\usepackage{titling}

% Create subtitle command for use in maketitle
\newcommand{\subtitle}[1]{
  \posttitle{
    \begin{center}\large#1\end{center}
    }
}

\setlength{\droptitle}{-2em}
  \title{The dream and the reality: meeting decision-making time frames while
incorporating ecosystem and economic models into management strategy
evaluation}
  \pretitle{\vspace{\droptitle}\centering\huge}
  \posttitle{\par}
  \author{Jonathan Deroba, Sarah Gaichas, Min-Yang Lee, Rachel G. Feeney, Deirdre
Boelke, Brian J. Irwin}
  \preauthor{\centering\large\emph}
  \postauthor{\par}
  \date{}
  \predate{}\postdate{}

\usepackage{setspace}

\usepackage{float}
\let\origfigure\figure
\let\endorigfigure\endfigure
\renewenvironment{figure}[1][2] {
    \expandafter\origfigure\expandafter[H]
} {
    \endorigfigure
}

\begin{document}
\maketitle

\doublespacing

\section{Introduction}\label{introduction}

Management strategy evaluation (MSE) uses simulation to evaluate the
trade-offs resulting from alternative management options in the face of
uncertainty (Punt et al. 2016a). MSEs require time, however, for
stakeholder input, data collection, and model development (Butterworth
2007) (Punt et al., 2014). As such, the process can take much longer
than ``traditional'' management time frames (Butterworth 2007). The
development time is also likely to lengthen when explicit ecosystem,
multi-species, or socioeconomic considerations are desired because the
data and modeling needs, and subsequent uncertainties, are all greater
than in a single species approach. This manuscript chronicles the
development of an MSE done on a truncated timetable (\textasciitilde{}12
months) required to meet management time frames. The objectives of this
manuscript are to:

\begin{enumerate}
\def\labelenumi{\arabic{enumi}.}
\item
  Evaluate the relative performance of HCRs at meeting herring fishery
  objectives, including those related to predators of herring, as
  informed by stakeholder input, and,
\item
  Discuss our approach to developing an MSE on a relatively truncated
  timetable in order to meet management time frames, and identify the
  lessons learned throughout the process, especially as they relate to
  using MSE as a tool to advance an ecosystem based approach to
  management (Plagányi et al. 2014).
\end{enumerate}

The fisheries management process in New England usually starts when
managers percieve a problem causing a management goal or objective to be
unmet. Managers will propose solutions and a technical group, typically
composed of scientists and policy analysts from state agencies and the
National Marine Fisheries Service (NMFS), will analyze these possible
solutions. Both managers and the technical group can be aided and
advised by an Advisory Panel, a closed group appointed by managers that
is typically drawn from members of the fishing industry, processing
sector, and Environmental Non-Governmental Organizations (ENGOs). In
addition to the public comment process, the Advisory Panel is one of the
mechanisms for stakeholders to raise awareness of management problems to
managers. The Council will eventually select a solution through a vote
and NMFS will, after verifying that it is consistent with applicable
laws, translate those solutions into regulations and enforce those
regulations. Management actions, particularly contentious ones, can take
years to develop and implement.

In January 2016, the New England Fishery Management Council (NEFMC), the
political body responsible for federally managed species in the
northeast US, requested an MSE to evaluate harvest control rules (HCRs)
for Atlantic herring (hereafter herring) \emph{Clupea harengus}
\footnote{The process is described in detail in Feeney \textit{et al}, YYYY}.
Fishery managers wanted to develop an HCR that, among other things,
accounted for the role of Atlantic herring as forage in the ecosystem;
however, the exact ``accounting'' system was left to stakeholders.
Perhaps because recent herring management actions have been quite
contentious, managers decided that any member of the public was a
stakeholder. Stakeholder input was solicited on objectives, performance
metrics, and control rules themselves during a workshop in May 2016.
Members of the herring, lobster, groundfish, tuna, recreational
fishermen, and whale-watching industries participated as well as ENGOs,
federal and state agencies, and academics. Input from these diverse
stakeholders were then distilled into a set of desired outcomes and
metrics that could be included in a scientific model. Notably, while the
techincal group knew the general scope of the modeling exercise (HCRs
that account for herring as forage), detailed modeling of many
components of the ecosystem could not begin in earnest until this step
was finished.

The NEFMC desired to have results from the MSE ready to inform fisheries
management decisions within one year, which left little time to develop
the technical aspects of the MSE. A particularly challenging aspect of
the time frame was deciding what technical aspects of the MSE (e.g.,
operating models) could be compromised (i.e., perhaps not ideal from a
scientific or best-practices standpoint) while ensuring the results
would still accurately portray the trade-offs among objectives and
remain relevant for decision-making.

\textbf{Background on herring} Mostly used as lobster bait. Caught with
purse seines (night, inshore), single and paired midwater trawl
(anytime, anywhere) and bottomtrawl (mostly southern new england). High
volume, low prices. large boats. Single species stock assessments
indicated biomass was historically high. Small retrospective pattern?
Changes over time in some life-history parameters
(weight-at-age=growth). Higher uncertainty about other parameters (M and
steepness). Show up in diets of many predators that are sampled in
NEFSC's bottom-trawl survey. It's forage, so it's also eaten by birds,
whales, and other things.\\

\section{Methods}\label{methods}

\textbf{Herring}

\emph{Basics.--} An MSE was developed specific to Gulf of Maine -
Georges Bank Atlantic herring. The MSE was a modified version of that
used in Deroba et al. (2015), and symbols were largely consistent with
Deroba et al. (2015) (Table 1). The MSE was based on an age-structured
simulation that considered fish from age-1 through age-8+ (age-8 and
older), which is consistent with the age ranges used in the 2012 and
2015 Atlantic herring stock assessments (Northeast Fisheries Science
Center 2012, Deroba et al. 2015). The abundances at age in year one of
all simulations equaled the equilibrium abundances produced by the
fishing mortality rate that would reduce the population to 40\% of
\(SSB_{F=0}\). Abundance in each subsequent age and year was calculated
assuming that fish died exponentially according to an age and year
specific total instantaneous mortality rate (Table XX T1-T2).

Recruitment followed Beverton-Holt dynamics (Francis 1992) (Table XX
T3-T5). The variance of recruitment process errors (\(\sigma_R^2\))
equaled 0.36 and the degree of autocorrelation (\(\omega\)) equaled 0.1,
which are values consistent with recruitment estimates from a recent
Atlantic herring stock assessment (Deroba et al. 2015).

\emph{Assessment Error.--} A stock assessment was approximated (i.e.,
assessment errors) similar to Punt et al. (2008) and Deroba et al.
(2015). Assessment error was modeled as a year-specific lognormal random
deviation common to all ages, with first-order autocorrelation and a
term that created the option to include bias (Table XX T6-T7). The
variance of assessment errors (\(\sigma_\phi^2\) ) equaled 0.05 and
autocorrelation (\(\vartheta\)) equaled 0.7. Rho (\(\rho\)) allowed for
the inclusion of bias in the assessed value of abundance (see below;
Deroba 2014). Assessed spawning stock biomass (\(\widehat{SSB}_y\)) was
calculated similarly to \(SSB_y\) except with \(N_{a,y}\) replaced with
\(\widehat{N}_{a,y}\) (Table XX T5), and assessed total biomass
(\(\widehat{B}_y\)) was calculated as the sum across ages of the product
of \(\widehat{N}_{a,y}\) and \(W_a\).

\emph{Operating Models.--} The stakeholder workshops identified
uncertainties about herring life history traits and stock assessment,
and the effect of some of these uncertainties on harvest control rule
performance was evaluated by simulating the control rules for each of
eight operating models (Table 2; Figures 1-2). The uncertainties
addressed by the eight operating models included: Atlantic herring
natural mortality and recruitment , Atlantic herring weight-at-age, and
possible bias in the stock assessment beyond the unbiased measurement
error (\(\epsilon_{\phi y}\)).

The specific values used in the operating models for each of the
uncertainties were premised on data used in recent stock assessments or
estimates from fits of stock assessment models (Northeast Fisheries
Science Center 2015). Natural mortality in recent stock assessments has
varied among ages and years, with \(M\) being higher during 1996-2014
than in previous years (Northeast Fisheries Science Center 2012, 2015).
Natural mortality, however, has also been identified as an uncertainty
in the stock assessments and sensitivity runs have been conducted
without higher \(M\) during 1996-2014, such that \(M\) was constant
among years (Northeast Fisheries Science Center 2012, 2015). To capture
uncertainty in \(M\) in the MSE, operating models were run with either
relatively high or low \(M_a\) (Table 2; Figure 1). Relatively high
\(M_a\) values equaled the age-specific natural mortality rates used for
the years 1996-2014 in the stock assessment. Relatively low \(M_a\)
values in the MSE equaled the age-specific natural mortality rates used
for the years 1965-1995 in the stock assessment. In the MSE, \(M_a\) was
always time invariant.

Uncertainty in estimates of stock-recruit parameters were represented in
the MSE by using the parameters estimated by stock assessments fit with
and without the higher \(M\) during 1996-2014. Stock assessment fits
with higher \(M\) during 1996-2014 produced estimates of steepness and
\(SSB_{F=0}\) that were lower than in stock assessment fits without
higher \(M\) during 1996-2014 (Table 3; Figure 1). Thus, operating
models with relatively high \(M_a\) always had relatively low steepness
and \(SSB_{F=0}\), and the opposite held with relatively low \(M_a\)
(Table 2).

Uncertainty in Atlantic herring size-at-age was accounted for by having
operating models with either fast or slow growth (i.e., weights-at-age;
Table 2; Figure 3). Atlantic herring weight-at-age generally declined
from the mid-1980s through the mid-1990s, and has been relatively stable
since. Reasons for the decline are speculative and no causal
relationships have been established. Thus, fast growth operating models
had weights-at-age that equaled the January 1 weights-at-age from the
most recent stock assessment averaged over the years 1976-1985, while
the slow growth operating models averaged over the years 2005-2014
(Deroba 2015). In the MSE, weight-at-age was always time invariant.
Differences in \(M\), stock-recruit parameters, and weights-at-age led
to differences in unfished and \(MSY\) reference points among operating
models (Table 3). The effect of \(M\) and stock-recruit parameters was
larger than the effect of differences in weight-at-age (Table 3).

To address concerns about possible stock assessment bias, operating
models with and without a positive bias were included. In operating
models without bias, \(\rho\)=0 and the only assessment error was that
caused by the unbiased measurement errors (\(\epsilon_{\phi y}\)). In
operating models with bias, \(\rho\)=0.6, which was based on the degree
of retrospective pattern in \(SSB\) from the most recent stock
assessment (Northeast Fisheries Science Center 2015).

\emph{Harvest Control Rules.--} Several basic control rules were
evaluated, including a biomass based control rule (Katsukawa 2004), a
constant catch rule, and a conditional constant catch rule (Figure 3;
Clark and Hare (2004), Deroba and Bence (2012)). The biomass based
control rule was defined by three parameters: the proportion (\(\psi\))
of \(F_{MSY}\) that dictates the maximum desired fishing mortality rate
(\(\tilde{F}\)), an upper SSB threshold (\(SSB_{up}\)), and a lower SSB
threshold (\(SSB_{low}\)). The \(\tilde{F}\) equaled the maximum when
\(\widehat{SSB}\) was above the upper threshold, declined linearly
between the upper and lower thresholds, and equaled zero below the lower
threshold:

\begin{equation}
  \label{Fy_equation}
    \tilde{F}_y=
    \begin{cases}
      \psi F_{MSY}, & \text{if}\ \widehat{SSB}_y \geq SSB_{up} \\
      \psi F_{MSY} \frac{\widehat{SSB}_y - SSB_{low}}{SSB_{up}-SSB_{low}}, & \text{if}\ SSB_{low} < \widehat{SSB}_y < SSB_{up}\\
      0, & \text{if}\ \widehat{SSB}_y \leq SSB_{low}
    \end{cases}
  \end{equation}

The \(\tilde{F}_y\) was then used to set a quota in year y + 1 (Table XX
T8). \(\tilde{F}_{ay}\) equaled \(\tilde{F}_y\) times \(S_a\), and
\(S_a\) was time and simulation invariant selectivity at age equal to
the values for the mobile gear fishery reported in (Northeast Fisheries
Science Center 2015, Table 1). \(\tilde{F}_y\) was used to set a quota
in the following year to approximate the practice of using projections
based on an assessment using data through year y - 1 to set quotas in
the following year(s). Furthermore, although \(\tilde{F}_y\) was set
using \(\widehat{SSB}_y\), the quota was based on \(\widehat{B}_y\)
because the fishery selects some immature ages. The fully selected
fishing mortality rate that would remove the quota from the true
population (\(\bar{F}_y\)) was found using Newton-Raphson iterations.
Several variations of the biomass based rule were also evaluated. These
variations included applying the control rule annually, using the same
quota for three year blocks such that the control rule is applied every
fourth year (i.e., \(Q_{y+1}=Q_{y+2}=Q_{y+3}\)), using the same quota
for 5 year blocks, and using the same quota for three year blocks but
restricting the change in the quota to 15\% in either direction when the
control rule was reapplied in the fourth year. Thus, four variants of
the biomass based control rule were evaluated: 1) annual application, 2)
three year blocks, 3) five year blocks, and 4) 3 year blocks with a 15\%
restriction.

For each biomass based control rule variant, a range of values for the
three parameters defining the control rule were evaluated. The
proportion (\(\psi\)) of \(F_{MSY}\) that dictates the maximum desired
fishing mortality rate was varied from 0.1\(F_{MSY}\) to 1.0\(F_{MSY}\)
in increments of 0.1, while the upper and lower SSB threshold parameters
(\(SSB_{up} \; SSB_{low}\)) were varied from 0.0\(SSB_{MSY}\) to
4\(SSB_{MSY}\) but with inconsistent increments (i.e., 0.0, 0.1, 0.3,
0.5, 0.7, 0.9, 1.0, 1.1, 1.3, 1.5, 1.7, 2.0, 2.5, 3, 3.5, 4). The full
factorial of combinations for the three biomass based control rule
parameters produced 1,360 shapes (note \(SSB_{low}\) must be
\textless{}= \(SSB_{up}\)) and each of these shapes was evaluated for
each of the four biomass based control rule variants described above.

The constant catch control rule is defined by one parameter, a desired
constant catch (i.e., quota) amount (Figure 3). The constant catch
amounts were varied from 0.1\(MSY\) to 1.0\(MSY\) in increments of 0.1.

The conditional constant catch rule used a constant desired catch amount
unless removing that desired catch from the assessed biomass caused the
fully selected fishing mortality rate to exceed a pre-determined
maximum, in which case the desired catch was set to the value produced
by applying the maximum fully selected fishing mortality rate to the
assessed biomass (Figure 3). Thus, the conditional constant catch rule
has two policy parameters: a desired constant catch amount, and a
maximum fishing mortality rate. The constant catch amounts were varied
from 0.1\(MSY\) to 1.0\(MSY\) in increments of 0.1, while the maximum
fishing mortality rate equaled 0.5\(F_{MSY}\). When the maximum fishing
mortality rate portion the conditional constant catch rule was invoked,
a quota was set in the same manner as when
\(\widehat{SSB}_y >= SSB_{up}\) in the biomass based control rule
described above.

\emph{Implementation Error.--} Implementation errors were also included
in a similar way as in Punt et al. (2008) and Deroba and Bence (2012),
as year-specific lognormal random deviations (Table XX T9). The variance
of implementation errors (\(\sigma_\theta^2\)) equaled 0.001.

\textbf{Predators}

There are two components of predator modeling for the herring MSE: a
predator population model, and a herring-predator relationship model to
link herring with predator populations. Here, we give an overview of the
modeling process, and we describe the decisions made in parameterizing
individual predator models and herring-predator relationships in the
following sections. The overall population in numbers for each predator
each year \(N_{y}\) is modeled with a delay-difference function:

\begin{equation}
N_{y+1} = N_{y}S_{y} +  R_{y+1} \label{delaydiffN_equation},
\end{equation}

where annual predator survival \(S_{y}\) is based on annual natural
mortality \(v\) and exploitation \(u\)

\begin{equation}
S_{y} =  (1-v_{y})(1-u) \label{survival_equation},
\end{equation}

and annual recruitment \(R_{y}\) (delayed until recruitment age a) is a
Beverton-Holt function:

\begin{equation}
R_{y+a} = \frac{\alpha B_{y}}{\beta + B_{y}} \label{delaydiffrec_equation}.
\end{equation}

Predator recruitment parameters are defined with steepness = \(h\),
unfished recruitment \(R_{F=0}\), and unfished spawning biomass
\(B_{F=0}\) as

\begin{equation}
\alpha = \frac{4hR_{F=0}}{5h-1} \label{predalpha_equation}, and
\end{equation}

\begin{equation}
\beta = \frac{(B_{F=0}/R_{F=0})((1-h)/(4h))}{(5h-1)/(4hR_{F=0})} \label{predbeta_equation}
\end{equation}

.

Predator population biomass is defined with Ford-Walford plot intercept
(\(FWint\)) and slope (\(FWslope\)) growth parameters

\begin{equation}
B_{y+1} = S_{y} (FWint N_{y} + FWslope B_{y}) + FWint R_{y+1} \label{delaydiffB_equation}
\end{equation}

Parameterizing this model requires specification of the
stock-recruitment relationship (steepness h and unfished spawning stock
size in numbers or biomass), the natural mortality rate, the fishing
mortality (exploitation) rate, the initial population size, and the
weight at age of the predator (Ford-Walford plot intercept and slope
parameters). For each predator, population parameters were derived from
different sources (Tab. 1).

Table 1. Predator population model specification and parameter sources

\begin{longtable}[]{@{}lllll@{}}
\toprule
\begin{minipage}[b]{0.17\columnwidth}\raggedright\strut
\strut
\end{minipage} & \begin{minipage}[b]{0.17\columnwidth}\raggedright\strut
Highly migratory\strut
\end{minipage} & \begin{minipage}[b]{0.17\columnwidth}\raggedright\strut
Seabird\strut
\end{minipage} & \begin{minipage}[b]{0.17\columnwidth}\raggedright\strut
Groundfish\strut
\end{minipage} & \begin{minipage}[b]{0.17\columnwidth}\raggedright\strut
Marine mammal\strut
\end{minipage}\tabularnewline
\midrule
\endhead
\begin{minipage}[t]{0.17\columnwidth}\raggedright\strut
Stakeholder preferred species\strut
\end{minipage} & \begin{minipage}[t]{0.17\columnwidth}\raggedright\strut
Bluefin tuna\strut
\end{minipage} & \begin{minipage}[t]{0.17\columnwidth}\raggedright\strut
Common tern\strut
\end{minipage} & \begin{minipage}[t]{0.17\columnwidth}\raggedright\strut
not specified\strut
\end{minipage} & \begin{minipage}[t]{0.17\columnwidth}\raggedright\strut
not specified\strut
\end{minipage}\tabularnewline
\begin{minipage}[t]{0.17\columnwidth}\raggedright\strut
Species modeled\strut
\end{minipage} & \begin{minipage}[t]{0.17\columnwidth}\raggedright\strut
Bluefin tuna (western Atlantic stock)\strut
\end{minipage} & \begin{minipage}[t]{0.17\columnwidth}\raggedright\strut
Common tern (Gulf of Maine colonies as defined by the GOM Seabird
Working Group)\strut
\end{minipage} & \begin{minipage}[t]{0.17\columnwidth}\raggedright\strut
Spiny dogfish (GOM and GB cod stocks also examined)\strut
\end{minipage} & \begin{minipage}[t]{0.17\columnwidth}\raggedright\strut
none, data limited (Minke \& humpback whales, harbor porpoise, harbor
seal examined)\strut
\end{minipage}\tabularnewline
\begin{minipage}[t]{0.17\columnwidth}\raggedright\strut
Stock-recruitment\strut
\end{minipage} & \begin{minipage}[t]{0.17\columnwidth}\raggedright\strut
Current assessment and literature\strut
\end{minipage} & \begin{minipage}[t]{0.17\columnwidth}\raggedright\strut
Derived from observations\strut
\end{minipage} & \begin{minipage}[t]{0.17\columnwidth}\raggedright\strut
Current assessment and literature\strut
\end{minipage} & \begin{minipage}[t]{0.17\columnwidth}\raggedright\strut
No time series data for our region\strut
\end{minipage}\tabularnewline
\begin{minipage}[t]{0.17\columnwidth}\raggedright\strut
Natural mortality\strut
\end{minipage} & \begin{minipage}[t]{0.17\columnwidth}\raggedright\strut
Current assessment\strut
\end{minipage} & \begin{minipage}[t]{0.17\columnwidth}\raggedright\strut
Literature\strut
\end{minipage} & \begin{minipage}[t]{0.17\columnwidth}\raggedright\strut
Current assessment\strut
\end{minipage} & \begin{minipage}[t]{0.17\columnwidth}\raggedright\strut
Derivable from assessment?\strut
\end{minipage}\tabularnewline
\begin{minipage}[t]{0.17\columnwidth}\raggedright\strut
Fishing mortality\strut
\end{minipage} & \begin{minipage}[t]{0.17\columnwidth}\raggedright\strut
Current assessment\strut
\end{minipage} & \begin{minipage}[t]{0.17\columnwidth}\raggedright\strut
n/a\strut
\end{minipage} & \begin{minipage}[t]{0.17\columnwidth}\raggedright\strut
Current assessment\strut
\end{minipage} & \begin{minipage}[t]{0.17\columnwidth}\raggedright\strut
Derivable from assessment?\strut
\end{minipage}\tabularnewline
\begin{minipage}[t]{0.17\columnwidth}\raggedright\strut
Initial population\strut
\end{minipage} & \begin{minipage}[t]{0.17\columnwidth}\raggedright\strut
Current assessment\strut
\end{minipage} & \begin{minipage}[t]{0.17\columnwidth}\raggedright\strut
Derived from observations\strut
\end{minipage} & \begin{minipage}[t]{0.17\columnwidth}\raggedright\strut
Current assessment\strut
\end{minipage} & \begin{minipage}[t]{0.17\columnwidth}\raggedright\strut
Derivable from assessment?\strut
\end{minipage}\tabularnewline
\begin{minipage}[t]{0.17\columnwidth}\raggedright\strut
Weight at age\strut
\end{minipage} & \begin{minipage}[t]{0.17\columnwidth}\raggedright\strut
Literature\strut
\end{minipage} & \begin{minipage}[t]{0.17\columnwidth}\raggedright\strut
Literature\strut
\end{minipage} & \begin{minipage}[t]{0.17\columnwidth}\raggedright\strut
Literature\strut
\end{minipage} & \begin{minipage}[t]{0.17\columnwidth}\raggedright\strut
Literature\strut
\end{minipage}\tabularnewline
\bottomrule
\end{longtable}

Predator population models were based on either the most recent stock
assessment for the predator or from observational data from the
Northeast US shelf. Herring-predator relationships were based on either
peer-reviewed literature or observational data specific to the Northeast
US shelf. We did not include process or observation error in any of
these modeled relationships. This is obviously unrealistic, but the
primary obective of the herring MSE is to evaluate the effect of herring
management on predators. Leaving out variability driven by anything
other than herring is intended to clarify the effect of herring
managment.

To develop the herring-predator relationship model, specific herring
population characteristics (e.g.~total abundance or biomass, or
abundance/biomass of certain ages or sizes) were related to either
predator growth, predator reproduction, or predator survival. Our aim
was to use information specific to the Northeast US shelf ecosystem,
either from peer-reviewed literature, from observations, or a
combination.

In general, if support for a relationship between herring and predator
recruitment was evident, it was modeled as a predator recruitment
multiplier based on the herring population (\(Hpop_{y}\)) relative to a
specified threshold \(Hthresh\):

\begin{equation}
R_{y+a} = \frac{\alpha B_{y}}{\beta + B_{y}} * \frac{\gamma(Hpop_{y}/Hthresh)}{(\gamma-1)+(Hpop_{y}/Hthresh)} \label{recwithherring_equation}, 
\end{equation}

where \(\gamma\) \textgreater{} 1 links herring population size relative
to the threshold level to predator recruitment.

If a relationship between predator growth and herring population size
was evident, annual changes in growth were modeled by modifying either
the Ford-Walford intercept (\(AnnAlpha\)) or slope (\(AnnRho\)):

\begin{equation}
B_{y+1} = S_{y} (AnnAlpha_{y} N_{y} + FWslope B_{y}) + AnnAlpha_{y}R_{y+1} \label{delaydiffB_equation}, or
\end{equation}

\begin{equation}
B_{y+1} = S_{y} (FWint N_{y} + AnnRho_{y} B_{y}) + FWint R_{y+1} \label{delaydiffB_equation}.
\end{equation}

Finally, herring population size \(Hpop_{y}\) could be related to
predator survival using a multiplier on constant predator annual natural
mortality \(v\):

\begin{equation}
v_{y} =  v e ^ {-(\frac{Hpop_{y}}{Hpop_{F=0}})\delta} \label{varmort_equation},
\end{equation}

where 0 \textless{} \(\delta\) \textless{}1 links herring population
size to predator survival.

After specifying the population model parameters and herring-predator
relationship, we applied the (Hilborn and Walters 2003) equilibruim
calculation for the delay difference model with F=0 to get the unfished
spawners per recruit ratio. This ratio was then used in a new
equilibruim calculation with the current predator exploitation rate to
estimate Beverton-Holt stock recruitment parameters, equilibrium
recruitment and equilibrium individual weight under exploitation. Then,
each model was run forward for 150 years with output from the herring
operating model specifying the herring population characteristics.

\subsection{Highly migratory species}\label{highly-migratory-species}

Bluefin tuna were identified at the stakeholder workshop as the
recommended highly migratory herring predator.

\subsubsection{Tuna population model}\label{tuna-population-model}

Western Atlantic bluefin tuna population parameters were drawn from the
2014 stock assessment (ICCAT 2015), the growth curve from (Restrepo et
al. 2010), and recruitment parameters from a detailed examination of
alternative stock recruit relationships (Porch and Lauretta 2016).
Ultimately, the “low recruitment” scenario was selected to represent
bluefin tuna productivity in the Gulf of Maine, which defines Bmsy as
13,226 t and therefore affects measures of status relative to Bmsy.
Continuation of the current tuna fishing strategy (F\textless{}0.5Fmsy
under the low recruitment scenario) is assumed. All predator population
model parameters are listed in Table 2.

\subsubsection{Herring-tuna relationship
model}\label{herring-tuna-relationship-model}

Tuna diets are variable depending on location and timing of foraging
(Chase 2002, Golet et al. 2013, 2015, Logan et al. 2015), but for the
purposes of this analysis, we assumed that herring is an important
enough prey of tuna to impact tuna growth in the Northeast US shelf
ecosystem. A relationship between bluefin tuna growth and herring
average weight was implemented based on information and methods in Golet
et al. (2015). The relationship between tuna condition anomaly (defined
as proportional departures from the weight-at-length relationship used
in the assessment) and average weight of tuna-prey-sized herring
(\(Havgwt_{y}\), herring \textgreater{}180 mm collected from commercial
herring fisheries) was modeled as a generalized logistic function with
lower and upper bounds on tuna growth parameters:

\begin{equation}
AnnAlpha_{y} = (0.9 FWint) + \frac{(1.1 FWint) - (0.9 FWint)}{1+e^{(1-\lambda)*(100(Havgwt_{y}-T)/T)}} \label{tunagrow_equation},
\end{equation}

where \(\lambda\) \textgreater{} 1 links herring average weight
anomalies to tuna growth.

The inflection point of \(T\) = 0.15 kg average weight matches the 0
tuna weight anomaly in Golet et al. (2015) (p.~186, Fig 2C), and upper
and lower bounds were determined by estimating the growth intercept with
weight at age 10\% higher or lower, respectively from the average weight
at age obtained by applying the length to weight conversion reported in
the 2014 stock assessment (ICCAT 2015) to the length at age estimated
from the Restrepo et al. (2010) growth curve (Fig \ref{herringtuna}).
When included in the model with \(\lambda\) = 1.1 in equation
\ref{tunagrow_equation}, the simulated variation in tuna weight at age
covered the observed range reported in Golet et al. (2015).

\begin{figure}

{\centering \includegraphics[width=300pt]{HerringTechnicalManu_files/figure-latex/unnamed-chunk-1-1} 

}

\caption{Modeled herring average weight-tuna growth relationship \label{herringtuna}}\label{fig:unnamed-chunk-1}
\end{figure}

\subsection{Seabirds}\label{seabirds}

Common terns were identified at the stakeholder workshop as the
recommended seabird herring predator.

\subsubsection{Tern population model}\label{tern-population-model}

There is no published stock assessment or population model for most
seabirds in the Northeast US. Therefore, Gulf of Maine Common and Arctic
tern population parameters were drawn from accounts in the Birds of
North America (Hatch 2002, Nisbet 2002) and estimated from counts of
breeding pairs and estimates of fledgling success summarized by the Gulf
of Maine Seabird Working Group (GOMSWG; data at
\url{http://gomswg.org/minutes.html}), as corrected and updated by
seabird experts from throughout Maine. While we analyzed both Arctic and
Common tern information, the stakeholder workshop identified Common
terns as the example species for modeling, and this species has more
extensive data and a generally higher proportion of herring in its diet
based on that data. Therefore, the model is based on common terns in the
Gulf of Maine.

Adult breeding pairs by colony were combined with estimated productivity
of fledglings per nest to estimate the annual number of fledglings for
each year. A survival rate of 10\% was applied to fledglings from each
year to represent “recruits” to the breeding adult population age 4
and up (Nisbet 2002). This “stock-recruit” information was used to
estimate steepness for the delay difference model based on common tern
information only. Fitting parameters with R nls (R Core Team 2016) had
variable success, with the full dataset unable to estimate a significant
beta parameter (cyan line, Fig. \ref{ternSR}) for common terns, and a
truncated dataset resulting in low population production rates
inconsistent with currently observed common tern trends (bright green
line overlaid with black, Fig. \ref{ternSR}). Therefore, steepness was
estimated to give a relationship (black line, Fig. \ref{ternSR}) falling
between these two lines. The resulting stock recruit relationship set
steepness at 0.26, a theoretical maximum breeding adult population of
45,000 pairs (Nisbet (2002), 1930’s New England population), and a
theoretical maximum recruitment of 4,500 individuals annually
(reflecting approximately a productivity of 1.0 at “carrying
capacity” resulting in a stable population). Average common tern
productivity is 1.02 (all GOM colony data combined). Adult mortality was
assumed to be 0.1 for the delay difference model (survival of 90\%
(Nisbet 2002) for adults).

\begin{figure}

{\centering \includegraphics[width=400pt]{HerringTechnicalManu_files/figure-latex/unnamed-chunk-2-1} 

}

\caption{Stock-recruitment function for Gulf of Maine common terns \label{ternSR}}\label{fig:unnamed-chunk-2}
\end{figure}

The resulting model based on common tern population dynamics in the Gulf
of Maine (with no link to herring) predicts that the population will
increase to its carrying capacity under steady conditions over a 150
year simulation. The actual population has increased at
\textasciitilde{}2\% per year between 1998 and 2015 (Fig.
\ref{terntrend}). Given the lack of detailed demographic information in
the delay-difference model, this was considered a good representation of
the average observed trend in current common tern population dynamics.

\begin{figure}

{\centering \includegraphics[width=400pt]{HerringTechnicalManu_files/figure-latex/unnamed-chunk-3-1} 

}

\caption{Population trends for Gulf of Maine terns, no herring link \label{terntrend}}\label{fig:unnamed-chunk-3}
\end{figure}

\subsubsection{Herring-tern relationship
model}\label{herring-tern-relationship-model}

The relationship between herring abundance and tern reproductive success
was built based on information from individual colonies on annual
productivity, proportion of herring in the diet, and amount of herring
in the population as estimated by the current stock assessment. Since
little of this information has appeared in the peer-reviewed literature,
we present it in detail here. First, productivity information was
evaluated by major diet item recorded for chicks over all colonies and
years. In general, common tern productivity was higher when a
streamlined fish species was the major diet item relative to
invertebrates, but having herring as the major diet item resulted in
about the same distribution of productivities as having hake or
sandlance as the major diet item for these colonies (Fig.
\ref{chickdiet}).

\begin{figure}

{\centering \includegraphics[width=400pt]{HerringTechnicalManu_files/figure-latex/unnamed-chunk-4-1} 

}

\caption{Major diet items for Gulf of Maine tern fledgelings \label{chickdiet}}\label{fig:unnamed-chunk-4}
\end{figure}

Individual colonies showed different trends in number of nesting pairs,
productivity, and proportion of herring in the diet (plots available
upon request). When both Arctic and Common terns shared a colony,
interannual changes in productivity were generally similar between
species, suggesting that conditions at and around the colony (weather,
predation pressure, and prey fields) strongly influenced productivity
rather than species-specific traits. Only two colonies (Machias Seal
Island near the Canadian Border and Stratton Island in southern Maine)
showed a significant positive correlation between the proportion of
herring in the chick diet and productivity. Other islands showed either
non-significant (no) relationships, or in one case (Metinic Island) a
significant negative relationship (Fig. \ref{proddiet}).

\begin{figure}

{\centering \includegraphics[width=400pt]{HerringTechnicalManu_files/figure-latex/unnamed-chunk-5-1} 

}

\caption{Herring proporiton in diet and tern productivity by colony \label{proddiet}}\label{fig:unnamed-chunk-5}
\end{figure}

The estimated population size of herring on the Northeast US shelf had
some relationship to the amount of herring in tern diet at several
colonies (4 of 13 common tern colony diets related to herring Age 1
recruitment, 6 of 13 common tern colony diets related to herring total
B, and 4 of 13 common tern colony diets related to herring SSB; detailed
statistics and plots available upon request). However, statistically
significant direct relationships between herring population size and
tern productivity were rare, with only Ship Island productivity
increasing with herring total B, and Eastern Egg Rock, Matinicus Rock,
Ship, and Monomoy Islands productivity increasing with herring SSB.
Given that Monomoy Island tern chicks consistently received the lowest
proportion of herring in their diets of any colony (0-11\%), we don’t
consider this relationship further to build the model.

Based on tern feeding observations, we would expect the number of age 1
herring in the population to be most related to tern productivity since
that is the size class terns target, but this relationship was not found
in analyzing the data. Herring total biomass was positively related to
tern diets at nearly half of the colonies, and reflects all size classes
including the smaller sizes most useful as tern forage, but was only
directly related to tern productivity at one colony. Herring SSB was not
considered further as an index of tern prey because it represents sizes
larger than tern forage.

To represent the potential for herring to influence tern productivity,
we parameterized a tern “recruitment multiplier” based on herring
assessed total biomass and common tern productivity across all colonies
(except Monomoy where terns eat sandlance). This relationship includes a
threshold herring biomass where common tern productivity would drop
below 1.0, and above that threshold productivity exceeds 1.0 (Fig.
\ref{herrternmod}). The threshold of \textasciitilde{}400,000 tons is
set where a linear relationship between herring total biomass and common
tern productivity crosses productivity=1 (black dashed line in Fig.
\ref{herrternmod}). However, the selected threshold is uncertain because
there are few observations of common tern productivity at low herring
total biomass (1975-1985). The linear relationship does not have a
statistically significant slope; a curve was fit to represent a level
contribution of herring total biomass to common tern productivity above
the threshold. The curve descends below the threshold, dropping below
0.5 productivity at around 50,000 tons and representing the extreme
assumption that herring extinction would result in tern productivity of
0. Although the relationship of tern productivity to herring biomass at
extremely low herring populations has not been quantified, control rules
that allow herring extinction do not meet stated management objectives
for herring, so this extreme assumption for terns will not change any
decisions to include or exclude control rules.

\begin{figure}

{\centering \includegraphics[width=400pt]{HerringTechnicalManu_files/figure-latex/unnamed-chunk-6-1} 

}

\caption{Modeled influence of herring total biomass on tern reproductive success \label{herrternmod}}\label{fig:unnamed-chunk-6}
\end{figure}

When included in the model using \(\gamma\) = 1.09 in equation
\ref{recwithherring_equation}, this relationship adjusts the modeled
common tern population increase to match the current average increase in
common tern nesting pairs observed in the data (Fig.
\ref{terntrendwherring}). There is still considerable uncertainty around
this mean population trajectory which cannot be reflected in our simple
model.

\begin{figure}

{\centering \includegraphics[width=400pt]{HerringTechnicalManu_files/figure-latex/unnamed-chunk-7-1} 

}

\caption{Population trends for Gulf of Maine terns with simulated herring-common tern productivity relationship \label{terntrendwherring}}\label{fig:unnamed-chunk-7}
\end{figure}

\subsection{Groundfish}\label{groundfish}

Because no specific groundfish was identified as a representative
herring predator during the stakeholder workshop, the first decision was
which groundfish to model. Annual diet estimates (based on sample sizes
of \textasciitilde{}100+ stomachs) are available for the top three
groundfish predators of herring (those with herring occurring in the
diets most often in the entire NEFSC food habits database): spiny
dogfish, Atlantic cod, and silver hake. Cod and spiny dogfish were
considered first because their overall diet proportions of herring are
higher, and because silver hake has the least recently updated
assessment. Diet compositions by year were estimated for spiny dogfish,
Georges Bank cod, and Gulf of Maine cod to match the scale of stock
assessments. Full weighted diet compositions were estimated, and suggest
considerable interannual variability in the herring proportion in
groundfish diets (filled blue proportions of bars in Fig.
\ref{gfishdiets}).

\begin{figure}

{\centering \includegraphics{HerringTechnicalManu_files/figure-latex/unnamed-chunk-8-1} 

}

\caption{Annual diet compositions for major groundfish predators of herring estimated from NEFSC food habits database \label{gfishdiets}}\label{fig:unnamed-chunk-8}
\end{figure}

Some interannual variation in diet may be explained by changing herring
abundance. Dogfish and both cod stocks had positive relationships
between the amount of herring observed in annual diets and the size of
the herring population according to the most recent assessment
(statistics and plots available upon request). This suggests that these
groundfish predators are opportunistic, eating herring in proportion to
their availability in the ecosystem. However, monotonically declining
cod populations for both GOM and GB cod stocks resulted in either no
herring-cod relationship, or a negative relationship between herring
populations and cod populations (Fig. \ref{gfishBherrdiet}). Only
dogfish spawning stock biomass had a positive relationship with the
proporiton of herring in dogfish diet. Therefore, we selected dogfish as
the groundfish predator for modeling.

\begin{figure}

{\centering \includegraphics[width=400pt]{HerringTechnicalManu_files/figure-latex/unnamed-chunk-9-1} 

}

\caption{Relationship of assessed groundfish spawning stock biomass (SSB) with the proportion of herring in diet \label{gfishBherrdiet}}\label{fig:unnamed-chunk-9}
\end{figure}

\subsubsection{Dogfish population model}\label{dogfish-population-model}

The dogfish model stock recruitment function, initial population, and
annual natural mortality were adapted from information in (Rago et al.
1998, Rago and Sosebee 2010, 2013, Bubley et al. 2012). Due to
differential growth and fishing mortality by sex, our model best
represents female dogfish (a split-sex delay difference model was not
feasible within the time constraints of this MSE). Further, dogfish
stock-recruit modeling to date based on Ricker functions (Rago and
Sosebee 2010) captures more nuances in productivity than the
Beverton-Holt model we used. Our recruitment parameterization reflects a
stock with generally low productivity and relatively high resilience,
which we recognize is a rough approximation for a species such as
dogfish. The annual fishing exploitation rate applied is average of the
catch/adult female biomass from the most recent years of the 2016 data
update provided to the Mid-Atlantic Fishery Management Council (Rago
pers comm).

\subsubsection{Herring-dogfish relationship
model}\label{herring-dogfish-relationship-model}

There was a weak positive relationship between dogfish total biomass and
herring total biomass from the respective stock assessments (Fig.
\ref{pupherring}), but no clear relationship between dogfish weight or
dogfish recruitment and herring population size. During the recent
period of relatively low dogfish recruitment (1995-2007), there is a
positive relationship between dogfish pup average weight and herring
proportion in diet, suggesting a potential growth and or recruitment
mechanism; however this relationship does not hold throughout the time
series (Fig. \ref{pupherring}).

\begin{figure}

{\centering \includegraphics[width=.49\linewidth]{HerringTechnicalManu_files/figure-latex/unnamed-chunk-10-1} \includegraphics[width=.49\linewidth]{HerringTechnicalManu_files/figure-latex/unnamed-chunk-10-2} 

}

\caption{Dogfish population relationships with herring total biomass (left) and herring proportion in diet (right) \label{pupherring}}\label{fig:unnamed-chunk-10}
\end{figure}

Therefore, to simulate a potential positive relationship between herring
and dogfish, we assumed that dogfish survival increased (natural
mortality was reduced) by an unspecified mechanism as herring abundance
increased (Fig. \ref{herringdogfish}). Because dogfish are fully
exploited by fisheries in this model, the impact of this change in
natural mortality on total survival has small to moderate benefits to
dogfish population numbers and biomass. Using a \(\delta\) = 0.2 in
equation \ref{varmort_equation} results in weak increases in dogfish
biomass with herring abundance consistent with observations.

\begin{figure}

{\centering \includegraphics[width=300pt]{HerringTechnicalManu_files/figure-latex/unnamed-chunk-11-1} 

}

\caption{Modeled herring relative population size-dogfish natural mortality relationship \label{herringdogfish}}\label{fig:unnamed-chunk-11}
\end{figure}

\subsection{Marine mammals}\label{marine-mammals}

Because no specific marine mammal was identified as a representative
herring predator in the stakeholder workshop, as with groundfish, the
first decision was which marine mammal to model. Diet information for a
wide range of marine mammals on the Northeast US shelf suggests that
minke whales, humpback whales, harbor seals, and harbor porpoises have
the highest proportions of herring in their diets (Smith et al. 2015),
and therefore may show some reaction to changes in the herring ABC
control rule.

While some food habits data existed for marine mammals, consultation
with marine mammal stock assessment scientists at the Northeast
Fisheries Science Center confirmed that no data were available to
parameterize a stock-recruitment relationship for any of these marine
mammal species in the Northeast US region, and no such information was
available in the literature for stocks in this region. Althouth it may
be possible to develop stock-recruitment models for one or more of these
species in the future, it was not possible within the time frame of the
herring MSE. Therefore, we were unable to model marine mammals within
the same framework as other predators.

Potential effects of changes in herring production and/or biomass on
marine mammals were instead evaluated using an updated version of an
existing food web model for the Gulf of Maine (Link et al. 2006, 2008,
2009) and incorporating food web model parameter uncertainty. Overall,
food web modeling showed that a simulated increase in herring production
in the Gulf of Maine may produce modest but uncertain benefits to marine
mammal predators, primarily because increased herring was associated
with decreases in other forage groups also preyed on by marine mammals.
Please see Appendix 1 of this document for full analyses and results.

\subsection{Summary of predator model input
parameters}\label{summary-of-predator-model-input-parameters}

Table 2. Predator model input parameters

\begin{longtable}[]{@{}llll@{}}
\toprule
Parameter & Tuna & Tern & Dogfish\tabularnewline
\midrule
\endhead
Numbers or Weight? & Weight & Numbers & Weight\tabularnewline
Unfished spawning pop & 6.69E+04 & 45000 & 300000\tabularnewline
Steepness \emph{h} & 1.0 & 0.26 & 0.97\tabularnewline
Annual nat. mortality \emph{v} & 0.14 & 0.1 & 0.092\tabularnewline
Annual exploitation \emph{u} & 0.079 & 0.00 & 0.092\tabularnewline
Growth intercept \emph{FWint} & 0.020605 & 0.00015 &
0.000278\tabularnewline
Growth slope \emph{FWslope } & 0.9675 & 0.0 & 0.9577\tabularnewline
Initial abundance & 111864 & 3000 & 49629630\tabularnewline
Initial biomass & 27966 & 1.5 & 134000\tabularnewline
Recruit delay (age) \emph{a} & 1 & 4 & 10\tabularnewline
Prey-recruitment link & 1 (off) & 1.09 & 1 (off)\tabularnewline
Prey-mortality link & 0 (off) & 0 (off) & 0.2\tabularnewline
Prey-growth link & 1.1 & 1 (off) & 1 (off)\tabularnewline
\bottomrule
\end{longtable}

\subsection{\texorpdfstring{\textbf{Performance
Metrics}}{Performance Metrics}}\label{performance-metrics}

\textbf{Herring}

For each combination of control rule shape and operating model, 100
simulations were conducted, each for 150 years. Preliminary simulations
suggested that this number of simulations and years was sufficient for
results to be insensitive to starting conditions and short-term dynamics
caused by auto-correlated processes. Median SSB,
\(\frac{SSB}{SSB_{F=0}}\), \(\frac{SSB}{SSB_{MSY}}\), \emph{yield},
\(\frac{yield}{MSY}\), biomass of herring dying due to M, and the
proportion of the herring population comprised of age-1 fish over the
last 50 years of each simulation were recorded as performance metrics.
Additional performance metrics included the proportion of the last 50
years of each simulation with \(SSB < SSB_{MSY}\),
\(SSB < \frac{SSB_{MSY}}{2}\) (i.e., proportion of the last 50 years
that are overfished), \(SSB < 0.3SSB_{F=0}\), \(SSB < 0.75SSB_{F=0}\),
fully-selected \(F > F_{MSY}\) (i.e., proportion of the last 50 years
that overfishing occurred), and \emph{Q}=0 (i.e., proportion of the last
50 years that the fishery was closed). Interannual variation in yield
(\emph{IAV}) was also recorded over the last 50 years of each simulation
(Table XX T10). These performance metrics were highlighted to be of
interest at the stakeholder workshops.

\textbf{Predators} Predator performance metrics included those described
at the stakeholder workshop, as well as several others drawn from MSE
best practices (Punt et al. 2016b). Each of the simulated herring time
series for every operating model, control rule, and simulation was
passed to each predator model, resulting in outputs as described below
using the equations above.

All predator performance metrics were calculated based on the final 50
years of each replicate simulation. For all metrics other than
``frequency of good status'' metrics, we took the median value over the
final 50 years of each replicate simulation. Then, the 25th percentile,
the median, and the 75th percentile of these 100 medians were calcualted
to represent the performance metric for a particular control rule.
Results reported here focus on the median.

\subsubsection{Biomass, Abundance,
Recruitment}\label{biomass-abundance-recruitment}

Population abundance and recruitment in numbers were output for all
modeled predators. Population biomass was output for tuna and dogfish.
These quantities were directly output by the models.

\subsubsection{Predator condition}\label{predator-condition}

Stakeholders expressed interest in predator condition for fish and
marine mammal predators at the first workshop. While delay difference
models do not track individuals or age cohorts, a measure of population
average weight (population biomass/population numbers) was output for
tuna and dogfish.

\subsubsection{Predator productivity}\label{predator-productivity}

Productivity, the number of fledglings per breeding pair, was output for
the tern model. Productivity was calculated as recruitment times 10 (to
account for the 10\% survival rate of fledglings to adults) divided by
tern abundance 4 years earlier in the simulation.

\subsubsection{Status relative to
thresholds}\label{status-relative-to-thresholds}

Stakeholders were interested in different measures of population status
depending on the predator. For commercially fished species, status
relative to current management reference points was preferred. Tuna and
dogfish biomass was divided by a biomass reference point specified in
current stock assessments: tuna \(SSB_{MSY}\) was 13226 (ICCAT 2015),
and dogfish \(SSB_{MSY}\) was 159288 (Rago and Sosebee 2010). Because
dogfish were fully exploited in our model, they did not reach
\(SSB_{MSY}\), so we also evaluated status relative to 0.5 \(SSB_{MSY}\)
(``overfished''). Tuna condition status was assessed by dividing the
output population average weight with the equilibrium average weight.
Common tern colonies are managed to improve productivity, so
stakeholders suggested that a common tern productivity level of 0.8
would be a minimum threshold, while a productivity of 1.0 would be a
target. In addition, total population status was measured relative to
current population numbers using the rationale that maintaining at least
the current population was desirable. The average common tern population
of nesting pairs (including Monomoy) from 1998-2015 was 16000.

\subsubsection{Frequency of good status}\label{frequency-of-good-status}

Evaluating the frequency of desirable or undesirable states over the
course of a simulation is suggested by Punt et al. (2016b). We
calculated two metrics for each of the status determinations. First, we
calculated the minimum number of years in any individual simulation that
a metric was above a given threshold. This is a ``worst case scenario''
metric. Second, we calculated the median proportion of years across all
simulations for a control rule that were above the threshold. This is an
``average performance'' metric addressing how often good status is
maintained.

\textbf{An Economic Model of the Predators}

We didn't do one.

\textbf{An Economic Model of the Herring Fishery}

\emph{Basics and Intro--} There are many economic methods that can
inform ecosystem approaches to fisheries management\footnote{Edwards et
  al. (2004) and Jin et al. (2016) illustrate a portfolio approach. Jin
  et al. (2003) and Jin et al. (2012) link regional economic models to
  ecosystem models. Tschirhart (2000) and Finnoff and Tschirhart (2003)
  link structural economic models of constrained optimization to
  ecosystem models.}. Kirkley et al. (2011) use a static input-output
model to simulate the effects of changes in herring quotas and predator
biomass levels on the New England economy. Their analysis suggested that
the effects of changes in herring catch on other segments of the economy
are quite small; we, therefore confined the analysis to the herring
fleet. Lehuta et al. (2013) also construct an coupled
ecological-economic model of herring under the assumption of competitive
output markets for herring and zero economic profit. We did not make
this assumption.

The economic model of the herring fishery converts \textit{yield} from
the herring component into Gross Revenues (GR) and Net Operating
Revenues (NR). There are two fleets,trawl and purse seine, that are
assumed to have the ability to catch 70 and 30\% of the \textit{yield}
respectively\footnote{The fishery is managed with four sub-ACLs and the
  purse seine fleet can only fish in one of the four areas. This 70/30
  split corresponds to recent history.}. The midwater trawl, paired
midwater trawl, and bottom trawl are all aggregated into the trawl
fleet. Gross Revenue, Net Revenue, and the constraints on harvest can be
represented as (\(y\) subscripts omitted here for simplicity):

\begin{align}
GR&=p(q^t+q^s)q^t+p(q^t+q^s)q^s\\
NR&=GR-c^t(q^t)-c^s(q^s)
\end{align}

where \(q^i\) is the quantity landed for fleet \(i\), \(c^i(q^i)\) is
cost function for fleet \(i\), and \(p(\cdot)\) is a function that
relates total landings to prices.\\

\begin{align}
\label{optimization_problem}
\max_{q_i} NR^i&=p(q^t+q^s)q^i - c^i(q^i)\\
q^s &\leq .3yield; \hspace{2mm} q^t \leq .7yield 
\end{align}

The optimization problem in equation \ref{optimization_problem} contains
two embedded assumptions: total catch is less than or equal to Yield and
that a fleet may catching less than it's fraction of yield (presumably
because it may be more profitable to select a lower level of landings).

\emph{Production and Costs--} Economic data collected from 2011-2015 by
the Northest Fisheries Observer Program (NEFOP) are used to construct
average daily costs for the trawl and purse seine fleets. Fuel prices
were much lower in 2011-2014 compared to 2015. We adjusted fuel prices
to the 2011-2014 average; sensitivity analysis was performed by settting
fuel prices to the 2015 levels but results not reported here. Other
costs of fishing included water, oil, and damage costs. Crew pay and
fixed costs were not included.

We construct average catch per day fished for each fleet from the Vessel
Trip Report databases over the same time period. The trip lengths in the
VTR and observer data were very similar. This allows us to construct the
average cost of catching a metric ton of herring for the trawl and purse
seine fleet (\(c^t\) and \(c^s\) respectively). We assume that the
average cost is equal to the marginal cost for each fleet. These figures
are presented in Table \ref{costs_combined}.

\begin{table}[htpb]
\begin{centering}
\begin{tabular}{l r r r r r r r r}\hline
    &   Marginal &Catch &   Trip length&\multicolumn{3}{c}{Cost per day}    &   Observed    &   VTR \\
Year    &   Cost (\$/mt)& (mt/day) &days    &   Low &   High    &   Raw &    trips  &    trips  \\\hline
Seine   &   14.27 & 87.5& 1.0&  1,249   &   1,664   &   1,396   &   207 &   1,413   \\
Trawl   &   62.43 & 61.4    &2.9&   3,833   &   5,009   &   4,315   &   573 &   2,005   \\\hline
\end{tabular}
\caption{Marginal cost of 1mt of herring, daily catch, trip length, and cost per day for the purse seine and trawl fleets averaged over 2011-2015. \label{costs_combined}}
\end{centering}
\end{table}

\emph{Prices--} Annual prices were constructed from NMFS dealer data for
1982 through 2016\footnote{Prices have been normalized to 2015 real
  dollars using the Bureau of Labor Statistics (BLS) Producer Price
  Index (PPI) for Unprocessed and Packaged Fish (WPU0223). Because
  Atlantic Herring was not federally managed prior to the implementation
  of the Herring FMP in 2000; the NMFS dealer databases may not contain
  all landings prior to this time. The ME DMR data do not contain prices
  but is a census of landings.}. Annual landings were constructed from
the processed ME DMR landings dataset for the same time period.
Exploratory analysis suggested both a regime change in the mid-1990s and
likely non-stationarity of both landings and prices. We used the Bounds
testing methodology developed by Pesaran et al. (2001) test for a
long-run relationship between prices and quantities. This method does
not require pretesting for stationarity; however, the test statistic
does have an inconclusive zone in which knowledge of stationarity would
required. A long-run relationship between prices and landings can be
modeled as:

\begin{equation}
\label{ARDL}
p_{y}= c+\sum_{i=1}^p a_i p_{t-i}+ \sum_{i=0}^n b_i q_{y-i} + e_{t},
\end{equation}

or equivalently as an error correction model (ECM):

\begin{equation}
\label{ECM}
\Delta p_y= \gamma+ \alpha_1 p_{y-1} + \sum_{i=1}^p \theta_i \Delta p_{y-i}+ \sum_{i=0}^n  \beta_i \Delta q_{y-i} + e_{y},
\end{equation}

where \(\Delta\) is the first-differences operator (Pesaran et al.
2001). The \(\gamma\) parameter must also be restricted
(\(\gamma=c / \alpha_1\)) for these equations \ref{ARDL} and \ref{ECM}
to be equivalent. Pesaran et al. (2001) tests the null of no long-run
relationship using a joint F-test that the \(\alpha_1\) and \(\beta_i\)
parameters in equation \ref{ECM} are non-zero; however, the F-statistic
has a non-standard distribution with an inconclusive area.

Equation \ref{ECM} was first estimated on the full 1982-2015 dataset;
model selection criteria indicated (p=4,n=0) was preferred and that
prices were not affected by quantities. We suspect this is likely caused
by a combination of overfitting of the model and a regimes shift evident
in the exploratory graphs. Rather than explore a regime switching model,
we simply estimated equation \ref{ECM} with \(p=1, n=0\) on a subset of
the data (1995-2015). If there was a regime shift, the current regime is
more likely to be similar to the future. Models estimated in natural
logarithms and in levels fit well. The PSS F-statistics of 9.34 and
10.20 are above the upper critical value, strongly suggesting a long-run
relationship between prices and quantities\footnote{As a robustness
  check, we also tried varying the first (1996) and last (2016) year
  included in the dataset. This did not change the estimated results
  substantially. We also estimated a short-run relationship between
  prices and quantities in which \(\Delta p_y\) was regressed on
  \(\Delta q_y\). The short-run effects were qualitatively similar to
  the ``long-run'' model in Table \ref{ardl_regression}.} (Table
\ref{ardl_regression}). We also present the results of the ARDL(1,0)
formulation because it is a bit easier to interpret. Coefficients from
the ``level'' equation (Column 1 of Table \ref{ardl_regression}) are
used in the simulation.

\begin{table}[htbp]
\begin{center}
\begin{tabular}{lcc} \hline
Model       &   (1) &   (2)     \\
    &   price & \textit{ln} price   \\\hline
Price$_{t-1}$       &   0.646***    &   0.666***    \\
        &   -0.0937 &   -0.0935 \\
Quantity        &   -1.194***   &   -0.395***   \\
        &   -0.277  &   -0.0956 \\
Constant        &   217.8***    &   6.423***    \\
        &   -48.16  &   -1.484  \\
Observations        &   21  &   21  \\
R-squared       &   0.906   &   0.906   \\
BGp     &   0.34    &   0.547   \\
BGF     &   1   &   0.4 \\
PSS F-statistic &   10.2    &   9.34    \\\hline
1\% Critical values &\multicolumn{2}{c}{[6.84, 7.84]} \\\hline
\multicolumn{3}{c}{ Standard errors in parentheses} \\
\multicolumn{3}{c}{ *** p$<$0.01, ** p$<$0.05, * p$<$0.1} \\\hline
\end{tabular}
\caption{Regression results from the ``ARDL'' specification (Equation \ref{ARDL}), PSS Bounds test and associated critical values.  1995-2015 data only. The explanatory variables enter in levels in the first column and natural logarithms in the second column.  \label{ardl_regression}}
\end{center}
\end{table}

\emph{Net Operating Revenues--} The simulation of Gross operating
revenues occurs in a few steps. Herring prices in a year are simulated
using equations \ref{optimization_problem} and \ref{ARDL}, parameters
from Table \ref{ardl_regression} previous year prices (initialized to
the 2011-2015 average for the first year) under the assumption that both
fleets combine to land the entire Yield. If simulated prices are higher
than the marginal cost of the trawl fleet(\$63.24/mt), both fleets are
assumed to catch the entire Yield. Otherwise, we use equations
\ref{optimization_problem} and \ref{ARDL} to solve for quantity landed
by the trawl fleet when the purse seine fleet lands 30\% of yield. If it
optimal for the trawl fishery to land nothing, we use equations
\ref{optimization_problem} and \ref{ARDL} to find the purse seine's
optimal amount of landings\footnote{Because the marginal costs for the
  purse seine are always less than the marginal costs of the trawl
  fishery, any landings by the trawl fleet imply the purse seine fleet
  is landing 30\% of the yield.}. Net revenues to the fishery can then
be calculated directly from equation \ref{optimization_problem}.

\emph{Stability --} Stakeholders were interested in understanding
stability of the herring industry. We computed Interannual Variation
(IAV) of Net Revenues. While IAV can summarize variability, it does not
provide insight into the equilibrium properties of the time series. An
additional performance metric, stationarity, is used to assess the
stability of net revenues over the terminal period. For each simulation,
we perform an econometric test of stationarity (Dickey and Fuller 1979),
by estimating

\begin{equation}
\label{DF_estimated}
\Delta NR_{y} = \beta  NR_{y-1} + \xi_1 \Delta NR_{y-1} + \varepsilon 
\end{equation}

Econometric evidence that \(\beta=0\) is evidence of a unit root
(nonstationarity) while statistical rejections of \(H_0: \beta=0\) in
favor of \(H_A: \beta<0\) is evidence of stationarity of the
time-series. The results of these tests are summarized in two ways.
First, we classify a simulation as stationary if we reject the null
\(H_0: \beta=0\) in favor of \(H_A: \beta<0\) at the 10\% significance
level, then report the percentage of simulations that are stationary. We
also use a weighted Z-score method to combine these trials (Whitlock
2005) to test the null hypothesis that a particular control rule
implemented on a simulated stock of herring does not produce a stable
equilibrium. Defining \(\phi\) as the standard normal cumulative
distribution function, we construct:

\begin{align}
Z  &= \frac{\sum_{i=1}^k \phi^{-1} (1-p_i)}{\sqrt(k)}
\end{align}

\(Z\) has a standard normal distribution under the null hypothesis of no
stable equilibrium. We report the p-value associated with rejecting the
null that particular control rule leads to non-stationary trajectory of
revenue.

\emph{Caveats to the Herring Fishery Economic model--} We did not
include fixed costs. If firms do not enter or exit, then the exclusion
of fixed costs from the model has minimal qualitative effects. The
stationarity metric would be unaffected; however, IAV constructed
without fixed costs will be smaller than the true IAV that contains
fixed costs. A bigger problem is that, in the long run, firms can enter
and exit this industry. If firms can enter and exit, then exclusion of
fixed costs from the model could result in misleading recommendations.
We did not model firm entry and exit; this is a difficult decision to
model. Economic theory suggests that firms will enter (exit) if they
anticipate large positive (negative) profits over a particular planning
horizon; however, understanding exactly how these decisions are made was
not possible. Less than three-quarters of the permitted herring vessels
were active in the fishery, suggesting that firms could enter if future
profitability is expected to be high.

We also assume that marginal costs are equal to average variable costs,
constant for each fleet, and do not depend on the level of biomass. A
more rigorous approach might include estimating a (economic) production
function for the herring fishery; this was not done in the interest of
time. It is difficult to predict how estimating a true cost function and
integrating those results would change the results of the study.

Catch in the economic model can be different from the yield (from the
herring model). This is frequently handled as symmetric implimentation
error in the fisheries literature (as it is in this model). However, the
market model suggests that this is not so simple.

\emph{Caveats to the Economic model--} The largest limitation and caveat
to the economic model is that only the herring fishery is quantitatively
modeled. We discuss how other economic groups could be modeled here.

\emph{Direct Consumers of landed herring--} We note that we have
estimated a price-quantity relationship and not necessarily a demand
curve for herring. If we believe that the lobster bait market is well
functioning, then consumer welfare measures could be determined from a
demand curve for herring. Rigorous estimation of a demand curve for
herring likely requires modeling goods are substitutes for herring,
possibly including mackerel, menhaden, squid, and substitute bait. This
was not done due to time pressures.

\emph{Humans and Herring in the Ecosystem--} Finnoff and Tschirhart
(2003) and Brown et al. (2005) describe methods to trace the effects of
changes in harvest or biomass of one species through an ecosystem. The
predator section models a few representative consumers of live herring:
terns, tuna, whales, and predatory fish. Ecosystem Valuation methods
could be used to measure changes in outcomes for those species in dollar
value (Loomis and White 1996, Richardson and Loomis 2009, Lew 2015).
People may derive value from changes in the status of these predators
through either use or non-use values. For example, people may directly
value higher abundances of an animal or protection of an endangered
species, even if they have no plans to watch or view them (Lew et al.
2010, Lew and Wallmo 2017). Quantifying these values typically is done
using stated preference methods with data collected using surveys; these
studies are costly and time consuming to develop and conduct rigorously.
Benefit transfer, a method in which valuation from previous studies is
applied to a new study area,may be the only way to overcome these
barriers (Navrud and Ready 2007, Johnston and Rosenberger 2010).

Consumers may also derive use value from changes in the modeled species.
For example, changes in whale populations may change outcomes for
whale-watching customers (Larson et al. 2004). This type of value can be
quantified using both stated and revealed preference data, both of which
typically require collection of survey data; these studies are also
costly and time consuming to develp and conduct rigorously. Perhaps more
discouragingly, the precise good being valued (say, a doubling of the
tern population) needs to be known quite early in the research process.
In this application, development of a valuation survey for terns would
not have been able to start until after the predator modeling was nearly
complete. Benefit transfer may be the only way to value some of the use
values on the timeframe required by NEFMC.

Another avenue for use value to arise is though changes in costs,
catches, or prices in the associated commercial fishery. For example,
increases in biomass of spiny dogfish could lead to both higher quotas
and lower costs to catch more abundant fish. Examining changes in costs
would require an economic model of production for the spiny dogfish
fishery (similar to the model not used for the herring fleet) (Holland
and Sutinen 1999, Hutniczak 2014, Reimer et al. 2017). Changes in
product quality could result in higher prices (Larkin and Sylvia 1999,
Asche et al. 2015). For example, because larger tuna receives higher
prices, the effects of changes in average weight could be deduced from
existing hedonic models (McConnell and Strand 2000, Carroll et al.
2001).

Note that increases in the biomass of a particular predator could be a
net ``bad'' for society. For example, an increase in the biomass of a
predator that is low-valued but skilled at consuming herring could
result in disproportionate increases in that low-valued predator. If
that low-valued predator is not a complete specialist (in consuming
herring), it may also drive down the biomass of high-valued predators.
The ability to manipulate the ecosystem with a prey-level ABC control
rule to achieve desirable outcomes depends on the rates at which these
increases in prey are converted into social utility. This conversion
depends on the ecosystem technology (conversion of prey into additional
biomass of high- and low-valued predators), human technology (conversion
of prey and predator biomass into catch or tourism), and human
preferences (converting catch or tourism into utility). Despite our
current efforts, many of these technologies are not particularly well
understood at this time.

One of the difficulties with doing an economic extension that models the
prey species is that it's difficult to know which margin (dimension) to
address. For example, we didn't know that the tuna model would result in
changes in average weight and not abundances. If we had tried to model
how changes in abundance affect affect costs of harvesting tuna, this
would have not been a fruitful use of time.

\emph{Random thoughts. this probably belongs in the non-tech manuscript
--} A key factor in management performance is the process by which
management tools are developed and implemented.

\begin{quote}
One approach to the creation of regulatory legitimacy is to structure
the management process around user participation in decision-making.
User participation addresses the twin problems of integrating human and
ecological systems and bringing individual users into the definition of
the collective good. Participation offers certain benefits to the
management process but it also carries costs. Whether it is worthwhile
for fishery management depends on the relative magnitudes of those
benefits and costs. (Hanna 1995)
\end{quote}

\begin{quote}
What we ask of fishery governance is that it coordinate institutional
rules and individual actions by performing certain functions:
incorporate multiple objectives representing different types of
conservation and use; bring the time horizons of private individuals
into line with those of the public; send signals of resource scarcity
and enable effective adaptive responses; promote legitimacy by
reflecting accepted norms of equity and by controlling harmful
opportunism; contain both the level and distribution of transactions
costs.These functions are expressions of Williamson's organizational
imperative to minimize costs, accommodate limited information, and
safeguard against harmful opportunism. (Hanna 1999)
\end{quote}

Fishery management must, among other things, incorporate multiple
objectives, promote legitimacy by reflecting accepted norms, and have
low transactions costs (Hanna 1999). User participation in
decision-making is a way to increase regulatory legitimacy (Hanna 1995).
One drawback to user-participation is that meetings tend to attract
participants with the most extreme views (Osborne et al. 2000, Turner
and Weninger 2005).

\begin{center}\rule{0.5\linewidth}{\linethickness}\end{center}

\section{Results}\label{results}

\textbf{Effects of Operating Models}

The effect of growth on the relative performance of control rules was
relatively minor for all metrics than the effect of \emph{M} and
steepness (INSERT A PLOT - maybe boxplot among all CRs?). Furthermore,
performance of control rules was similar between the operating models
with different \emph{M} and steepness values when reported relative to
biological reference points (e.g., \(\frac{SSB}{SSB_{MSY}}\),
\(\frac{yield}{MSY}\)), but the scale of results in absolute values was
generally different (INSERT PLOT). Consequently, results were only
reported in relative units for the low \emph{M}/high steepness, and slow
growth operating model, with the understanding that the operating models
differ in meaningful ways (e.g., different \emph{MSY}) if metrics are
reported in absolute units.

Positive bias in assessment errors resulted in more control rules that
led to lower amounts of biomass, more frequent biomass levels below
\(SSB_{MSY}\), higher \emph{IAV}, but a generally similar range of yield
(BOXPLOT). \#\#Perhaps also add summary effect and boxplot for predator
and econ metrics\#\#. For the sake of brevity, additional results and
tradeoffs (see below) were only reported using an unbiased assessment
error operating model.

\textbf{Tradeoffs}

\emph{Herring}

Several of the herring performance metrics were redundant and exhibited
similar tradeoffs. So, only a few tradeoffs that are commonly of
interest were reported. Namely, \(\frac{yield}{MSY}\) vs.
\(\frac{SSB}{SSB_{MSY}}\), \(\frac{yield}{MSY}\) vs. \emph{IAV},
\(\frac{yield}{MSY}\) vs.~the frequncy that \emph{Q}=0, and
\(\frac{yield}{MSY}\) vs.~the frequncy that
\(SSB < \frac{SSB_{MSY}}{2}\).

Biomass based control rule variants without a restriction on interannual
change in the quota provided a broader range of \(yield\) and \(SSB\)
than other control rules (). At similar levels of yield, using three or
five year blocks for the biomass based control rule produced more
control rule shapes with less \(SSB\), such that the short-term
stability of such quota blocks comes at the potential cost of less
\(SSB\).

Biomass based control rule variants without a restriction on interannual
change in the quota provided a broader range of \(yield\) and \emph{IAV}
than other control rules (). Restricting the interannual change in the
quota for the biomass based control rule or using one of the constant
catch variants produced options with lower \emph{IAV} than the biomass
based alternatives, but biomass based options were available that could
attain higher yields that the other control rules could not. Thus,
restricting the interannual change in the quota or using constant catch
variants can produce more stable yields than alternatives, but at the
cost of yield.

Tradeoffs for \(\frac{yield}{MSY}\) vs.~the frequncy that \emph{Q}=0
were similar to that of \(\frac{yield}{MSY}\) vs. \emph{IAV}. Biomass
based control rule variants without a restriction on interannual change
in the quota provided a broader range of results than other control rule
alternatives. Restricting the interannual change in the quota for the
biomass based control rule or using one of the constant catch variants
produced less, and often no fishery closures, that could not be achieved
by the biomass based control rules, but at the cost of not offering
options that could achieve yields as high as biomass based control
rules.

All of the control rule alternatives offered options that had near zero
frequency of \(SSB < \frac{SSB_{MSY}}{2}\). At similar levels of yield,
using three or five year blocks for the biomass based control rule
produced more control rule shapes with higher frequency of
\(SSB < \frac{SSB_{MSY}}{2}\), such that the short-term stability of
such quota blocks comes at the potential cost of more frequently
dropping to relatively low levels of biomass. While restricting the
interannual change in the quota or using one of the constant catch
variants offered options that had near zero frequency of
\(SSB <\frac{SSB_{MSY}}{2}\), these options could not achieve as high of
yield as biomass based options that also had near zero frequency of
\(SSB <\frac{SSB_{MSY}}{2}\). Some constant catch options also had near
100\% frequency of \(SSB <\frac{SSB_{MSY}}{2}\).

\section{Discussion}\label{discussion}

Relative results insensitive to OM, which has been found before.

CC options not flexible and can usually be outperformed by other CRs,
unless stability is a heavily preferred objective.

We only evaluated a 15\% restriction on interannual change in the quote
for a BB control rule because this was a value preferred by
stakeholders. Exploring a broader range of percentages, however, may
result in a broader range of performance that may sacrfice less yield
(than our 15\%) but still gain some stability. See Laurie Kell's stuff;
Deroba and Bence.

If short-term stability is an objective, then using quota blocks may be
effective. Some BB CR options with quota blocks could produce similar
performance as annual changes in the quota. Although, at similar level
of yields, the quota blocks produced more options with lower biomass.
Thus, a closed-loop simulation should be used to evaluate the changes in
relative performance between applying a CR annually or using a quota
blocks.

We assumed reference points were known without error, and so our results
may not generalize to cases where such values are poorly estimated.
Incorporating error in the estimation of reference points within
simulations is an area of active research, but is not straightforward.
For example, the reference points (e.g., MSY) depend on life history
traits that may change through time, and so how to accommodate time
varying life history in the calculation of the reference values would
need to be considered as part of a management strategy. See Deroba and
Bence.

We did not incorporate a full stock assessment due to time contraints.
Discuss pros and cons; this is part of the reality and dream, as is the
above paragraph.

MICE can be a reasonable approach to conducting science to meet
relatively short managment time frames. Scientists may have to sacrifice
some dreams in order to meet realities of doing research in support of
managment. Review other MICE literature. Discuss MICE in the context of
multi-species/ecosystemy based management, and whether this even
qualifies as ecosystem based.

Dive into the ``forage'' fish debate!? While herring may constitute
``large'' fractions of diet, our population level analyses suggest only
relatively weak relationships. Likely due to alternative prey. Herring
abundance may also not be the most important aspect for predators, as
their condition seems to be more directly linked to tuna condition, and
tuna condition has been good at low herring N and visa versa.

An open stakeholder driven process produced dividends for this analysis
in the form of improved relationships and data. Greater attention should
be paid to what stakeholders are involved and how they are involved in
MSE processes. Probably leave a short paragraph assuming Feeney et al.
accepted.

Econ something. Sorry M-Y, but that's the best I can do getting started
on Econ discussion topics.

\begin{center}\rule{0.5\linewidth}{\linethickness}\end{center}

\section*{References}\label{references}
\addcontentsline{toc}{section}{References}

\hypertarget{refs}{}
\hypertarget{ref-Asche2015EconomicFisheries}{}
Asche, F., Chen, Y., and Smith, M.D. 2015. Economic incentives to target
species and fish size: Prices and fine-scale product attributes in
Norwegian fisheries. ICES Journal of Marine Science \textbf{72}(3).
doi:\href{https://doi.org/10.1093/icesjms/fsu208}{10.1093/icesjms/fsu208}.

\hypertarget{ref-Brown2005AFisheries}{}
Brown, G., Berger, B., and Ikiara, M. 2005. A predator-prey model with
an application to Lake Victoria fisheries. Marine Resource Economics
\textbf{20}(3): 221--247.

\hypertarget{ref-bubley_reassessment_2012}{}
Bubley, W.J., Kneebone, J., Sulikowski, J.A., and Tsang, P.C.W. 2012.
Reassessment of spiny dogfish Squalus acanthias age and growth using
vertebrae and dorsal-fin spines. Journal of Fish Biology \textbf{80}(5):
1300--1319.
doi:\href{https://doi.org/10.1111/j.1095-8649.2011.03171.x}{10.1111/j.1095-8649.2011.03171.x}.

\hypertarget{ref-butterworth2007management}{}
Butterworth, D.S. 2007. Why a management procedure approach? Some
positives and negatives. ICES Journal of Marine Science \textbf{64}(4):
613--617. Oxford University Press.

\hypertarget{ref-Carroll2001PricingManagement}{}
Carroll, M.T., Anderson, J.L., and
Mart\textbackslash{}'\textbackslash{}inez-Garmendia, J. 2001. Pricing US
North Atlantic bluefin tuna and implications for management.
Agribusiness \textbf{17}(2): 243--254.

\hypertarget{ref-chase_differences_2002}{}
Chase, B.C. 2002. Differences in diet of Atlantic bluefin tuna (Thunnus
thynnus) at five seasonal feeding grounds on the New England continental
shelf. Fishery Bulletin \textbf{100}(2): 168--180. Available from
\url{http://aquaticcommons.org/15201/} {[}accessed 26 April 2016{]}.

\hypertarget{ref-clark2004conditional}{}
Clark, W.G., and Hare, S.R. 2004. A conditional constant catch policy
for managing the pacific halibut fishery. North American Journal of
Fisheries Management \textbf{24}(1): 106--113. Wiley Online Library.

\hypertarget{ref-deroba2015simulation}{}
Deroba, J., Butterworth, D.S., Methot Jr, R., De Oliveira, J.,
Fernandez, C., Nielsen, A., Cadrin, S., Dickey-Collas, M., Legault, C.,
Ianelli, J., and others. 2015. Simulation testing the robustness of
stock assessment models to error: Some results from the ices strategic
initiative on stock assessment methods. ICES Journal of Marine Science
\textbf{72}(1): 19--30. Oxford University Press.

\hypertarget{ref-Deroba2012Evaluating}{}
Deroba, J.J., and Bence, J.R. 2012. Evaluating harvest control rules for
lake whitefish in the great lakes: Accounting for variable life-history
traits. Fisheries Research \textbf{121}: 88--103. Elsevier.

\hypertarget{ref-Dickey1979DistributionRoot}{}
Dickey, D.A., and Fuller, W.A. 1979. Distribution of the estimators for
autoregressive time series with a unit root. Journal of the American
statistical association \textbf{74}(366a): 427--431. Taylor \& Francis.

\hypertarget{ref-Edwards2004PortfolioStocks}{}
Edwards, S.F., Link, J.S., and Rountree, B.P. 2004. Portfolio Management
of Wild Fish Stocks. Ecological Economics \textbf{49}: 317--329.

\hypertarget{ref-Finnoff2003HarvestingEcosystem}{}
Finnoff, D., and Tschirhart, J. 2003. Harvesting in an eight-species
ecosystem. Journal of Environmental Economics and Management
\textbf{45}(3): 589--611. Academic Press Orlando, USA.

\hypertarget{ref-francis1992use}{}
Francis, R.I.C. 1992. Use of risk analysis to assess fishery management
strategies: A case study using orange roughy (hoplostethus atlanticus)
on the chatham rise, new zealand. Canadian Journal of Fisheries and
Aquatic Sciences \textbf{49}(5): 922--930. NRC Research Press.

\hypertarget{ref-golet_paradox_2015}{}
Golet, W., Record, N., Lehuta, S., Lutcavage, M., Galuardi, B., Cooper,
A., and Pershing, A. 2015. The paradox of the pelagics: Why bluefin tuna
can go hungry in a sea of plenty. Marine Ecology Progress Series
\textbf{527}: 181--192.
doi:\href{https://doi.org/10.3354/meps11260}{10.3354/meps11260}.

\hypertarget{ref-golet_changes_2013}{}
Golet, W.J., Galuardi, B., Cooper, A.B., and Lutcavage, M.E. 2013.
Changes in the Distribution of Atlantic Bluefin Tuna ( Thunnus thynnus )
in the Gulf of Maine 1979-2005. PLOS ONE \textbf{8}(9): e75480.
doi:\href{https://doi.org/10.1371/journal.pone.0075480}{10.1371/journal.pone.0075480}.

\hypertarget{ref-Hanna1995UserCouncil}{}
Hanna, S.S. 1995. User participation and fishery management performance
within the pacific fishery management council. Ocean and Coastal
Management \textbf{28}(1-3): 23--44.
doi:\href{https://doi.org/10.1016/0964-5691(95)00046-1}{10.1016/0964-5691(95)00046-1}.

\hypertarget{ref-Hanna1999StrengtheningResources}{}
Hanna, S.S. 1999. Strengthening governance of ocean fishery resources.
Ecological Economics \textbf{31}(2): 275--286.
doi:\href{https://doi.org/10.1016/S0921-8009(99)00084-1}{10.1016/S0921-8009(99)00084-1}.

\hypertarget{ref-hatch_arctic_2002}{}
Hatch, J.J. 2002. Arctic Tern (Sterna paradisaea). The Birds of North
America Online.
doi:\href{https://doi.org/10.2173/bna.707}{10.2173/bna.707}.

\hypertarget{ref-hilborn_quantitative_2003}{}
Hilborn, R., and Walters, C.J. 2003. Quantitative Fisheries Stock
Assessment: Choice, Dynamics and Uncertainty. Springer Science \&
Business Media.

\hypertarget{ref-holland1999empirical}{}
Holland, D.S., and Sutinen, J.G. 1999. An empirical model of fleet
dynamics in new england trawl fisheries. Canadian Journal of Fisheries
and Aquatic Sciences \textbf{56}(2): 253--264. NRC Research Press.

\hypertarget{ref-hutniczak2014increasing}{}
Hutniczak, B. 2014. Increasing pressure on unregulated species due to
changes in individual vessel quotas: An empirical application to trawler
fishing in the baltic sea. Marine Resource Economics \textbf{29}(3):
201--217. University of Chicago Press Chicago, IL.

\hypertarget{ref-iccat_report_2015}{}
ICCAT. 2015. Report of the 2014 Atlantic bluefin tuna stock assessment
session (Madrid, Spain, 22--27 September 2014). Collected Volume
Scientific Papers \textbf{71}(2): 692--945. Available from
\url{https://www.iccat.int/Documents/Meetings/Docs/2014_BFT_ASSESS-ENG.pdf}.

\hypertarget{ref-Jin2016ApplyingEBFM}{}
Jin, D., DePiper, G., and Hoagland, P. 2016. Applying Portfolio
Management to Implement Ecosystem-Based Fishery Management (EBFM). North
American Journal of Fisheries Management \textbf{36}: 652--669.
doi:\href{https://doi.org/10.1080/02755947.2016.1146180}{10.1080/02755947.2016.1146180}.

\hypertarget{ref-Jin2003LinkingEcosystem}{}
Jin, D., Hoagland, P., and Dalton, T.M. 2003. Linking economic and
ecological models for a marine ecosystem. Ecological Economics.
doi:\href{https://doi.org/10.1016/j.ecolecon.2003.06.001}{10.1016/j.ecolecon.2003.06.001}.

\hypertarget{ref-Jin2012DevelopmentEngland}{}
Jin, D., Hoagland, P., Dalton, T.M., and Thunberg, E.M. 2012.
Development of an integrated economic and ecological framework for
ecosystem-based fisheries management in New England. Progress in
Oceanography \textbf{102}: 93--101. Elsevier.
doi:\href{https://doi.org/10.1016/j.pocean.2012.03.007}{10.1016/j.pocean.2012.03.007}.

\hypertarget{ref-johnston2010methods}{}
Johnston, R.J., and Rosenberger, R.S. 2010. Methods, trends and
controversies in contemporary benefit transfer. Journal of Economic
Surveys \textbf{24}(3): 479--510. Wiley Online Library.

\hypertarget{ref-katsukawa2004numerical}{}
Katsukawa, T. 2004. Numerical investigation of the optimal control rule
for decision-making in fisheries management. Fisheries science
\textbf{70}(1): 123--131. Wiley Online Library.

\hypertarget{ref-Kirkley2011AConsiderations}{}
Kirkley, J.E., Walden, J., and Färe, R. 2011. A general equilibrium
model for Atlantic herring (Clupea harengus) with ecosystem
considerations. ICES Journal of Marine Science: Journal du Conseil
\textbf{68}(5): 860--866. Oxford University Press.

\hypertarget{ref-Larkin1999IntrinsicFishery}{}
Larkin, S.L., and Sylvia, G. 1999. Intrinsic fish characteristics and
intraseason production efficiency: a management-level bioeconomic
analysis of a commercial fishery. American Journal of Agricultural
Economics \textbf{81}(1): 29--43. Agricultural \& Applied Economics
Association; Blackwell Publishing.

\hypertarget{ref-Larson2004RevealingData}{}
Larson, D.M., Shaikh, S.L., and Layton, D.F. 2004. Revealing Preferences
for Leisure Time from Stated Preference Data. American Journal of
Agricultural Economics \textbf{86}(2): 307--320. Blackwell Synergy.

\hypertarget{ref-Lehuta2013InvestigatingEngland}{}
Lehuta, S., Holland, D.S., and Pershing, A.J. 2013. Investigating
interconnected fisheries: a coupled model of the lobster and herring
fisheries in New England. Canadian Journal of Fisheries and Aquatic
Sciences \textbf{71}(2): 272--289. NRC Research Press.

\hypertarget{ref-lew2015willingness}{}
Lew, D.K. 2015. Willingness to pay for threatened and endangered marine
species: A review of the literature and prospects for policy use.
Frontiers in Marine Science \textbf{2}: 96. Frontiers.

\hypertarget{ref-lew2017temporal}{}
Lew, D.K., and Wallmo, K. 2017. Temporal stability of stated preferences
for endangered species protection from choice experiments. Ecological
Economics \textbf{131}: 87--97. Elsevier.

\hypertarget{ref-lew2010valuing}{}
Lew, D.K., Layton, D.F., and Rowe, R.D. 2010. Valuing enhancements to
endangered species protection under alternative baseline futures: The
case of the steller sea lion. Marine Resource Economics \textbf{25}(2):
133--154. University of Chicago Press Chicago, IL.

\hypertarget{ref-link_response_2009}{}
Link, J., Col, L., Guida, V., Dow, D., O'Reilly, J., Green, J.,
Overholtz, W., Palka, D., Legault, C., Vitaliano, J., Griswold, C.,
Fogarty, M., and Friedland, K. 2009. Response of balanced network models
to large-scale perturbation: Implications for evaluating the role of
small pelagics in the Gulf of Maine. Ecological Modelling
\textbf{220}(3): 351--369.
doi:\href{https://doi.org/10.1016/j.ecolmodel.2008.10.009}{10.1016/j.ecolmodel.2008.10.009}.

\hypertarget{ref-link_documentation_2006}{}
Link, J., Griswold, C., Methratta, E., and Gunnard, J. (\emph{Editors}).
2006. Documentation for the Energy Modeling and Analysis eXercise
(EMAX). Northeast Fish. Sci. Cent. Ref. Doc. 06-15. US Dep. Commer.,
National Marine Fisheries Service, Woods Hole, MA.

\hypertarget{ref-link_northeast_2008}{}
Link, J., Overholtz, W., O'Reilly, J., Green, J., Dow, D., Palka, D.,
Legault, C., Vitaliano, J., Guida, V., Fogarty, M., Brodziak, J.,
Methratta, L., Stockhausen, W., Col, L., and Griswold, C. 2008. The
Northeast U.S. continental shelf Energy Modeling and Analysis exercise
(EMAX): Ecological network model development and basic ecosystem
metrics. Journal of Marine Systems \textbf{74}(1--2): 453--474.
doi:\href{https://doi.org/10.1016/j.jmarsys.2008.03.007}{10.1016/j.jmarsys.2008.03.007}.

\hypertarget{ref-logan_diet_2015}{}
Logan, J.M., Golet, W.J., and Lutcavage, M.E. 2015. Diet and condition
of Atlantic bluefin tuna (Thunnus thynnus) in the Gulf of Maine,
2004--2008. Environmental Biology of Fishes \textbf{98}(5): 1411--1430.
doi:\href{https://doi.org/10.1007/s10641-014-0368-y}{10.1007/s10641-014-0368-y}.

\hypertarget{ref-loomis1996economic}{}
Loomis, J.B., and White, D.S. 1996. Economic benefits of rare and
endangered species: Summary and meta-analysis. Ecological Economics
\textbf{18}(3): 197--206. Elsevier.

\hypertarget{ref-McConnell2000HedonicHawaii}{}
McConnell, K.E., and Strand, I.E. 2000. Hedonic Prices for Fish: Tuna
Prices in Hawaii. American Journal of Agricultural Economics
\textbf{82}(1): 133--144. American Agricultural Economics Association.

\hypertarget{ref-navrud2007environmental}{}
Navrud, S., and Ready, R.C. 2007. Environmental value transfer: Issues
and methods. Springer.

\hypertarget{ref-nisbet_common_2002}{}
Nisbet, I.C.T. 2002. Common Tern (Sterna hirundo). The Birds of North
America Online.
doi:\href{https://doi.org/10.2173/bna.618}{10.2173/bna.618}.

\hypertarget{ref-NEFSC2012Assessment}{}
Northeast Fisheries Science Center. 2012. 54\(^{th}\) Northeast Regional
Stock Assessment Workshop (54th SAW) Assessment Report. US Dept of
Commerce, Northeast Fisheries Science Center,166 Water Street, Woods
Hole, MA 02543.

\hypertarget{ref-Deroba2015atlantic}{}
Northeast Fisheries Science Center. 2015. Atlantic herring operational
assessment report 2015. US Dept of Commerce, Northeast Fisheries Science
Center,166 Water Street, Woods Hole, MA 02543.

\hypertarget{ref-Osborne2000MeetingsParticipation}{}
Osborne, M.J., Rosenthal, J.S., and Turner, M.A. 2000. Meetings with
Costly Participation Meetings with Costly Participation. American
Economic Review \textbf{90}(4): 927--943. Available from
\href{http://www.jstor.org/stable/117315\%20http://www.jstor.org/stable/117315}{http://www.jstor.org/stable/117315 http://www.jstor.org/stable/117315}.

\hypertarget{ref-Pesaran2001BoundsRelationships}{}
Pesaran, M.H., Shin, Y., and Smith, R.J. 2001. Bounds testing approaches
to the analysis of level relationships. Journal of Applied Econometrics
\textbf{16}(3): 289--326.
doi:\href{https://doi.org/10.1002/jae.616}{10.1002/jae.616}.

\hypertarget{ref-plaganyi2014multispecies}{}
Plagányi, É.E., Punt, A.E., Hillary, R., Morello, E.B., Thébaud, O.,
Hutton, T., Pillans, R.D., Thorson, J.T., Fulton, E.A., Smith, A.D., and
others. 2014. Multispecies fisheries management and conservation:
Tactical applications using models of intermediate complexity. Fish and
Fisheries \textbf{15}(1): 1--22. Wiley Online Library.

\hypertarget{ref-porch_making_2016}{}
Porch, C.E., and Lauretta, M.V. 2016. On Making Statistical Inferences
Regarding the Relationship between Spawners and Recruits and the
Irresolute Case of Western Atlantic Bluefin Tuna ( Thunnus thynnus ).
PLOS ONE \textbf{11}(6): e0156767.
doi:\href{https://doi.org/10.1371/journal.pone.0156767}{10.1371/journal.pone.0156767}.

\hypertarget{ref-punt2016management}{}
Punt, A.E., Butterworth, D.S., Moor, C.L., De Oliveira, J.A., and
Haddon, M. 2016a. Management strategy evaluation: Best practices. Fish
and Fisheries \textbf{17}(2): 303--334. Wiley Online Library.

\hypertarget{ref-punt_management_2016}{}
Punt, A.E., Butterworth, D.S., Moor, C.L. de, De Oliveira, J.A.A., and
Haddon, M. 2016b. Management strategy evaluation: Best practices. Fish
and Fisheries \textbf{17}(2): 303--334.
doi:\href{https://doi.org/10.1111/faf.12104}{10.1111/faf.12104}.

\hypertarget{ref-punt2008evaluation}{}
Punt, A.E., Dorn, M.W., and Haltuch, M.A. 2008. Evaluation of threshold
management strategies for groundfish off the us west coast. Fisheries
Research \textbf{94}(3): 251--266. Elsevier.

\hypertarget{ref-Rcite}{}
R Core Team. 2016. R: A language and environment for statistical
computing. R Foundation for Statistical Computing, Vienna, Austria.
Available from \url{https://www.R-project.org/}.

\hypertarget{ref-rago_update_2013}{}
Rago, P., and Sosebee, K. 2013. Update on the status of spiny dogfish in
2013 and projected harvests at the Fmsy proxy and Pstar of 40\%. Mid
Atlantic Fishery Management Council Scientific and Statistical
Committee. Available from
\url{https://www.mafmc.org/s/2015-Status-Report-and-Projections_final.pdf}
{[}accessed 15 November 2016{]}.

\hypertarget{ref-rago_biological_2010}{}
Rago, P.J., and Sosebee, K.A. 2010. Biological Reference Points for
Spiny Dogfish. Northeast Fish Sci Cent Ref Doc. 10-06: 52. Available
from
\url{http://docs.lib.noaa.gov/noaa_documents/NMFS/NEFSC/NEFSC_reference_documnet/NEFSC_RD_10_12.pdf}
{[}accessed 19 November 2016{]}.

\hypertarget{ref-rago_implications_1998}{}
Rago, P.J., Sosebee, K.A., Brodziak, J.K.T., Murawski, S.A., and
Anderson, E.D. 1998. Implications of recent increases in catches on the
dynamics of Northwest Atlantic spiny dogfish (Squalus acanthias).
Fisheries Research \textbf{39}(2): 165--181. Available from
\url{http://www.sciencedirect.com/science/article/pii/S0165783698001817}
{[}accessed 15 November 2016{]}.

\hypertarget{ref-reimer2017fisheries}{}
Reimer, M.N., Abbott, J.K., and Wilen, J.E. 2017. Fisheries production:
Management institutions, spatial choice, and the quest for policy
invariance. Marine Resource Economics \textbf{32}(2): 143--168.
University of Chicago Press Chicago, IL.

\hypertarget{ref-restrepo_updated_2010}{}
Restrepo, V.R., Diaz, G.A., Walter, J.F., Neilson, J.D., Campana, S.E.,
Secor, D., and Wingate, R.L. 2010. Updated estimate of the growth curve
of Western Atlantic bluefin tuna. Aquatic Living Resources
\textbf{23}(4): 335--342.
doi:\href{https://doi.org/10.1051/alr/2011004}{10.1051/alr/2011004}.

\hypertarget{ref-richardson2009total}{}
Richardson, L., and Loomis, J. 2009. The total economic value of
threatened, endangered and rare species: An updated meta-analysis.
Ecological Economics \textbf{68}(5): 1535--1548. Elsevier.

\hypertarget{ref-smith_consumption_2015}{}
Smith, L.A., Link, J.S., Cadrin, S.X., and Palka, D.L. 2015. Consumption
by marine mammals on the Northeast U.S. continental shelf. Ecological
Applications \textbf{25}(2): 373--389.
doi:\href{https://doi.org/10.1890/13-1656.1}{10.1890/13-1656.1}.

\hypertarget{ref-Tschirhart2000GeneralEcosystem}{}
Tschirhart, J. 2000. General Equilibrium of an Ecosystem. Journal of
Theoretical Biology \textbf{203}(1): 13--32.

\hypertarget{ref-Turner2005MeetingsAnalysis}{}
Turner, M., and Weninger, Q. 2005. Meetings with costly participation:
An empirical analysis. Review of Economic Studies \textbf{72}(1):
247--268.
doi:\href{https://doi.org/10.1111/0034-6527.00331}{10.1111/0034-6527.00331}.

\hypertarget{ref-Whitlock2005CombiningApproach}{}
Whitlock, M.C. 2005. Combining probability from independent tests: The
weighted Z-method is superior to Fisher's approach. Journal of
Evolutionary Biology.
doi:\href{https://doi.org/10.1111/j.1420-9101.2005.00917.x}{10.1111/j.1420-9101.2005.00917.x}.


\end{document}
