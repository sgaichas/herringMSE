\documentclass[]{article}
\usepackage{lmodern}
\usepackage{amssymb,amsmath}
\usepackage{ifxetex,ifluatex}
\usepackage{fixltx2e} % provides \textsubscript
\ifnum 0\ifxetex 1\fi\ifluatex 1\fi=0 % if pdftex
  \usepackage[T1]{fontenc}
  \usepackage[utf8]{inputenc}
\else % if luatex or xelatex
  \ifxetex
    \usepackage{mathspec}
  \else
    \usepackage{fontspec}
  \fi
  \defaultfontfeatures{Ligatures=TeX,Scale=MatchLowercase}
\fi
% use upquote if available, for straight quotes in verbatim environments
\IfFileExists{upquote.sty}{\usepackage{upquote}}{}
% use microtype if available
\IfFileExists{microtype.sty}{%
\usepackage{microtype}
\UseMicrotypeSet[protrusion]{basicmath} % disable protrusion for tt fonts
}{}
\usepackage[margin=1in]{geometry}
\usepackage{hyperref}
\hypersetup{unicode=true,
            pdftitle={The dream and the reality: meeting decision-making time frames while incorporating ecosystem and economic models into management strategy evaluation},
            pdfauthor={Jonathan J. Deroba, Sarah K. Gaichas, Min-Yang Lee, Rachel G. Feeney, Deirdre Boelke, Brian J. Irwin},
            pdfborder={0 0 0},
            breaklinks=true}
\urlstyle{same}  % don't use monospace font for urls
\usepackage{longtable,booktabs}
\usepackage{graphicx,grffile}
\makeatletter
\def\maxwidth{\ifdim\Gin@nat@width>\linewidth\linewidth\else\Gin@nat@width\fi}
\def\maxheight{\ifdim\Gin@nat@height>\textheight\textheight\else\Gin@nat@height\fi}
\makeatother
% Scale images if necessary, so that they will not overflow the page
% margins by default, and it is still possible to overwrite the defaults
% using explicit options in \includegraphics[width, height, ...]{}
\setkeys{Gin}{width=\maxwidth,height=\maxheight,keepaspectratio}
\IfFileExists{parskip.sty}{%
\usepackage{parskip}
}{% else
\setlength{\parindent}{0pt}
\setlength{\parskip}{6pt plus 2pt minus 1pt}
}
\setlength{\emergencystretch}{3em}  % prevent overfull lines
\providecommand{\tightlist}{%
  \setlength{\itemsep}{0pt}\setlength{\parskip}{0pt}}
\setcounter{secnumdepth}{0}
% Redefines (sub)paragraphs to behave more like sections
\ifx\paragraph\undefined\else
\let\oldparagraph\paragraph
\renewcommand{\paragraph}[1]{\oldparagraph{#1}\mbox{}}
\fi
\ifx\subparagraph\undefined\else
\let\oldsubparagraph\subparagraph
\renewcommand{\subparagraph}[1]{\oldsubparagraph{#1}\mbox{}}
\fi

%%% Use protect on footnotes to avoid problems with footnotes in titles
\let\rmarkdownfootnote\footnote%
\def\footnote{\protect\rmarkdownfootnote}

%%% Change title format to be more compact
\usepackage{titling}

% Create subtitle command for use in maketitle
\newcommand{\subtitle}[1]{
  \posttitle{
    \begin{center}\large#1\end{center}
    }
}

\setlength{\droptitle}{-2em}
  \title{The dream and the reality: meeting decision-making time frames while
incorporating ecosystem and economic models into management strategy
evaluation}
  \pretitle{\vspace{\droptitle}\centering\huge}
  \posttitle{\par}
  \author{Jonathan J. Deroba, Sarah K. Gaichas, Min-Yang Lee, Rachel G. Feeney,
Deirdre Boelke, Brian J. Irwin}
  \preauthor{\centering\large\emph}
  \postauthor{\par}
  \date{}
  \predate{}\postdate{}

\usepackage{setspace}
\usepackage{authblk}

\usepackage{lineno}
\linenumbers

\usepackage{float}
\let\origfigure\figure
\let\endorigfigure\endfigure
\renewenvironment{figure}[1][2] {
    \expandafter\origfigure\expandafter[H]
} {
    \endorigfigure
}

\begin{document}
\maketitle

\doublespacing
J.J. Deroba (corresponding author): NOAA Fisheries, Northeast Fisheries
Science Center. 166 Water St.~Woods Hole MA. 02543.
\href{mailto:Jonathan.Deroba@noaa.gov}{\nolinkurl{Jonathan.Deroba@noaa.gov}}
tel: (508) 495-2310 Fax: (508) 495 - 2393

S.K. Gaichas: NOAA Fisheries, Northeast Fisheries Science Center. 166
Water St.~Woods Hole MA. 02543.
\href{mailto:Sarah.Gaichas@noaa.gov}{\nolinkurl{Sarah.Gaichas@noaa.gov}}

M. Lee: NOAA Fisheries, Northeast Fisheries Science Center. 166 Water
St.~Woods Hole MA. 02543.
\href{mailto:Min-Yang.Lee@noaa.gov}{\nolinkurl{Min-Yang.Lee@noaa.gov}}

R.G. Feeney: New England Fisheries Management Council,50 Water Street,
Mill 2, Newburyport, MA 01950.
\href{mailto:RFeeney@nefmc.org}{\nolinkurl{RFeeney@nefmc.org}}

D. Boelke: New England Fisheries Management Council,50 Water Street,
Mill 2, Newburyport, MA 01950.
\href{mailto:DBoelke@nefmc.org}{\nolinkurl{DBoelke@nefmc.org}}

B.J. Irwin: U.S. Geological Survey. Georgia Cooperative Fish and
Wildlife Research Unit. University of Georgia. 180 E. Green St.~Athens,
GA 30602. \href{mailto:irwin@uga.edu}{\nolinkurl{irwin@uga.edu}}.

This draft manuscript is distributed solely for purposes of scientific
peer review. Its content is deliberative and predecisional, so it must
not be disclosed or released by reviewers. Because the manuscript has
not yet been approved for publication by the U.S. Geological Survey
(USGS), it does not represent any official USGS finding or policy.
\newpage

\section{Abstract}\label{abstract}

Atlantic herring in the Northwest Atlantic have been managed with
interim harvest control rules (HCRs). A stakeholder driven management
strategy evaluation (MSE) was conducted that incorporated a broad range
of objectives. The MSE process was completed within one year. Constant
catch, conditional constant catch, and a biomass-based (BB) HCR with a
15\% restriction on the interannual change in the quota could achieve
more stable yields than BB HCRs without such restrictions, but could not
attain as high of yields and resulted in more negative outcomes for
terns ( a predator of herring). A similar range of performance could be
achieved by applying a BB HCR annually, every three years, or every five
years. Predators (i.e., dogfish, tuna, and terns) were generally
insensitive to the range of HCRs, and net revenues were stable across a
range of net revenues. In order to meet management needs, some aspects
of the simulations were less than might be considered scientifically
ideal, but using ``models of intermediate complexity'' were informative
for managers and formed a foundation for future improvements.

\section{Introduction}\label{introduction}

Atlantic herring (hereafter herring; \emph{Clupea harengus}) in the
Northwest Atlantic are preyed upon by fish, seabirds, and marine
mammals, and can account for 20-50\% of the diet of these predators
(Overholtz and Link 2007, Smith and Link 2010, Curti et al. 2013).
Herring are also subject to a directed fishery, mostly using midwater
trawls, purse seines, and bottom trawls, that averaged 85,000mt of
landings during 2008-2017. Much of the herrring landed by humans are
used as bait in the relatively high value American lobster
(\emph{Homarus americanus}) fishery. Herring life-history traits (e.g.,
weight-at-age) have varied through time, and their complex stock
structure creates uncertainty in their assessment and management
(Northeast Fisheries Science Center 2012, Deroba 2015). Thus, herring
are of broad interest in the region, but anticipating the relative
performance of mangement strategies (e.g., harvest control rules) in the
face of uncertainties is challenging, which makes a management strategy
evaluation (MSE) for herring in the region a potentially informative
tool.

The federal fisheries management process in New England usually starts
when the New England Fishery Managment Council (NEFMC), the political
body responsible for federally managed species in the northeast US,
perceive a problem causing a management goal or objective to be unmet.
Managers will propose a range of potential solutions and a technical
group, typically composed of scientists and policy analysts from state
agencies and the National Marine Fisheries Service (NMFS), will analyze
these possible solutions. Council meetings are public, and stakholder
input is solicited through open comment periods and diverse advisory
panels. Once the NEFMC votes on a solution, NMFS will, after verifying
that it is consistent with applicable laws, translate those solutions
into regulations and enforce those regulations. Management actions,
particularly contentious ones, can take several years to develop and
implement.

Management strategy evaluation uses simulation to evaluate the
trade-offs resulting from alternative management options in the face of
uncertainty (Punt et al. 2016a). MSEs require time, however, for
stakeholder input, data collection, and model development (Butterworth
2007, Punt et al. 2016a). As such, the process can take much longer than
``traditional'' management time frames (Butterworth 2007). The
development time is also likely to lengthen when explicit ecosystem,
multi-species, or socioeconomic considerations are included because the
data and modeling needs, and subsequent uncertainties, are all greater
than in a single species approach. This manuscript chronicles the
development of a MSE done on a truncated timetable (\textasciitilde{}12
months) required to meet management time frames. The objectives of this
manuscript were to:

\begin{enumerate}
\def\labelenumi{\arabic{enumi}.}
\item
  Evaluate the relative performance of HCRs at meeting herring fishery
  objectives, including those related to predators of herring, as
  informed by stakeholder input, and,
\item
  Discuss our approach to developing a MSE on a relatively truncated
  timetable in order to meet management time frames, and identify the
  lessons learned throughout the process, especially as they relate to
  using MSE as a tool to advance an ecosystem based approach to
  management (Plagányi et al. 2014).
\end{enumerate}

During January 2016, the NEFMC requested a MSE to evaluate harvest
control rules (HCRs) for herring\footnote{The process is described in
  detail in Feeney et al. (In review).}. Fishery managers wanted to
develop a HCR that, among other things, accounted for the role of
Atlantic herring as forage in the ecosystem. Except that the main
interest was in the effect that herring have on predators, as opposed to
top-down effects of predators on herring, the exact ``accounting''
system was left to be defined by stakeholder-driven workshops that were
open to the public. Two stakeholder workshops were conducted, one in May
2016 and another in December 2016. Members of the herring, lobster,
groundfish, tuna, recreational, and whale-watching industries
participated, as well as environmental non-government organizations
(ENGO), federal and state agencies, and academics. Input from these
diverse stakeholders were then utilized to construct the closed-loop
simulation portion of the MSE. Notably, while the techincal group knew
the general scope of the modeling exercise (HCRs that account for
herring as forage), detailed modeling of many components of interest
could not begin in earnest until this step was finished.

Constucting an ecosystem model with an interdisciplinary team of
scientists to support these management decisions would be a useful and
perhaps ideal tool. However, the NEFMC desired results from the MSE
within one year, which constrained the development of many components of
the model. Deciding how to allocate scarce research effort and time to
various components of model development is always challenging; the
limited time available for model development made this process even more
challenging. Relatively simple models linking key ecosystem components
can have many advantages over more complex models for ecosystem analysis
(Plagányi et al. 2014, Collie et al. 2016, Punt et al. 2016b). These
``models of intermediate complexity for ecosystem assessments (MICE;
Plagányi et al. (2014))'' can also be a reasonable approach to
conducting science to meet relatively short management timelines while
ensuring the models and results reasonably describe the trade-offs among
objectives and remain relevant for decision-making. The methods in this
article describe the reality of what was achieved in order to enable
managers to make informed decisions within their preferred time frames.
This reality is then contrasted with more comprehensive scientific
approaches that should be the ultimate goal for use in future
decisionmaking (i.e., the ``dream''). This manuscript is a companion to
Feeney et al. (In review) which more fully describes the managment
context and how this MSE was blended with fisheries managment.

\section{Methods - The Reality}\label{methods---the-reality}

\subsection{Herring}\label{herring}

a MSE was developed specific to Gulf of Maine - Georges Bank Atlantic
herring. The herring component was modified from Deroba (2014), and
symbols are largely consistent (Table \ref{symtab}). The MSE was based
on an age-structured simulation that considered fish from age-1 through
age-8+ (age-8 and older), which is consistent with the age ranges used
in the 2012 and 2015 Atlantic herring stock assessments (Northeast
Fisheries Science Center 2012, Deroba 2015). The abundances at age in
year one of all simulations equaled the equilibrium abundances produced
by the fishing mortality rate that would reduce the population to 40\%
of \(SSB_{F=0}\), but simulations were of sufficient length to make the
starting values moot. Abundance in each subsequent age and year was
calculated assuming that fish died exponentially according to an age and
year specific total instantaneous mortality rate.

Recruitment followed Beverton-Holt dynamics (Francis 1992):

\begin{align}
 R_{1,y+1}&=\frac{(\frac{SSB_{F=0}}{R_{F=0}} \frac{1-h}{4h})SSB_y}{1+(\frac{5h-1}{4hR_{F=0}})SSB_y}e^{\varepsilon_{R,y}-\frac{\sigma_R^2}{2}} \\
 \varepsilon_{R,y}&=\omega \varepsilon_{R,y-1}+\sqrt{1-\omega^2}\rho_y \\
 \rho &\sim N(0,\sigma_R^2) \\
 SSB_y&=\sum\limits_{a=1}^{8+}N_{a,y}m_aW_a.
\end{align}

The variance of recruitment process errors (\(\sigma_R^2\)) equaled 0.36
and the degree of autocorrelation (\(\omega\)) equaled 0.1, which are
values consistent with recruitment estimates from a recent Atlantic
herring stock assessment (Deroba 2015).

\subsubsection{Assessment Error}\label{assessment-error}

A stock assessment was approximated (i.e., assessment errors) similar to
Punt et al. (2008) and Deroba (2014). Assessment error was modeled as a
year-specific lognormal random deviation common to all ages, with
first-order autocorrelation and a term that created the option to
include bias \(\rho\):

\begin{align}
\widehat{N}_{a,y}&=[N_{a,y}(\rho+1)]e^{\varepsilon_{\phi,y}-\frac{\sigma_\phi^2}{2}} \\
\varepsilon_{\phi,y}&=\vartheta\varepsilon_{\phi,y-1}+\sqrt{1-\vartheta^2}\tau_y \\
\tau &\sim N(0,\sigma_\phi^2). 
\end{align}

The variance of assessment errors (\(\sigma_\phi^2\)) equaled 0.05 and
autocorrelation (\(\vartheta\)) equaled 0.7. A range of values for
\(\sigma_\phi^2\) and \(\vartheta\) were not evaluated because previous
research using a similar approach to applying assessment errors has
found relative control rule performance to be robust to these quantities
(Irwin et al. 2008, Punt et al. 2008, Deroba and Bence 2012). Rho
(\(\rho\)) allowed for the inclusion of bias in the assessed value of
abundance (see below; Deroba (2014)). Assessed spawning stock biomass
(\(\widehat{SSB}_y\)) was calculated similarly to \(SSB_y\) except with
\(N_{a,y}\) replaced with \(\widehat{N}_{a,y}\), and assessed total
biomass (\(\widehat{B}_y\)) was calculated as the sum across ages of the
product of \(\widehat{N}_{a,y}\) and \(W_a\).

\subsubsection{Operating Models}\label{operating-models}

The stakeholder workshops identified uncertainties about herring life
history traits and stock assessment. The effect of some of these
uncertainties on harvest control rule performance was evaluated by
simulating the control rules for each of eight operating models (Table
\ref{OMs}). The uncertainties addressed by the eight operating models
included: Atlantic herring natural mortality and steepness, Atlantic
herring weight-at-age, and possible bias in the stock assessment beyond
the unbiased measurement error (\(\varepsilon_{\phi y}\)).

The specific values used in the operating models for each of the
uncertainties were premised on data used in recent stock assessments or
estimates from fits of stock assessment models (Deroba 2015). Natural
mortality in recent stock assessments has varied among ages and years,
with \(M\) being higher during 1996-2014 than in previous years
(Northeast Fisheries Science Center 2012, Deroba 2015). Natural
mortality, however, has also been identified as an uncertainty in the
stock assessments and sensitivity runs have been conducted without
higher \(M\) during 1996-2014, such that \(M\) was constant among years
(Northeast Fisheries Science Center 2012, Deroba 2015). To capture
uncertainty in \(M\) in the MSE, operating models were run with either
relatively high or low \(M_a\) (Table \ref{MWttab}). Relatively high
\(M_a\) values equaled the age-specific natural mortality rates used for
the years 1996-2014 in the stock assessment. Relatively low \(M_a\)
values in the MSE equaled the age-specific natural mortality rates used
for the years 1965-1995 in the stock assessment. In the MSE, \(M_a\) was
always time invariant.

Uncertainty in estimates of stock-recruit parameters were represented in
the MSE by using the parameters estimated by stock assessments fit with
and without the higher \(M\) during 1996-2014. Stock assessment fits
with higher \(M\) during 1996-2014 produced estimates of steepness and
\(SSB_{F=0}\) that were lower than in stock assessment fits without
higher \(M\) during 1996-2014 (Tables \ref{OMs} and \ref{MWttab}). Thus,
operating models with relatively high \(M_a\) always had relatively low
steepness and \(SSB_{F=0}\), and the opposite held with relatively low
\(M_a\) (Table \ref{OMs}).

Uncertainty in Atlantic herring size-at-age was accounted for by having
operating models with either fast or slow growth (i.e., weights-at-age;
Table \ref{MWttab}). Atlantic herring weight-at-age generally declined
from the mid-1980s through the mid-1990s, especially at ages-3 and
older, and has been relatively stable since (Deroba 2015). Reasons for
the decline are speculative and no causal relationships have been
established. Likewise, the reasons why growth has declined for older
ages, but has either remained stable or increased at age-1 and age-2 are
unclear. Thus, the terms ``fast'' and ``slow'' growth were used as
generally describing the growth conditions of older Atlantic herring.
Fast growth operating models had weights-at-age that equaled the January
1 weights-at-age from the most recent stock assessment averaged over the
years 1976-1985, while the slow growth operating models averaged over
the years 2005-2014 (Deroba 2015). In the MSE, weight-at-age was always
time invariant.

Differences in \(M\), stock-recruit parameters, and weights-at-age led
to differences in unfished (i.e., virgin) and \(MSY\) reference points
among operating models (Table \ref{RefPts}). The effect of \(M\) and
stock-recruit parameters was larger than the effect of differences in
weight-at-age (Table \ref{RefPts}).

To address concerns about possible stock assessment bias, operating
models with and without a positive bias were included. In operating
models without bias, \(\rho\)=0 and the only assessment error was that
caused by the unbiased measurement errors (\(\epsilon_{\phi y}\)). In
operating models with bias, \(\rho\) equaled 0.6, which was based on the
degree of retrospective pattern in \(SSB\) from the most recent stock
assessment (Deroba 2015).

\subsubsection{Harvest Control Rules}\label{harvest-control-rules}

Several classes of control rules were evaluated, including a biomass
based control rule (Katsukawa 2004), a constant catch rule, and a
conditional constant catch rule (Clark and Hare 2004, Deroba and Bence
2012). The biomass based control rule was defined by three parameters:
the proportion (\(\psi\)) of \(F_{MSY}\) that dictates the maximum
desired fishing mortality rate (\(\tilde{F}\)), an upper SSB threshold
(\(SSB_{up}\)), and a lower SSB threshold (\(SSB_{low}\)). The
\(\tilde{F}\) equaled the maximum when \(\widehat{SSB}\) was above the
upper threshold, declined linearly between the upper and lower
thresholds, and equaled zero below the lower threshold:

\begin{equation}
  \label{Fy_equation}
    \tilde{F}_y=
    \begin{cases}
      \psi F_{MSY}, & \text{if}\ \widehat{SSB}_y \geq SSB_{up} \\
      \psi F_{MSY} \frac{\widehat{SSB}_y - SSB_{low}}{SSB_{up}-SSB_{low}}, & \text{if}\ SSB_{low} < \widehat{SSB}_y < SSB_{up}\\
      0, & \text{if}\ \widehat{SSB}_y \leq SSB_{low}
    \end{cases}
  \end{equation}

The \(\tilde{F}_y\) was then used to set a quota in year y + 1.
\(\tilde{F}_{ay}\) equaled \(\tilde{F}_y\) times \(S_a\), and \(S_a\)
was time and simulation invariant selectivity at age equal to the values
for the mobile gear fishery as in Deroba (2015). \(\tilde{F}_y\) was
used to set a quota in the following year to approximate the practice of
using projections based on an assessment using data through year y - 1
to set quotas in the following year(s). Furthermore, although
\(\tilde{F}_y\) was set using \(\widehat{SSB}_y\), the quota was based
on \(\widehat{B}_y\) because the fishery selects some immature ages. The
fully selected fishing mortality rate that would remove the quota from
the true population (\(\bar{F}_y\)) was found using Newton-Raphson
iterations. Several variations of the biomass based rule were also
evaluated. These variations included applying the control rule annually,
using the same quota for three year blocks such that the control rule is
applied every fourth year (i.e., \(Q_{y+1}=Q_{y+2}=Q_{y+3}\)), using the
same quota for 5 year blocks, and using the same quota for three year
blocks but restricting the change in the quota to 15\% in either
direction when the control rule was reapplied in the fourth year. Thus,
four variants of the biomass based control rule were evaluated: 1)
annual application, 2) three year blocks, 3) five year blocks, and 4) 3
year blocks with a 15\% restriction.

For each biomass based control rule variant, a range of values for the
three parameters defining the control rule were evaluated. The
proportion (\(\psi\)) of \(F_{MSY}\) that dictates the maximum desired
fishing mortality rate was varied from 0.1\(F_{MSY}\) to 1.0\(F_{MSY}\)
in increments of 0.1, while the lower and upper SSB threshold parameters
(\(SSB_{low} \; SSB_{up}\)) were varied from 0.0\(SSB_{MSY}\) to
4\(SSB_{MSY}\) but with inconsistent increments (i.e., 0.0, 0.1, 0.3,
0.5, 0.7, 0.9, 1.0, 1.1, 1.3, 1.5, 1.7, 2.0, 2.5, 3, 3.5, 4). The full
factorial of combinations for the three biomass based control rule
parameters produced 1,360 shapes (note \(SSB_{low}\) must be
\textless{}= \(SSB_{up}\)) and each of these shapes was evaluated for
each of the four biomass based control rule variants described above.

The constant catch control rule is defined by one parameter, a desired
constant catch (i.e., quota) amount. The constant catch amounts were
varied from 0.1\(MSY\) to 1.0\(MSY\) in increments of 0.1.

The conditional constant catch rule used a constant desired catch amount
unless removing that desired catch from the assessed biomass caused the
fully selected fishing mortality rate to exceed a pre-determined
maximum, in which case the desired catch was set to the value produced
by applying the maximum fully selected fishing mortality rate to the
assessed biomass. Thus, the conditional constant catch rule has two
policy parameters: a desired constant catch amount, and a maximum
fishing mortality rate. The constant catch amounts were varied over the
same range as in the constant catch control rule, while the maximum
fishing mortality rate equaled 0.5\(F_{MSY}\). When the maximum fishing
mortality rate portion of the conditional constant catch rule was
invoked, a quota was set in the same manner as when
\(\widehat{SSB}_y >= SSB_{up}\) in the biomass based control rule
described above.

\subsubsection{Implementation Error}\label{implementation-error}

Implementation errors were also included in a similar way as in Punt et
al. (2008) and Deroba and Bence (2012), as year-specific lognormal
random deviations:

\begin{equation}
F_{a,y}=\bar{F}_yS_ae^{\varepsilon_{\theta,y}-\frac{\sigma_\theta^2}{2}} \\
\varepsilon_{\theta} \sim N(0,\sigma_\theta^2).
\end{equation}

The variance of implementation errors (\(\sigma_\theta^2\)) equaled
0.001. The US Atlantic herring fishery in the Northwest Atlantic
generally catches about the full amount of annual quota. Catches are
monitored through manadatory federal and state reporting requirements,
which are used to close the fishery within 10\% of the annual quota.
Thus, unbiased implementation errors seemed justified.

\subsection{Predators}\label{predators}

The food web of the Northeast US continental shelf large marine
ecosystem is characterized by many diverse predators and prey (Link
2002). There is wealth of scientific information to characterize
predator-prey relationships in this region, including feeding ecology
data for fish predators (e.g., Smith and Link 2010), seabirds (Hall et
al. 2000), bluefin tuna (\emph{Thunnus thynnus}) (Chase 2002, Golet et
al. 2013, 2015, Logan et al. 2015), and marine mammals (Smith et al.
2015). Consumption of herring by predators has been extensively studied
in this ecosystem (Overholtz et al. 2000, Overholtz and Link 2007), and
multiple methods were evaluated to include this consumption within the
most recent herring benchmark stock assessment (Northeast Fisheries
Science Center 2012).

Much of this information was presented at the first stakeholder workshop
in May 2016, where it was agreed that separate ``general predator''
models linked to herring would be a reasonable approach, with the goal
of developing one model for each of the four predator categories: highly
migratory fish, groundfish, seabirds, and marine mammals (Feeney et al.
In review). Bluefin tuna were identified at the stakeholder workshop as
a recommended highly migratory herring predator, and common terns
(\emph{Sterna hirundo}) were identified at the stakeholder workshop as
the recommended seabird herring predator, so these predators were
modeled. No specific groundfish or marine mammal was identified as a
representative herring predator during the stakeholder workshop. In
sections below we discuss these decisions further.

Predators were therefore modeled with fairly simple delay-difference
population dynamics that allowed different predator population processes
to be dependent on some aspect of herring population status, following
(Plagányi and Butterworth 2012). Each predator model takes output from
the herring OM as input, and outputs performance metrics identified at
the stakeholder workshop. While this allows ``bottom up'' effects of
herring on predators to be examined, this configuration does not
consider ``top down'' effects of predators on herring, or simultaneous
interactions of multiple predators with herring.

There were two modeling components for each predator included in the
herring MSE: a predator population model, and a herring-predator
relationship model to link herring with predator populations. Here, we
give an overview of the modeling process, and we describe the decisions
made in parameterizing individual predator models and herring-predator
relationships in the following sections. The overall population in
numbers for each predator \(P\) each year \(N_{y}^P\) is modeled with a
delay-difference function:

\begin{equation}
N_{y+1}^P = N_{y}^PS_{y}^P +  R_{y+1}^P \label{delaydiffN_equation},
\end{equation}

where annual predator survival \(S_{y}^P\) is based on annual natural
mortality \(v\) and exploitation \(u\)

\begin{equation}
S_{y}^P =  (1-v_{y})(1-u) \label{survival_equation},
\end{equation}

and annual recruitment \(R_{y}^P\) (delayed until recruitment age a) is
a Beverton-Holt function defined as above for herring.

Predator population biomass is defined with Ford-Walford plot intercept
(\(FWint\)) and slope (\(FWslope\)) growth parameters

\begin{equation}
B_{y+1}^P = S_{y}^P (FWint N_{y}^P + FWslope B_{y}^P) + FWint R_{y+1}^P \label{delaydiffB_equation}
\end{equation}

Parameterizing this model requires specification of the
stock-recruitment relationship (steepness \(h\) and unfished spawning
stock size in numbers or biomass), the natural mortality rate, the
fishing mortality (exploitation) rate, the initial population size, and
the weight at age of the predator (Ford-Walford plot intercept and slope
parameters). For each predator, population parameters were derived from
different sources (Table \ref{predsource}).

Predator population models were based on either the most recent stock
assessment for the predator or from observational data from the
Northeast US shelf. Herring-predator relationships were based on either
peer-reviewed literature or observational data specific to the Northeast
US shelf. We did not include process or observation error in any of
these modeled relationships. This is obviously unrealistic, but the
primary obective of the herring MSE is to evaluate the effect of herring
management on predators. Leaving out variability driven by anything
other than herring is intended to clarify the effect of herring
managment.

To develop the herring-predator relationship model, specific herring
population characteristics (e.g.~total abundance or biomass, or
abundance/biomass of certain ages or sizes) were related to either
predator growth, predator reproduction, or predator survival. Our aim
was to use information specific to the Northeast US shelf ecosystem,
either from peer-reviewed literature, from observations, or a
combination.

In general, if support for a relationship between herring and predator
recruitment was evident, it was modeled as a predator recruitment
multiplier based on the herring population (\(N_{y}\)) relative to a
specified threshold \(N_{thresh}\):

\begin{equation}
\bar{R}_{y+a}^P = R_{y+a}^P  * \frac{\gamma(N_{y}/N_{thresh})}{(\gamma-1)+(N_{y}/N_{thresh})} \label{recwithherring_equation}, 
\end{equation}

where \(\gamma\) \textgreater{} 1 links herring population size relative
to the threshold level to predator recruitment.

If a relationship between predator growth and herring population size
was evident, annual changes in growth were modeled by modifying either
the Ford-Walford intercept (\(AnnAlpha\)) or slope (\(AnnRho\)):

\begin{equation}
B_{y+1}^P = S_{y}^P (AnnAlpha_{y} N_{y} + FWslope B_{y}) + AnnAlpha_{y}R_{y+1}, or
\end{equation}

\begin{equation}
B_{y+1} = S_{y} (FWint N_{y} + AnnRho_{y} B_{y}) + FWint R_{y+1}, 
\end{equation}

where either \(AnnAlpha\) or \(AnnRho\) are defined for a predator using
herring population parameters (see Eqn. \ref{tunagrow_equation} below).

Finally, herring population size \(N_{y}\) could be related to predator
survival using an annual multiplier on constant predator annual natural
mortality \(v\):

\begin{equation}
v_{y} =  v e ^ {-(\frac{N_{y}}{N_{F=0}})\delta} \label{varmort_equation},
\end{equation}

where 0 \textless{} \(\delta\) \textless{}1 links herring population
size to predator survival.

After specifying the population model parameters and herring-predator
relationship, we applied the (Hilborn and Walters 2003) equilibruim
calculation for the delay difference model with \(F\)=0 to get the
unfished spawners per recruit ratio. This ratio was then used in a
second equilibruim calculation with the current predator exploitation
rate to estimate Beverton-Holt stock recruitment parameters, equilibrium
recruitment and equilibrium individual weight under exploitation. Then,
each model was run forward for 150 years with output from the herring
operating model specifying the herring population characteristics.

\subsubsection{Tuna population model}\label{tuna-population-model}

Western Atlantic bluefin tuna population parameters were drawn from the
2014 stock assessment (ICCAT 2015), the growth curve from (Restrepo et
al. 2010), and recruitment parameters from a detailed examination of
alternative stock recruit relationships (Porch and Lauretta 2016).
Ultimately, the ``low recruitment'' scenario was selected to represent
bluefin tuna productivity in the Gulf of Maine, which defines Bmsy as
13,226 t and therefore affects measures of status relative to Bmsy.
Continuation of the current tuna fishing strategy (F\textless{}0.5Fmsy
under the low recruitment scenario) is assumed. All predator population
model parameters are listed in Table \ref{predpars}.

\subsubsection{Herring-tuna relationship
model}\label{herring-tuna-relationship-model}

Tuna diets are variable depending on location and timing of foraging
(Chase 2002, Golet et al. 2013, 2015, Logan et al. 2015), but for the
purposes of this analysis, we assumed that herring is an important
enough prey of tuna to impact tuna growth in the Northeast US shelf
ecosystem. A relationship between bluefin tuna growth and herring
average weight was implemented based on information and methods in Golet
et al. (2015). The relationship between tuna condition anomaly (defined
as proportional departures from the weight-at-length relationship used
in the assessment) and average weight of tuna-prey-sized herring
(\(Havgwt_{y}\), herring \textgreater{}180 mm collected from commercial
herring fisheries) was modeled as a generalized logistic function with
lower and upper bounds on tuna growth parameters:

\begin{equation}
AnnAlpha_{y} = (0.9 FWint) + \frac{(1.1 FWint) - (0.9 FWint)}{1+e^{(1-\lambda)*(100(Havgwt_{y}-T)/T)}} \label{tunagrow_equation},
\end{equation}

where \(\lambda\) \textgreater{} 1 links herring average weight
anomalies to tuna growth.

The inflection point of \(T\) = 0.15 kg average weight aligns with 0
tuna weight anomaly from Fig. 2C on p 186 in Golet et al. (2015), and
upper and lower bounds were determined by estimating the growth
intercept with weight at age 10\% higher or lower, respectively from the
average weight at age obtained by applying the length to weight
conversion reported in the 2014 stock assessment (ICCAT 2015) to the
length at age estimated from the Restrepo et al. (2010) growth curve
(Fig. \ref{herringtuna}). When included in the model with \(\lambda\) =
1.1 in equation \ref{tunagrow_equation}, the simulated variation in tuna
weight at age covered the observed range reported in Golet et al.
(2015).

\subsubsection{Tern population model}\label{tern-population-model}

There is no published stock assessment or population model for most
seabirds in the Northeast US. Therefore, Gulf of Maine common and Arctic
tern (\emph{Sterna paradisaea}) population parameters were drawn from
accounts in the Birds of North America (Hatch 2002, Nisbet 2002) and
estimated from counts of breeding pairs and estimates of fledgling
success summarized by the Gulf of Maine Seabird Working Group (GOMSWG;
data at \url{http://gomswg.org/minutes.html}), as corrected and updated
by seabird experts from throughout Maine. While we initially analyzed
both Arctic and common tern information, the stakeholder workshop
identified common terns as the example species for modeling, and this
species has more extensive data and a generally higher proportion of
herring in its diet based on that data. Therefore, this predator model
is based on common terns in the Gulf of Maine.

Adult breeding pairs by colony were combined with estimated productivity
of fledglings per nest to estimate the annual number of fledglings for
each year. A survival rate of 10\% was applied to fledglings from each
year to represent ``recruits'' to the breeding adult population age 4
and up (Nisbet 2002). This ``stock-recruit'' information was used to
estimate steepness for the delay difference model based on common tern
information only. Fitting parameters with R nls (R Core Team 2016) had
variable success, with the full dataset unable to estimate a significant
beta parameter (dashed line, Figure \ref{ternSR}) for common terns, and
a truncated dataset resulting in low population production rates
inconsistent with currently observed common tern trends (dotted line,
Figure \ref{ternSR}). Therefore, steepness was estimated to give a
relationship (solid line, Fig. \ref{ternSR}) falling between these two
lines. The resulting stock recruit relationship set steepness at 0.26, a
theoretical maximum breeding adult population of 45,000 pairs (Nisbet
(2002), 1930's New England population), and a theoretical maximum
recruitment of 4,500 individuals annually (reflecting approximately a
productivity of 1.0 at ``carrying capacity'' resulting in a stable
population). Average common tern productivity is 1.02 (all Gulf of Maine
colony data combined). Adult mortality was assumed to be 0.1 for the
delay difference model (survival of 90\% (Nisbet 2002) for adults). The
resulting model based on common tern population dynamics in the Gulf of
Maine (with no link to herring) predicts that the population will
increase to its carrying capacity under steady conditions over a 150
year simulation. The actual population has increased at
\textasciitilde{}2\% per year between 1998 and 2015 (GOMSWG data). Given
the lack of detailed demographic information in the delay-difference
model, this was considered a good representation of the average observed
trend in current common tern population dynamics.

\subsubsection{Herring-tern relationship
model}\label{herring-tern-relationship-model}

The relationship between herring abundance and tern reproductive success
was built based on information from individual colonies on annual
productivity, proportion of herring in the diet, and amount of herring
in the population as estimated by the current stock assessment. Since
little of this information has appeared in the peer-reviewed literature,
we present it in detail here. First, productivity information was
evaluated by major diet item recorded for chicks over all colonies and
years. In general, common tern productivity was higher when a
streamlined fish species (``hake'', ``herring'', and ``sandlance'' in
Fig. \ref{chickdiet}) was the major diet item relative to invertebrates
(``amphipod'', ``euphausiid'', ``inverts''=unidentified invertebrates in
Fig. \ref{chickdiet}). However, having herring as the major diet item
resulted in about the same distribution of annual productivities as
having unidentified juvenile hake (\emph{Urophycis} or \emph{Merluccius}
sp.) or sandlance (\emph{Ammodytes} sp.) as the major diet item for
these colonies (Fig. \ref{chickdiet}).

Individual colonies showed different trends in number of nesting pairs,
productivity, and proportion of herring in the diet (plots available
upon request). When both Arctic and common terns shared a colony,
interannual changes in productivity were generally similar between
species, suggesting that conditions at and around the colony (weather,
predation pressure, and prey fields) strongly influenced productivity
rather than species-specific traits. Only two colonies (Machias Seal
Island near the US/Canada Border and Stratton Island in southern Maine)
showed a significant positive correlation between the proportion of
herring in the chick diet and productivity (Machias Seal: Spearman's
rank rho = 0.63, p = 0.019; Stratton: Spearman's rank rho = 0.52, p =
0.035). Other islands showed either non-significant (no) relationships,
or in one case (Metinic Island) a significant negative relationship
(Fig. \ref{proddiet}).

The estimated population size of herring on the Northeast US shelf had
some relationship to the amount of herring in tern diet at several
colonies (4 of 13 common tern colony diets related to herring Age 1
recruitment, 6 of 13 common tern colony diets related to herring total
biomass (B), and 4 of 13 common tern colony diets related to herring
SSB; detailed statistics and plots available upon request). However,
statistically significant direct relationships between herring
population size and tern productivity were rare, with only Ship Island
productivity increasing with herring total B, and Eastern Egg Rock,
Matinicus Rock, Ship, and Monomoy Islands productivity increasing with
herring SSB. Given that Monomoy Island tern chicks consistently
displayed the lowest proportion of herring in their diets of any colony
(0-11\%), we don't consider this relationship further to build the
model.

Based on tern feeding observations, we would expect the number of age 1
herring in the population to be most related to tern productivity since
that is the size class terns target, but this relationship was not found
in analyzing the data. Herring total biomass was positively related to
tern diets at nearly half of the colonies, and reflects all size classes
including the smaller sizes most useful as tern forage, but was only
directly related to tern productivity at one colony. Herring SSB was not
considered further as an index of tern prey because it represents sizes
larger than tern forage.

To represent the potential for herring to influence tern productivity,
we parameterized a tern ``recruitment multiplier'' based on herring
assessed total biomass and common tern productivity across all colonies
(except Monomoy where terns eat sandlance). This relationship includes a
threshold herring biomass where common tern productivity would drop
below 1.0, and above that threshold productivity exceeds 1.0 (Fig.
\ref{herrternmod}). The threshold of \textasciitilde{}400,000 tons is
set where a linear relationship between herring total biomass and common
tern productivity crosses productivity=1 (black dashed line in Fig.
\ref{herrternmod}). However, the selected threshold is uncertain because
there are few observations of common tern productivity at low herring
total biomass (1975-1985). The linear relationship does not have a
statistically significant slope; a curve was fit to represent a level
contribution of herring total biomass to common tern productivity above
the threshold. The curve descends below the threshold, dropping below
0.5 productivity at around 50,000 tons and representing the extreme
assumption that herring extinction would result in tern productivity of
0. Although the relationship of tern productivity to herring biomass at
extremely low herring populations has not been quantified, control rules
that allow herring extinction do not meet stated management objectives
for herring, so this extreme assumption for terns will not change any
decisions to include or exclude control rules.

When included in the model using \(\gamma\) = 1.09 in equation
\ref{recwithherring_equation}, this relationship adjusts the modeled
common tern population increase to match the current average increase in
common tern nesting pairs observed in the data (Fig.
\ref{terntrendwherring}). There is still considerable uncertainty around
this mean population trajectory which cannot be reflected in our simple
model.

\subsubsection{Groundfish}\label{groundfish}

Because no specific groundfish was identified as a representative
herring predator during the stakeholder workshop, the first decision was
which groundfish to model. Annual diet estimates (based on sample sizes
of \textasciitilde{}100+ stomachs) are available for the top three
groundfish predators of herring (those with herring occurring in the
diets most often in the entire NEFSC food habits database): spiny
dogfish (\emph{Squalus acanthias}, hereafter dogfish), Atlantic cod
(\emph{Gadus morhua}, hereafter cod), and silver hake (\emph{Merluccius
bilinearis}). Cod and spiny dogfish were considered first because their
overall diet proportions of herring are higher, and because silver hake
has the least recently updated assessment. Diet compositions by year
were estimated for spiny dogfish, Georges Bank cod, and Gulf of Maine
cod to match the scale of stock assessments. Full weighted diet
compositions were estimated, and suggest considerable interannual
variability in the herring proportion in groundfish diets (filled blue
proportions of bars in Fig. \ref{gfishdiets}).

Some interannual variation in diet may be explained by changing herring
abundance. Dogfish and both cod stocks had positive relationships
between the amount of herring observed in annual diets and the size of
the herring population according to the most recent assessment
(statistics and plots available upon request). This suggests that these
groundfish predators are opportunistic, eating herring in proportion to
their availability in the ecosystem. However, monotonically declining
cod populations for both Gulf of Maine and Georges Bank cod stocks
resulted in either no herring-cod relationship, or a negative
relationship between herring populations and cod populations (Fig.
\ref{gfishBherrdiet}). Only dogfish spawning stock biomass had a
positive relationship with the proporiton of herring in dogfish diet.
Therefore, we selected dogfish as the groundfish predator for modeling.

\subsubsection{Dogfish population model}\label{dogfish-population-model}

The dogfish model stock recruitment function, initial population, and
annual natural mortality were adapted from information in Rago et al.
(1998); Rago and Sosebee (2010); Bubley et al. (2012), and Rago and
Sosebee (2013). Due to differential growth and fishing mortality by sex,
our model best represents female dogfish (a split-sex delay difference
model was not feasible within the time constraints of this MSE).
Further, dogfish stock-recruit modeling to date based on Ricker
functions (Rago and Sosebee 2010) captures more nuances in productivity
than the Beverton-Holt model we used. Our recruitment parameterization
reflects a stock with generally low productivity and relatively high
resilience, which we recognize is a rough approximation for a species
such as dogfish. The annual fishing exploitation rate applied is the
average of the catch/adult female biomass from the most recent years of
the 2016 data update provided to the Mid-Atlantic Fishery Management
Council (Rago pers comm 2016).

\subsubsection{Herring-dogfish relationship
model}\label{herring-dogfish-relationship-model}

There was a weak positive relationship between dogfish total biomass and
herring total biomass from the respective stock assessments (Spearman's
rank correlation = 0.36, p = 0.012; Pearson's correlation = 0.32, p =
0.026), but no significant relationship between dogfish weight or
dogfish recruitment and herring population size. During the recent
period of relatively low dogfish recruitment (1995-2007), there was a
positive relationship between juvenile dogfish (pup) average weight and
herring proportion in diet, suggesting a potential growth and or
recruitment mechanism; however this relationship does not hold
throughout the time series, so we considered it too weak as a basis for
population modeling.

Therefore, to simulate a potential positive relationship between herring
and dogfish, we assumed that dogfish survival increased (natural
mortality was reduced) by an unspecified mechanism as herring abundance
increased (Fig. \ref{herringdogfish}). Because dogfish are fully
exploited by fisheries in this model, the impact of this change in
natural mortality on total survival has small to moderate benefits to
dogfish population numbers and biomass. Using a \(\delta\) = 0.2 in
equation \ref{varmort_equation} results in weak increases in dogfish
biomass with herring abundance consistent with observations.

\subsubsection{Marine mammals}\label{marine-mammals}

Because no specific marine mammal was identified as a representative
herring predator in the stakeholder workshop, as with groundfish, the
first decision was which marine mammal to model. Diet information for a
wide range of marine mammals on the Northeast US shelf suggests that
minke whales (\emph{Balaenoptera acutorostrata}), humpback whales
(\emph{Megaptera novaeangliae}), harbor seals (\emph{Phoca vitulina}),
and harbor porpoises (\emph{Phocoena phocoena}) have the highest
proportions of herring in their diets (Smith et al. 2015), and therefore
may show some reaction to changes in the herring ABC control rule.

While some food habits data existed for marine mammals, consultation
with marine mammal stock assessment scientists at the Northeast
Fisheries Science Center confirmed that no data were available to
parameterize a stock-recruitment relationship for any of these marine
mammal species in the Northeast US region, and no such information was
available in the literature for stocks in this region. Althouth it may
be possible to develop stock-recruitment models for one or more of these
species in the future, it was not possible within the time frame of the
herring MSE. Therefore, we were unable to model marine mammals within
the same framework as other predators.

Potential effects of changes in herring production and/or biomass on
marine mammals were instead evaluated using an updated version of an
existing food web model for the Gulf of Maine (Link et al. 2006, 2008,
2009) and incorporating food web model parameter uncertainty. Overall,
food web modeling showed that a simulated increase in herring production
in the Gulf of Maine may produce modest but uncertain benefits to marine
mammal predators, primarily because increased herring was associated
with decreases in other forage groups also preyed on by marine mammals.
However, this could not be persued further within the MSE framework.

Predator model input parameters are summarized in Table \ref{predpars}.

\subsection{Economics}\label{economics}

\subsubsection{The Herring Fishery}\label{the-herring-fishery}

The economic model of the herring fishery converts yield, \(Y\), from
the herring model component into Gross Revenues (GR) and Net Operating
Revenues (NR). There are two fleets,trawl and purse seine, that are
assumed to have the ability to catch 70 and 30\% of the yield,
respectively. This division corresponds to recent historical patterns.
The midwater trawl, paired midwater trawl, and bottom trawl are all
aggregated into the trawl fleet. Gross Revenue, Net Revenue, and the
constraints on harvest can be represented as (year subscripts omitted
here for simplicity):

\begin{align}
GR&=p(q^t+q^s)q^t+p(q^t+q^s)q^s\\
NR&=GR-c^t(q^t)-c^s(q^s)
\end{align}

where \(q^i\) is the quantity landed for fleet \(i\), \(c^i(q^i)\) is
cost function for fleet \(i\), and \(p(\cdot)\) is a function that
relates total landings to prices.\\

\begin{align}
\label{optimization_problem}
\max_{q_i} NR^i&=p(q^t+q^s)q^i - c^i(q^i)\\
q^s &\leq .3yield; \hspace{2mm} q^t \leq .7yield 
\end{align}

The optimization problem in equation \ref{optimization_problem} contains
two embedded assumptions: total catch is less than or equal to yield and
that a fleet may catching less than its fraction of yield (presumably
because it may be more profitable to select a lower level of landings).

Economic data collected from 2011-2015 by the Northeast Fisheries
Observer Program (NEFOP) were used to construct average daily costs for
the trawl and purse seine fleets. Fuel prices were much lower in
2011-2014 compared to 2015. We adjusted fuel prices to the 2011-2014
average; sensitivity analysis was performed by setting fuel prices to
the 2015 levels but results not reported here. Other costs of fishing
included water, oil, and damage costs. Crew pay and fixed costs were not
included.

We construct average catch per day fished for each fleet from the Vessel
Trip Report (VTR) databases over the same time period. The trip lengths
in the VTR and observer data were very similar. This allows us to
construct the average cost of catching a metric ton of herring for the
trawl and purse seine fleet (\(c^t\) and \(c^s\) respectively). We
assume that the average cost is equal to the marginal cost for each
fleet. These figures are presented in Table \ref{costs_combined}.

Annual prices were constructed from NMFS dealer data for 1982 through
2016. Annual landings were constructed from the processed ME DMR
landings dataset for the same time period\footnote{Prices have been
  normalized to 2015 real dollars using the Bureau of Labor Statistics
  (BLS) Producer Price Index (PPI) for Unprocessed and Packaged Fish
  (WPU0223). Because Atlantic Herring was not federally managed prior to
  the implementation of the Herring FMP in 2000; the NMFS dealer
  databases may not contain all landings prior to this time. The ME DMR
  data do not contain prices but is a census of landings.}. Exploratory
analysis suggested both a regime change in the mid-1990s and likely
non-stationarity of both landings and prices. We used the testing
methodology developed by Pesaran et al. (2001) to examine the existence
of a long-run relationship between prices and quantities. This method
does not require pretesting for stationarity; however, the test
statistic does have an inconclusive zone in which knowledge of
stationarity would be required. A long-run relationship between prices
and landings can be modeled as:

\begin{equation}
\label{ARDL}
p_{y}= c+\sum_{i=1}^p a_i p_{t-i}+ \sum_{i=0}^n b_i q_{y-i} + e_{t},
\end{equation}

or equivalently as an error correction model (ECM):

\begin{equation}
\label{ECM}
\Delta p_y= \gamma+ \alpha_1 p_{y-1} + \sum_{i=1}^p \theta_i \Delta p_{y-i}+ \sum_{i=0}^n  \beta_i \Delta q_{y-i} + e_{y},
\end{equation}

where \(\Delta\) is the first-differences operator (Pesaran et al.
2001). The \(\gamma\) parameter must also be restricted
(\(\gamma=c / \alpha_1\)) for equations \ref{ARDL} and \ref{ECM} to be
equivalent. Pesaran et al. (2001) tests the null of no long-run
relationship using a joint F-test that the \(\alpha_1\) and \(\beta_i\)
parameters in equation \ref{ECM} are non-zero; however, the F-statistic
has a non-standard distribution with an inconclusive area.

Equation \ref{ECM} was first estimated on the full 1982-2015 dataset;
model selection criteria indicated a model with four lags of price and
no lags of quanties (p=4, n=0) was preferred and that prices and
quantities had no long-run relationship. We suspect this is likely
caused by a combination of overfitting of the model and a regimes shift
evident in the exploratory graphs. Rather than explore a regime
switching model, we simply estimated equation \ref{ECM} with
\(p=1, n=0\) on a subset of the data (1995-2015). If there was a regime
shift, the current regime is more likely to be similar to the future.
Models estimated in natural logarithms and in levels fit well. The PSS
F-statistics of 9.34 and 10.20 are above the upper critical value,
strongly suggesting a long-run relationship between prices and
quantities\footnote{As a robustness check, we also tried varying the
  first (1996) and last (2016) year included in the dataset. This did
  not change the estimated results substantially. We also estimated a
  short-run relationship between prices and quantities in which
  \(\Delta p_y\) was regressed on \(\Delta q_y\). The short-run effects
  were qualitatively similar to the ``long-run'' model in Table
  \ref{ardl_regression}.} (Table \ref{ardl_regression}). We also present
the results of the ARDL(1,0) formulation because it is a bit easier to
interpret. Coefficients from the ``level'' equation (Column 1 of Table
\ref{ardl_regression}) are used in the simulation.

The simulation of Gross operating revenues occurs in a few steps.
Herring prices in a year are simulated using equations
\ref{optimization_problem} and \ref{ARDL} and parameters from Table
\ref{ardl_regression} previous year prices (initialized to the 2011-2015
average for the first year) under the assumption that both fleets
combine to land the entire Yield. Following Lehuta et al. (2013), if the
price of herring is sufficiently high, we assume that consumers find it
worthwhile to switch to menhaden. If the simulated prices are higher
than the price of menhaden (\$242/mt) plus transport costs (\$133/mt),
we set the price of herring to \$375/mt. When simulated prices are
higher than the marginal cost of the trawl fleet(\$63.24/mt), both
fleets are assumed to catch the entire Yield. Otherwise, we use
equations \ref{optimization_problem} and \ref{ARDL} to solve for
quantity landed by the trawl fleet when the purse seine fleet lands 30\%
of yield. If it optimal for the trawl fishery to land nothing, we use
equations \ref{optimization_problem} - \ref{ARDL} to find the purse
seine's optimal amount of landings. Because the marginal costs for the
purse seine are always less than the marginal costs of the trawl
fishery, any landings by the trawl fleet imply the purse seine fleet is
landing 30\% of the yield. Net revenues to the fishery can then be
calculated directly from equation \ref{optimization_problem}.

\subsection{Performance Metrics}\label{performance-metrics}

For each combination of control rule and operating model (43,680 unique
combinations), 100 simulations lasting 150 years were conducted.
Preliminary simulations suggested that this number of simulations and
years was sufficient for results to be insensitive to starting
conditions and short-term dynamics caused by auto-correlated processes.
The simulated herring time series for every operating model and control
rule was passed to the predator and economic submodels, resulting in
outputs as described below using the equations above. We report
performance metrics over the final fifty years as a way to describe the
long-run performance of a particular control rule. Performance metrics
were derived directly from the results of the stakeholder workshop and
supplemented with additional metrics drawn from MSE best practices (Punt
et al. 2016a).

\subsubsection{Herring performance
metrics}\label{herring-performance-metrics}

Median SSB, \(\frac{SSB}{SSB_{F=0}}\), \(\frac{SSB}{SSB_{MSY}}\),
\emph{yield}, \(\frac{yield}{MSY}\), biomass of herring dying due to
\(M\), and the proportion of the herring population comprised of age-1
fish were recorded as herring performance metrics. Additional
performance metrics included the proportion of years with
\(SSB < SSB_{MSY}\), \(SSB < \frac{SSB_{MSY}}{2}\) (i.e., proportion
years the the stock is overfished), \(SSB < 0.3SSB_{F=0}\),
\(SSB < 0.75SSB_{F=0}\), fully-selected \(F > F_{MSY}\) (i.e.,
proportion of years that overfishing occurred), and the proportion of
years \emph{Q}=0 (i.e., proportion of years that the fishery was
closed). Interannual variation in yield (\emph{IAV}) was also recorded:

\begin{equation}
IAV=\frac{\sqrt{\frac{1}{50}\sum\limits_{y=1}^{50}(Y_{y+1}-Y_{y})^2}}{(\frac{1}{50}\sum\limits_{y=1}^{50}Y_y)}.
\end{equation}

\subsubsection{Predator Performance
Metrics}\label{predator-performance-metrics}

Population abundance and recruitment were direct outputs for all modeled
predators. Population biomass was directly output for tuna and dogfish.
Stakeholders were interested in predator condition for fish and marine
mammal predators at the first workshop. While delay difference models do
not track individuals or age cohorts, a measure of population average
weight (population biomass/population numbers) was output for tuna and
dogfish.

Productivity, the number of fledglings per breeding pair, was output for
the tern model. Productivity was calculated as adult recruitment times
10 (to account for the 10\% survival rate of fledglings to adults)
divided by tern abundance 4 years earlier in the simulation.

Stakeholders were interested in different measures of population status
depending on the predator. For commercially fished species, status
relative to current management reference points was preferred. Tuna and
dogfish biomass was divided by a biomass reference point specified in
current stock assessments: tuna \(SSB_{MSY}^P\) was 13226 (ICCAT 2015),
and dogfish \(SSB_{MSY}^P\) was 159288 (Rago and Sosebee 2010). Because
dogfish were fully exploited in our model, they did not reach
\(SSB_{MSY}^P\), so we also evaluated status relative to 0.5
\(SSB_{MSY}^P\) (``overfished''). Tuna condition status was assessed by
dividing the output population average weight with the equilibrium
average weight. Common tern colonies are managed to improve
productivity, so stakeholders suggested that a common tern productivity
level of 0.8 would be a minimum threshold, while a productivity of 1.0
would be a target. In addition, total population status was measured
relative to current population numbers using the rationale that
maintaining at least the current population was desirable. The average
common tern population of nesting pairs (including Monomoy) from
1998-2015 was 16000.

Evaluating the frequency of desirable or undesirable states over the
course of a simulation is suggested by Punt et al. (2016a). We
calculated two metrics for each of the status determinations. First, we
calculated the minimum number of years in any individual simulation that
a metric was above a given threshold. This is a ``worst case scenario''
metric. Second, we calculated the median proportion of years across all
simulations for a control rule that were above the threshold. This is an
``average performance'' metric addressing how often good status is
maintained.

For all metrics other than ``frequency of good status'' metrics, we
report the median value for each simulation. Then, the 25th percentile,
the median, and the 75th percentile of these 100 medians were calcualted
to represent the performance metric for a particular control rule.
Results reported here focus on the median.

\subsubsection{Economic Performance
Metrics}\label{economic-performance-metrics}

Median Gross and Net Revenues were performance metrics that were
constructed directly from the economic submodel analogously to the way
predator and herring performance metrics were constructed. Stakeholders
were also interested in understanding stability of the herring industry.
We introduce a new stability performance metric that characterizes the
time-series of net revenues as in equilibrium or dis-equilibrium (Dickey
and Fuller 1979). For each simulation, we perform an econometric test of
stationarity (Dickey and Fuller 1979), by estimating:

\begin{equation}
\label{DF_estimated}
\Delta NR_{t} = \beta  NR_{t-1} + \xi_1 \Delta NR_{t-1} + \varepsilon. 
\end{equation}

Statistical evidence that \(\beta=0\) is evidence of dis-equilibrium
(nonstationarity) of NR while statistical rejections of \(H_0: \beta=0\)
in favor of \(H_A: \beta<0\) is evidence of equilibrium (stationarity).
Some of the control rules set quotas that are constant for three or five
years; for these policies, we aggregated Net Revenue into three or five
year blocks to examine the equilibrium properties across those blocks.
The results of these tests are summarized in two ways. First, we use the
unweighted Z-transform method from the meta-analysis literature to
combine the results of these simulations (Stouffer et al. 1949, Whitlock
2005). This allows for a test of the null hypothesis that a particular
control rule implemented on a particular operating model does not
produce a stable equilibrium. Defining \(\phi\) as the standard normal
cumulative distribution function, we construct:

\begin{align}
Z  &= \frac{\sum_{i=1}^k \phi^{-1} (1-p_i)}{\sqrt(k)}
\end{align}

\(Z\) has a standard normal distribution under the null hypothesis of
dis-equilibrium. We define the performance metric \emph{Equil1} as the
\(p\)-value associated with rejecting the null hypothesis that
particular control rule leads to dis-equilibrium of net revenue. Small
values of \emph{Equil1} are evidence of a stable equilbrium. As a
robustness check, we define the performance metric \emph{Equil2} as the
percentage of simulations in which we reject the null \(H_0: \beta=0\)
from equation \ref{DF_estimated}. Large values of \emph{Equil2} are
evidence of a stable equilbrium.

\section{Results}\label{results}

Some performance metrics were redundant, showed similar tradeoffs, or
were expected not to vary among operating models or control rules. For
example, the proportion of age-1 herring in the population was
insensitive to control rules because the fishery does not select age-1
fish. Such metrics were listed above to accurately document the outcome
of the stakeholder process, but results below were focused on metrics
likely of broad interest or with possible sensitivities to operating
models or control rules. A subset of eleven metrics is presented with
abbreviations for reference in figures: herring relative yield
(``relyeild'') \(\frac{yield}{MSY}\); herring interannual variation in
yield (``yieldvar'') \emph{IAV} above; the proportion of years
\emph{Q}=0 (``closure'') the herring fishery was closed; herring
relative spawning stock biomass (``relSSB'') \(\frac{SSB}{SSB_{F=0}}\);
the proportion of years the herring stock is overfished (``overfished'')
\(SSB < \frac{SSB_{MSY}}{2}\); probability of good tern productivity
(``ternprod'') the median proportion of years that tern
fledglings/breeding pair \textgreater{}= 1.0; probability of good
dogfish status (``dogstatus'') the median proportion of years that
dogfish SSB relative 0.5 \(SSB_{MSY}^P\) \textgreater{}= 1.0;
probability of good tuna weight status (``tunawt'') the median
proportion of years that tuna population weight (population population
biomass/population numbers) \textgreater{}= average; median herring
fishery net revenue (``net revenue'') as described above; herring
fishery stable equilibrium (``equil1'') if metric is small as described
above; and herring fishery stable equilibrium (``equil2'') if metric is
large as described above.

\subsection{Effect of Operating
Models}\label{effect-of-operating-models}

Several performance metrics performed similarly among operating models,
and were stable among control rules (Figure \ref{YAY}). Herring metrics
reported relative to reference points were generally less sensitive to
operating model than metrics in absolute units (Figure \ref{YAY}).
Consequently, results below focus on herring metrics reported relative
to reference points, with the understanding that the operating models
differ in meaningful ways (e.g., different \emph{MSY}) if metrics are
reported in absolute units. Dogfish performance metrics were also robust
to the operating models and control rules (Figure \ref{YAY}), and were
not considered further. Tuna were effected by variation in herring
growth among operating models, but did not vary among control rules
(Figure \ref{YAY}), and were also not considered further because herring
growth is not within the control of a management strategy (i.e., harvest
control rule). Other than tuna, most metrics were affected by
differences in herring \emph{M} and steepness more than herring growth
(Figure \ref{YAY}). The \emph{Equil1} and \emph{Equil2} metrics are
generally consistent with each other, and indicate that a stable
long-run equilibrium of net revenue is possible for any operating model
(Figure \ref{YAY}).

\subsection{Effect of Harvest Control
Rules}\label{effect-of-harvest-control-rules}

The constant catch, conditional constant catch, and the biomass based
control rule with a 15\% restriction, generally had less interannual
variation in yield than the other biomass based control rules (Figure
\ref{ZZZ}). This stability, however, came at the cost of foregone yield,
with fewer options for those three classes of control rule that could
achieve yields near \(MSY\) (Figure \ref{ZZZ}). The constant catch and
biomass based control rule with a 15\% restriction also had more
alternatives that led to poorer tern production than other control rules
(Figure \ref{ZZZ}). The conditional constant catch rule, however,
performed well for tern production because the cap of 0.5\(F_{MSY}\) on
fishing mortality effectively prevented levels of mortatliy that
decreased herring abundance to levels where tern production would be
compromised. Figure \ref{ZZZ} shows that the biomass based control rule
with a 15\% restriction would often lead to dis-equilibria of Net
Revenue and the HCR type responsible for the long right-tails seen in
the Figure \ref{YAY}. This finding of a disequilibria is likely
occurring because the 15\% restriction explicitly introduces memory into
the HCR, causing current period Net Revenue to be related to previous
period landings. The dis-equilibrium tests applied to harvest control
rules that have 3- or 5- year constant quotas use less data (the 10 or
16 year time periods corresponding to the final 50 years) and therefore
have lower statistical power than those applied to the HCRs that are set
annually. However, the standard BB3yr and BB5yr HCRs are characterized
by equilibrium (Figure \ref{ZZZ}), so we are doubtful that this finding
is caused by a statistical power problem. Ultimately, the NEFMC
eliminated the constant catch, conditional constant catch, and biomass
based rule with a 15\% restriction from consideration because of the
foregone yield and/or relatively poor performance of some alternatives
for the tern production metric.

Given that the metrics reported in results below focus on those that
were relatively robust to operating models, additional results were only
reported for the operating model with low \emph{M}, high steepness, and
slow growth (``HighSlowCorrect'' in figures). Further, due to Council
decisions to eliminate the constant catch, conditional constant catch,
and a biomass based rule with a 15\% restriction, more detailed tradeoff
analyses below focus on the biomass based control rule applied annually
(``BB''), or with three (``BB3yr'') or five year (``BB5yr'') quota
blocks.

\subsection{Tradeoffs}\label{tradeoffs}

At similar levels of yield, using three or five year blocks for the
biomass based control rule produced more control rule shapes with less
\(SSB\), such that the short-term stability of such quota blocks comes
at the potential cost of less \(SSB\) (Figure \ref{yieldSSB}). All three
biomass based variants had alternatives that could achieve yield near
\(MSY\) over a range of \(SSB\) levels.

At similar levels of yield, the application of quota blocks resulted in
lower extreme highs in \(IAV\) (Figure \ref{yieldIAV}). The application
of quota blocks also reduced the number of alternatives near \(MSY\),
further illustrating the tradeoff between stability and yield.

All three biomass based alternatives could achieve
\textgreater{}90\%\(MSY\) with nearly zero fishery closures, although
the number of alternatives was fewer with longer quota blocks (Figure
\ref{yieldPropClosure}). While the tradeoffs were generally similar
between \(\frac{yield}{MSY}\) and the frequncy that \emph{Q}=0 for all
three alternatives, vertical patterning becomes evident with longer
quota blocks. This patterning was caused by biomass based control rule
shapes becoming more alike with longer quota blocks, with their
performance dominated more by the length of fishery closures dictated by
the quota block length than by the specific shape of the control rule.
As an extreme example, if a 50 year quota block was applied, any control
rule that would close the fishery early in the time series would have a
nearly identical frequency of \emph{Q}=0 near 100\%, but these same
control rules would behave much differently with shorter quota blocks
where the responsiveness of the specific control rules would drive
results.

All of the control rule alternatives offered options that had near zero
frequency of \(SSB < \frac{SSB_{MSY}}{2}\) (Figure \ref{yieldProbOF}).
At similar levels of yield, using three or five year blocks for the
biomass based control rule produced more control rule shapes with higher
frequency of \(SSB < \frac{SSB_{MSY}}{2}\), such that the short-term
stability of such quota blocks comes at the potential cost of more
frequently dropping to relatively low levels of biomass.

All of the control rule alternatives offered options that had high
frequency of good tern productivity (equal to or greater than the
management target of 1.0; Figure \ref{yieldbirds}). At similar levels of
yield, using three or five year blocks for the biomass based control
rule produced more control rule shapes with slightly lower frequency of
good tern producivity, however, the difference is between
\textgreater{}80\% and \textgreater{}90\% frequency of good
productivity.

The control rule alternatives resulted in good stability characteristics
across a range of net revenues for the fishery (Figure
\ref{revenuestability}). We have included results from the biomass based
control rule with a 15\% restriction on interannual variation in the
quota to illustrate the relatively poor performance of this control rule
type.

\section{Discussion}\label{discussion}

Results were generally robust among operating models when presented
relative to biological reference points (e.g., \(\frac{yield}{MSY}\)),
but absolute scale differed among operating models. This result has been
previously reported (Deroba and Bence 2012) and is likely why the
results of most MSEs are presented in relative units (A'mar et al. 2009,
Deroba and Bence 2012, Wiedenmann et al. 2017). While convenient
scientifically, the differences in absolute scale can create challenges
when communicating with stakeholders because most people are more
comfortable with absolute units (e.g., fishing industry representatives
think in tons and not fractions of \(MSY\); Feeney et al. (In review)).
Thus, reporting results in absolute units will still have value in
public settings, and scientists should attempt to convey the reasons for
differences in scale among operating models (Feeney et al. In review).

The constant catch, conditional constant catch, and restricting annual
changes in the quota by 15\% produced less variable yield than the other
biomass based alternatives, but at the expense of yield, more frequent
low levels of herring biomass, and more outcomes relatively detrimental
to predators. This result is consistent with previous simulations of
similar harvest control rules for lake whitefish (\emph{Coregonus
clupeaformis}) in the Laurentian Great Lakes (Deroba and Bence 2012) and
roundfish stock managed by the International Council for the Exploration
of the Sea (Kell et al. 2006). For lake whitefish; Deroba and Bence
(2012) found that a conditional constant catch rule could achieve
similar stability in yield as a 15\% restriction on the interannual
change to a quota applied to a biomass based control rule, but with
higher yields. A broader range of percent restrictions on the
interannual variation in the quota should be evaluated, however, before
assuming that these conclusions are general. A 15\% restriction was used
in this analysis because that amount was specifed as desirable by
stakeholders. Conceivably, a different percentage may strike a more
agreeable tradeoff between stability and other metrics of interest. The
performance of restraints on the interannual variation in quotas also
depends on stock status and variation in life history traits, such as
growth (Punt et al. 2002 , Kell et al. 2006), and so should likely be
evaluated on a case by case basis.

If short-term stability in yield is a fishery objective, then using
quota blocks where the target harvest is the same for multiple years may
be an effective method that costs little in the performance of other
metrics. The biomass based options with three or five year quota blocks
produced similar ranges of performance and similar tradeoffs as annual
changes in the quota. Thus, short-term stability could be gained at
little long-term cost with appropriately selected harvest policy
parameters. Although, at similar level of yields, the quota block
alternatives produced more options with lower biomass. Thus, a
closed-loop simulation should be used to evaluate the changes in
relative performance between applying a control rule annually or using
multi-year specifications. In a simulation of lake whitefish, Li et al.
(2016) found that the method of specifying target harvests between stock
assessments was less important to relative performance than assessment
frequency. Furthermore, setting target harvest to the same value for
multiple years between assessments, as in the quota blocks used here,
performed similarly to using projections in the interim years.

Managers should understand the mechanisms by which ecosystem outcomes
affect stakeholders. Nearly every stakeholder group proposed metrics
that measured the health of the herring stock. The herring biomass
performance metric could capture passive (non-use) value associated with
the herring biomass. It may also be related to active use values if
higher herring biomass causes ecotourism, recreational, or commercial
fishery outcomes to be better. Finally, it may be a proxy for unmodeled
recreational, ecotourism, or socio-economic components of the ecosystem
that are believed to be positively associated with herring biomass.

\subsection{The Dream}\label{the-dream}

In this section, we discuss the limitations of the models and
simplifications necessary to meet decision-making time frames while
incorporating ecosystem and economic models into management strategy
evaluation.

\subsubsection{Herring Dreams}\label{herring-dreams}

The application of the biomass based and conditional constant catch
control rules assumed \(MSY\) reference points were known without error,
as in several other studies (Irwin et al. 2008, Punt et al. 2008). In
reality, such reference points are likely to be uncertain in most cases
and the bias and precision of reference point estimates depends on life
hisory traits and autocorrelation in recruitment (Haltuch et al. 2008,
2009). How the relative performance of control rules would change in the
presence of error in reference points is unclear and should be a topic
for future research. Given that assessment error affects relative
control rule performance (Deroba and Bence 2008), errors in reference
point estimation may compound those issues and affect results in
meaningful ways. Incorporating realistic errors in reference point
estimation, however, will be challenging because the effects will depend
on exploitation history and the degree to which life history traits vary
among years (Brodziak et al. 2008, Legault and Palmer 2015). This MSE
evaluated uncertainty in life history traits, but not time variation in
those traits. Methods for estimating reference points and the proper
response of managment in the presence of time varying life history
traits is an active area of research and has been a focal point of MSEs
(A'mar et al. 2009, Legault and Palmer 2015). For example, increases in
natural mortality through time have different implications on reference
points depending on whether a per recruit approach is used or if
managment is concerned with limiting total mortality, where a per
recruit approach would increase target \(F\) but limiting total
mortality would require decreasing target \(F\) (Legault and Palmer
2015). a MSE for Gulf of Alaska walleye pollock (\emph{Gadus
chalcogramma}) demonstrated that management strategies (i.e., the
combination of harvest control rule and estimation model) performed
differently depending on how reference points were estimated in the
presence of regime shifts in recruitment, with estimates of unfished
biomass depending on the range of years used to estimate average
recruitment (A'mar et al. 2009). Atlantic herring have experienced time
varying growth, and the possibility of time varying natural mortality
has been considered an uncertainty in recent stock assessments
(Northeast Fisheries Science Center 2012 , Deroba 2015). Thus,
evaluating the effect of errors in reference point estimation would be a
prudent advancement of the MSE. The stakeholder driven process, however,
was already overwhelming for many participants given the time frame, and
so adding this additional realism would likely be best completed as a
separate exercise that would rely moreso on technical experts than
stakeholder input.

This herring MSE did not incorporate a true stock assessment as part of
the managment strategy. The approach taken offered the advanatage of
simplicity, brevity, and the ability to control bias in the assessment
without having to specify a source of the bias (e.g., misspecified
\(M\), incorrect selectivity), as would be required with the
incorporation of a true stock assessment. This MSE also took advantage
of pre-existing and quality checked code that helped meet the
decision-making time frames. Amending this code to include a true stock
assessment likely would have precluded attaining the deadlines.
Conceivably, however, relative control rule performance could vary
depending on the mispecification that induces a bias in a true stock
assessment model, and so several alternative sources of bias would
likely have to be evaluated as part of a MSE, significantly increasing
computing time in an already demanding time frame. Given the technical
nature of stock assessments, constructing an evaluation of the effect of
different sources of bias in a stock assessment also could likely be
conducted outside of a stakeholder process. None the less, incorporating
a true stock assessment is considered best practice and may induce
realism, such as levels of autocorrelation among assessment fits, that
cannot be reproduced otherwise (Cox and Kronlund 2008, Punt et al.
2016a). Consequently, this MSE and subsequent managment decisions should
be revisited as such additional realism is implemented.

\subsubsection{Predator Dreams}\label{predator-dreams}

As has been found in other MSE analyses (Punt et al. 2016a), the
predator results may be more useful for eliminating poor control rule
options (BB3yrPerc, CC) than for optimizing herring control rules to
improve predator metrics. There are several reasons for this. Predator
populations are affected by many factors, while we attempted to isolate
factors associated with prey dynamics. Further, in the Northeast US,
predators have many prey options (Link 2002), while we attempted to
evaluate relationships with just one prey, herring. Finally, time
limitation necessitated simple, tractable models of complex ecological
relationships. Our approach was to use the best-supported relationship
for each predator based on observations from the Northeast US ecosystem.
We discuss the pros and cons of this approach for each predator below.

Western Atlantic bluefin tuna migrate widely and forage throughout North
Atlantic; their population footprint is much larger than that of
Northeast US Atlantic herring. However, tuna feed seasonally in the Gulf
of Maine, exploiting concentrated, high-energy prey to maximize growth
(Golet et al. 2013). Because tuna growth is key in the Northeast US, and
because there is a well-supported relationship between herring weight
and tuna growth (Golet et al. 2015), we used this relationship. Other
relationships were also investigated. Available data do not suggest a
positive relationship between herring and tuna populations in our models
for this MSE; Northeast US shelf herring have increased during a period
of bluefin tuna decline (Northeast Fisheries Science Center 2012, ICCAT
2015). Stakeholder observations and fine-scale analyses (e.g., Golet et
al. 2013) suggest that bluefin tuna follow herring in the Gulf of Maine
and likely aggregate around herring while feeding. However, our models
designed to address ABC control rules at the Northeast US shelf scale do
not address herring/tuna interactions in a specific place or time, and
we can draw no conclusions from our modeling about predator/prey
co-occurrence or availability at smaller, local scales. Similarly,
without additional observations, extrapolating local scale co-occurrence
to population level relationships is not well supported. Future
iterations of the MSE would require finer scale data on predators, prey,
and fisheries for both to address these questions.

Common terns, in contrast, are central-placed foragers seasonally near
their island breeding colonies in the Gulf of Maine (Nisbet 2002). Their
foraging footprint during chick production season is much smaller than
the scale of the Northeast US Atlantic herring population. Because tern
productivity is a key management objective for tern colonies in the Gulf
of Maine, we used the substantial existing data to explore a
relationship between herring populations and tern reproductive success.
However, many factors other than herring abundance affect tern
production. According to Gulf of Maine Seabird Working Group minutes and
other work, predation by mammals, gulls, and other birds is a major
factor that most colony management aims to control (Donehower et al.
2007 , Scopel and Diamond 2017b). Further, timing of weather events and
prey availability is important for production, but difficult to quantify
at all colonies from current data (Scopel and Diamond 2017a). Similarly,
the relatively small spatial scale and depth distribution of prey
affects tern foraging success as well as the overall abundance of prey
(Scopel et al. 2017). At one colony during the same year, the proporiton
of herring in tern chick diets was much lower than the proportion of
herring in razorbill (\emph{Alca torda}) diets at the same colony;
razorbills are capable of deeper dives than terns (GOMSWG minutes).
Spatial variability of predation, weather, and prey distribution may
drive the high variation in observed herring population-tern
productivity relationship among colonies, similar to observations in
other ecosystems (Sydeman et al. 2015). This high variance in the
observations is not considered by the modeled herring-tern relationship.
Further, the tern model is optimistic about population trajectory
because it considers only herring total biomass effects on terns, and
does not model predation, habitat quantity and quality, etc. In future
iterations of the MSE, a better match of spatial scale for fishery
removals and seabird foraging (similar to tunas above) could better
investigate management options that are not easily addressed with a
stock-scale annual harvest control rule (Sydeman et al. 2017).

Spiny dogfish may have the best spatial footprint match with Atlantic
herring in the Northeast US of the three predators modeled. Dogfish
forage through same range as herring for most of the year. Considerable
information on dogfish diet has been collected over time in the region,
and there are adequate data to conduct a stock assessment. However, the
dogfish relationship assumes herring abundance improves dogfish survival
because no clear relationship was found with recruitment or growth.
Increased survival may not be the mechanism for the observed positive
influence of herring in diet on the dogfish population.

Our approach allowed ``bottom up'' effects of herring on predators to be
examined, which was the key management question. Although we selected
predators with high herring diet proportions, observed predator
population responses to herring alone do not dominate dynamics, and our
herring-predator relationship models reflect that. Predator responses to
aggregate prey dynamics may be much clearer than responses to individual
prey in the Northeast US ecosystem given its food web structure with
many alternative prey (Link 2002). Further, food web modeling explored
here and in other studies suggested that ``top down'' effects of
predators on herring, simultaneous interactions of multiple predators
with herring, and side effects on other forage species could be
important in this ecosystem (Link et al. 2006, 2008, 2009). While
modeling multispecies interactions is a more complex and time-consuming
undertaking, the results may give clearer advice for managers making
decisions regarding multiple simultaneously exploited prey and predators
within the ecosystem (DeWitt and Langerhans 2003, Lovvorn et al. 2013).

Here, the general objective for the Council was to answer ``how do
changes in herring population abundance affect predator populations?''
This is a different and more complex question than that addressed in the
2012 herring assessment ``how much herring is consumed by predators?''
Our MICE models were designed specifically for evaluating alternative
herring control rules, not predator stock assessment and population
prediction. Council specifications and time constraints did not permit
development of integrated multispecies models (existing models: Curti et
al. 2013, Gaichas et al. 2017) that account for predation mortality on
herring, but not ``bottom up'' herring impacts on predators), nor
spatial or seasonal models accounting for migrations of wide-ranging
predators into or out of the Northeast US shelf ecosystem. A MICE
approach could be taken to incorporate key multispecies feedbacks and
even broadly spatial interactions in future iterations of the MSE, as
was done for the California Current (Punt et al. 2016b).

We caution against generalizing results for these particular predators
to other predators, as population parameters and herring relationships
differ. Although considerable work has been done examining forage fish
fishing in many ecosystems (Cury et al. 2011, Essington et al. 2015,
Hilborn et al. 2017), and it can be tempting to generalize control rules
specific to forage fish across all ecosystems (Pikitch et al. 2012), our
results demonstrated that many potential control rules for Northeast US
herring gave equally good results for the modeled herring predators.
While we do not suggest that relationships we found here will hold for
predators and prey in other ecosystems, we wholeheartedly recommend the
use of ecosystem-specific data to evaluate forage fish harvest control
rules and tradeoffs between objectives on a case-by-case basis.

\subsubsection{Economic Dreams}\label{economic-dreams}

The economic model of the herring fishery did not include fixed costs.
If firms do not enter or exit, then the exclusion of fixed costs from
the model has minimal effect when comparing HCRs within or across OMs.
Net revenues would all be overestimated by the same fixed amount. The
stationarity metric would be unaffected; however, IAV constructed
without fixed costs will be smaller than the true IAV that contains
fixed costs. In reality, firms can enter and exit this industry.
Economic theory suggests that firms will enter (exit) if they anticipate
large positive (negative) profits over a particular planning horizon.
While herring is a limited-access fishery, less than three-quarters of
the permits vessels are active, suggesting that firms could enter.
Understanding exactly how these entry and exit decisions are made was
not possible on the timeline requested by NEFMC.

We also assume that marginal costs are equal to average variable costs,
constant for each fleet, and do not depend on the level of biomass. A
more rigorous approach might include estimating a (economic) production
function for the herring fishery; this was not done in the interest of
time. It is difficult to predict how estimating a true cost function and
integrating those results would change the results of the study.

Catch in the economic model can be different from both quota and yield
(from the herring model). This is frequently handled as symmetric
implementation error in the fisheries literature (as it is in this
model). However, the economic model suggests that either an assymetric
error term, inwhich the error depends on prices of inputs and outputs,
or a more integrated biological and economic model is warranted.

Consumer welfare measures could be determined from a demand curve for
herring. Equation \ref{ARDL} estimates a price-quantity relationship,
which is not necessarily a demand curve for herring. Rigorous estimation
of a demand curve for herring requires modeling all goods that are
substitutes for herring, including mackerel, menhaden, squid, and other
substitute baits. This was not done due to limited time available.
Kirkley et al. (2011) use a static input-output model to simulate the
effects of changes in herring quotas and predator biomass levels on the
New England economy. Because the Kirkley et al. (2011) analysis
suggested that the effects of changes in herring catch on other segments
of the economy are quite small, economic analysis was confined to the
herring fleet.

Economic methods that can inform ecosystem approaches to fisheries
management include portfolio methods (Edwards et al. 2004, Jin et al.
2016), coupled ecosystem-region models (Jin et al. 2003, 2012, Kirkley
et al. 2011), and bioeconomic models (Tschirhart 2000, Finnoff and
Tschirhart 2003, Brown et al. 2005, Lehuta et al. 2013) of varying
complexity. However, there was simply not enough time to employ these
methods, nor to link economic outcomes to sociocultural outcomes (Zador
et al. 2017). The largest limitation of the economic model is that only
the herring fishery is quantitatively modeled. Humans who indirectly use
\textit{in-situ} herring were not formally modeled.

The predator section models a few representative consumers of live
herring: terns, tuna, whales, and predatory fish. Ecosystem valuation
methods could be used to measure changes in outcomes for those species
in dollar value (Loomis and White 1996, Richardson and Loomis 2009, Lew
2015). People may derive value from changes in the status of these
predators through either use or non-use values. For example, people may
directly value higher abundances of an animal or protection of an
endangered species, even if they have no plans to watch or view them
(Lew et al. 2010, Lew and Wallmo 2017). Quantifying these values
typically is done using stated preference methods with data collected
using surveys; these studies are costly and time consuming to develop
and conduct rigorously. Benefit transfer, a method in which valuation
from previous studies is applied to a new study area, may be a way to
overcome these barriers (Navrud and Ready 2007, Johnston and Rosenberger
2010).

Stakeholders may also derive use value from changes in the modeled
species. Changes in herring biomass may change costs, catches, or prices
in the commercial fishery for a predator. For example, increases in
biomass of spiny dogfish could lead to both higher quotas and lower
costs to catch more abundant fish. Examining changes in costs would
require an economic model of production for the spiny dogfish fishery
(similar to the model not used for the herring fleet) (Holland and
Sutinen 1999, Hutniczak 2014, Reimer et al. 2017). Changes in product
quality could affect prices (Larkin and Sylvia 1999, Asche et al. 2015).
For example, because larger tuna receives higher prices, the effects of
changes in average weight could be deduced from existing hedonic models
(McConnell and Strand 2000, Carroll et al. 2001).

These changes are not confined to an extractive sector of the economy.
For example, changes in whale populations may change outcomes for
whale-watching customers. This type of value could be quantified using
both stated and revealed preference data, both of which typically
require collection of survey data (Larson et al. 2004). These studies
are often quite costly and time consuming to develop and conduct
rigorously. Perhaps more discouragingly, the precise good being valued
needs to be known quite early in the research process. In this
application, development of a valuation survey for terns would not have
been able to start until after the predator modeling was nearly
complete. Benefit transfer may be the only way to value some of the use
values on the timeframe required by NEFMC.

Note that increases in the biomass of a particular predator could be a
net ``bad'' for society. For example, an increase in the biomass of a
predator that is low-valued but skilled at consuming herring could
result in disproportionate increases in that low-valued predator. If
that low-valued predator is not a complete specialist (in consuming
herring), it may also drive down the biomass of high-valued predators.
The ability to manipulate the ecosystem with a prey-level ABC control
rule to achieve desirable outcomes depends on the rates at which these
increases in prey are converted into social utility. This conversion
depends on the ecosystem technology (conversion of prey into additional
biomass of high- and low-valued predators), human technology (conversion
of prey and predator biomass into catch or tourism), and human
preferences (converting catch or tourism into utility). Despite our
current efforts, many of these relationships are not particularly well
understood at this time.

In order to meet management timelines, the herring, predator, and
economic models were developed in parallel, and not in sequence. This
required an educated guess about which predators and predator outcomes
were sensitive to the range of HCRs. For example, prior to completing
the predator models, we did not anticipate that tuna weight would be
sensitive to various operating models (i.e., fast and slow herring
growth) but not the harvest control rules. Developing the herring,
predator, and economic models sequentially could have have allowed for a
model of tuna fishery that accounts for the size-dependent prices
(Carroll et al. 2001). In contrast, devoting scarce research time to
examine the costs of harvesting tuna would not have resulted in
performance measures that would help managers select among control rules
because changes herring abundance had relatively little effect on tuna.

The simulation model used in this MSE can be thought of as a model of
intermediate complexity (Plagányi et al. 2014). Future MSE models that
are designed to address the role of herring as forage in the ecosystem
should graduate to advanced complexity. These advances should include
more realstic models of the scientific process, herring stocks,
predator-prey relationships, human behavior, and ecosystem valuation
methods. However, linking the appropriate components to derive the
desired set of performance metrics for management decision making need
not be an insurmountable challenge. For example, fairly simple methods
were used to link environmental drivers, prey and predator species, and
detailed social and economic components of fisheries in the Gulf of
Alaska (Zador et al. 2017). The expert opinion of stakeholders involved
in the MSE can be used to specify similar conceptual models in New
England. In future iterations of the MSE, conceptual and qualitative
modeling might be used to map out where critical data and further
integrated model development would most efficiently address priority
management needs.

\section{Acknowledgements}\label{acknowledgements}

We thank the more than 80 individuals from the fishing industry, federal
and state agencies, academia, environmental organizations, and other
entities who gave their time to contribute to the workshops. Madeleine
Hall-Arber, Jessica Joyce, Laura Singer, and Tiffany Vidal provided
excellent workshop facilitation. Beth Goettel, Linda Welch, and Aly
McKnight contributed seabird data and expertise. This manuscript
benefitted from the peer review panel: Lisa Kerr (Chair), Gavin Fay,
Douglas Lipton, and John Wiedenmann. Statements herein should neither be
considered formal positions of the USGS, NEFMC or NMFS nor a consensus
of the participants. Any use of trade, firm, or product names is for
descriptive purposes only and does not imply endorsement by the U.S.
Government

\section{References}\label{references}

\hypertarget{refs}{}
\hypertarget{ref-Asche2015EconomicFisheries}{}
Asche, F., Chen, Y., and Smith, M.D. 2015. Economic incentives to target
species and fish size: Prices and fine-scale product attributes in
Norwegian fisheries. ICES Journal of Marine Science \textbf{72}(3).
doi:\href{https://doi.org/10.1093/icesjms/fsu208}{10.1093/icesjms/fsu208}.

\hypertarget{ref-Amar2009TheFishery}{}
A'mar, Z.T., Punt, A.E., and Dorn, M.W. 2009. The impact of regime
shifts on the performance of management strategies for the Gulf of
Alaska walleye pollock (Theragra chalcogramma) fishery. Canadian Journal
of Fisheries and Aquatic Sciences \textbf{66}(12): 2222--2242. NRC
Research Press.

\hypertarget{ref-brodziak2008goals}{}
Brodziak, J., Cadrin, S.X., Legault, C.M., and Murawski, S.A. 2008.
Goals and strategies for rebuilding new england groundfish stocks.
Fisheries Research \textbf{94}(3): 355--366. Elsevier.

\hypertarget{ref-Brown2005AFisheries}{}
Brown, G., Berger, B., and Ikiara, M. 2005. A predator-prey model with
an application to Lake Victoria fisheries. Marine Resource Economics
\textbf{20}(3): 221--247.

\hypertarget{ref-bubley_reassessment_2012}{}
Bubley, W.J., Kneebone, J., Sulikowski, J.A., and Tsang, P.C.W. 2012.
Reassessment of spiny dogfish Squalus acanthias age and growth using
vertebrae and dorsal-fin spines. Journal of Fish Biology \textbf{80}(5):
1300--1319.
doi:\href{https://doi.org/10.1111/j.1095-8649.2011.03171.x}{10.1111/j.1095-8649.2011.03171.x}.

\hypertarget{ref-butterworth2007management}{}
Butterworth, D.S. 2007. Why a management procedure approach? Some
positives and negatives. ICES Journal of Marine Science \textbf{64}(4):
613--617. Oxford University Press.

\hypertarget{ref-Carroll2001PricingManagement}{}
Carroll, M.T., Anderson, J.L., and Martínez-Garmendia, J. 2001. Pricing
US North Atlantic bluefin tuna and implications for management.
Agribusiness \textbf{17}(2): 243--254.

\hypertarget{ref-chase_differences_2002}{}
Chase, B.C. 2002. Differences in diet of Atlantic bluefin tuna (Thunnus
thynnus) at five seasonal feeding grounds on the New England continental
shelf. Fishery Bulletin \textbf{100}(2): 168--180. Available from
\url{http://aquaticcommons.org/15201/} {[}accessed 26 April 2016{]}.

\hypertarget{ref-clark2004conditional}{}
Clark, W.G., and Hare, S.R. 2004. A conditional constant catch policy
for managing the pacific halibut fishery. North American Journal of
Fisheries Management \textbf{24}(1): 106--113. Wiley Online Library.

\hypertarget{ref-collie_ecosystem_2016}{}
Collie, J.S., Botsford, L.W., Hastings, A., Kaplan, I.C., Largier, J.L.,
Livingston, P.A., Plagányi, É., Rose, K.A., Wells, B.K., and Werner,
F.E. 2016. Ecosystem models for fisheries management: Finding the sweet
spot. Fish and Fisheries \textbf{17}(1): 101--125.
doi:\href{https://doi.org/10.1111/faf.12093}{10.1111/faf.12093}.

\hypertarget{ref-cox2008practical}{}
Cox, S.P., and Kronlund, A.R. 2008. Practical stakeholder-driven harvest
policies for groundfish fisheries in british columbia, canada. Fisheries
Research \textbf{94}(3): 224--237. Elsevier.

\hypertarget{ref-curti_evaluating_2013}{}
Curti, K.L., Collie, J.S., Legault, C.M., Link, J.S., and Hilborn, R.
2013. Evaluating the performance of a multispecies statistical
catch-at-age model. Canadian Journal of Fisheries and Aquatic Sciences
\textbf{70}(3): 470--484.
doi:\href{https://doi.org/10.1139/cjfas-2012-0229}{10.1139/cjfas-2012-0229}.

\hypertarget{ref-cury_global_2011}{}
Cury, P.M., Boyd, I.L., Bonhommeau, S., Anker-Nilssen, T., Crawford,
R.J.M., Furness, R.W., Mills, J.A., Murphy, E.J., Osterblom, H.,
Paleczny, M., Piatt, J.F., Roux, J.-P., Shannon, L., and Sydeman, W.J.
2011. Global Seabird Response to Forage Fish Depletion--One-Third for
the Birds. Science \textbf{334}(6063): 1703--1706.
doi:\href{https://doi.org/10.1126/science.1212928}{10.1126/science.1212928}.

\hypertarget{ref-Deroba2014EvaluatingRho}{}
Deroba, J.J. 2014. Evaluating the consequences of adjusting fish stock
assessment estimates of biomass for retrospective patterns using Mohn's
Rho. North American Journal of Fisheries Management \textbf{34}(2):
380--390. Taylor \& Francis.

\hypertarget{ref-Deroba2015atlantic}{}
Deroba, J.J. 2015. Atlantic herring operational assessment report 2015.
US Dept of Commerce, Northeast Fisheries Science Center,166 Water
Street, Woods Hole, MA 02543.

\hypertarget{ref-Deroba2008}{}
Deroba, J.J., and Bence, J.R. 2008. A review of harvest policies:
understanding relative performance of control rules. Fisheries Research
\textbf{94}: 210--223. Elsevier.

\hypertarget{ref-Deroba2012Evaluating}{}
Deroba, J.J., and Bence, J.R. 2012. Evaluating harvest control rules for
lake whitefish in the great lakes: Accounting for variable life-history
traits. Fisheries Research \textbf{121}: 88--103. Elsevier.

\hypertarget{ref-dewitt_multiple_2003}{}
DeWitt, T.J., and Langerhans, R.B. 2003. Multiple prey traits, multiple
predators: Keys to understanding complex community dynamics. Journal of
Sea Research \textbf{49}(2): 143--155.
doi:\href{https://doi.org/10.1016/S1385-1101(02)00220-4}{10.1016/S1385-1101(02)00220-4}.

\hypertarget{ref-Dickey1979DistributionRoot}{}
Dickey, D.A., and Fuller, W.A. 1979. Distribution of the estimators for
autoregressive time series with a unit root. Journal of the American
statistical association \textbf{74}(366a): 427--431. Taylor \& Francis.

\hypertarget{ref-donehower_effects_2007}{}
Donehower, C.E., Bird, D.M., Hall, C.S., and Kress, S.W. 2007. Effects
of Gull Predation and Predator Control on Tern Nesting Success at
Eastern Egg Rock, Maine. Waterbirds \textbf{30}(1): 29--39.
doi:\href{https://doi.org/10.1675/1524-4695(2007)030\%5B0029:EOGPAP\%5D2.0.CO;2}{10.1675/1524-4695(2007)030{[}0029:EOGPAP{]}2.0.CO;2}.

\hypertarget{ref-Edwards2004PortfolioStocks}{}
Edwards, S.F., Link, J.S., and Rountree, B.P. 2004. Portfolio Management
of Wild Fish Stocks. Ecological Economics \textbf{49}: 317--329.

\hypertarget{ref-essington_fishing_2015}{}
Essington, T.E., Moriarty, P.E., Froehlich, H.E., Hodgson, E.E., Koehn,
L.E., Oken, K.L., Siple, M.C., and Stawitz, C.C. 2015. Fishing amplifies
forage fish population collapses. Proceedings of the National Academy of
Sciences \textbf{112}(21): 6648--6652.
doi:\href{https://doi.org/10.1073/pnas.1422020112}{10.1073/pnas.1422020112}.

\hypertarget{ref-Feeney2018blending}{}
Feeney, R.G., Boelke, D., Deroba, J.J., Gaichas, S.K., Irwin, B., and
Lee, M.-Y. In review. Blending management strategy evaluation into the
us federal fisheries management process: Lessons from a stakeholder
driven mse for northeast us atlantic herring.

\hypertarget{ref-Finnoff2003HarvestingEcosystem}{}
Finnoff, D., and Tschirhart, J. 2003. Harvesting in an eight-species
ecosystem. Journal of Environmental Economics and Management
\textbf{45}(3): 589--611. Academic Press Orlando, USA.

\hypertarget{ref-francis1992use}{}
Francis, R.I.C. 1992. Use of risk analysis to assess fishery management
strategies: A case study using orange roughy (hoplostethus atlanticus)
on the chatham rise, new zealand. Canadian Journal of Fisheries and
Aquatic Sciences \textbf{49}(5): 922--930. NRC Research Press.

\hypertarget{ref-gaichas_combining_2017}{}
Gaichas, S.K., Fogarty, M., Fay, G., Gamble, R., Lucey, S., and Smith,
L. 2017. Combining stock, multispecies, and ecosystem level fishery
objectives within an operational management procedure: Simulations to
start the conversation. ICES Journal of Marine Science \textbf{74}(2):
552--565.
doi:\href{https://doi.org/10.1093/icesjms/fsw119}{10.1093/icesjms/fsw119}.

\hypertarget{ref-golet_paradox_2015}{}
Golet, W., Record, N., Lehuta, S., Lutcavage, M., Galuardi, B., Cooper,
A., and Pershing, A. 2015. The paradox of the pelagics: Why bluefin tuna
can go hungry in a sea of plenty. Marine Ecology Progress Series
\textbf{527}: 181--192.
doi:\href{https://doi.org/10.3354/meps11260}{10.3354/meps11260}.

\hypertarget{ref-golet_changes_2013}{}
Golet, W.J., Galuardi, B., Cooper, A.B., and Lutcavage, M.E. 2013.
Changes in the Distribution of Atlantic Bluefin Tuna ( Thunnus thynnus )
in the Gulf of Maine 1979-2005. PLOS ONE \textbf{8}(9): e75480.
doi:\href{https://doi.org/10.1371/journal.pone.0075480}{10.1371/journal.pone.0075480}.

\hypertarget{ref-hall_composition_2000}{}
Hall, C.S., Kress, S.W., and Griffin, C.R. 2000. Composition, Spatial
and Temporal Variation of Common and Arctic Tern Chick Diets in the Gulf
of Maine. Waterbirds: The International Journal of Waterbird Biology
\textbf{23}(3): 430.
doi:\href{https://doi.org/10.2307/1522180}{10.2307/1522180}.

\hypertarget{ref-haltuch2008evaluating}{}
Haltuch, M.A., Punt, A.E., and Dorn, M.W. 2008. Evaluating alternative
estimators of fishery management reference points. Fisheries Research
\textbf{94}(3): 290--303. Elsevier.

\hypertarget{ref-haltuch2009evaluating}{}
Haltuch, M.A., Punt, A.E., and Dorn, M.W. 2009. Evaluating the
estimation of fishery management reference points in a variable
environment. Fisheries Research \textbf{100}(1): 42--56. Elsevier.

\hypertarget{ref-hatch_arctic_2002}{}
Hatch, J.J. 2002. Arctic Tern (Sterna paradisaea). The Birds of North
America Online.
doi:\href{https://doi.org/10.2173/bna.707}{10.2173/bna.707}.

\hypertarget{ref-hayes_us_2017}{}
Hayes, S., Josephson, E., Maze-Foley, K., and Rosel, P.E.
(\emph{Editors}). 2017. US Atlantic and Gulf of Mexico Marine Mammal
Stock Assessments - 2016. NOAA Technical Memorandum NMFS-NE-241. US
DEPARTMENT OF COMMERCE. Available from
\url{https://repository.library.noaa.gov/view/noaa/14864} {[}accessed 7
June 2018{]}.

\hypertarget{ref-hilborn_when_2017}{}
Hilborn, R., Amoroso, R.O., Bogazzi, E., Jensen, O.P., Parma, A.M.,
Szuwalski, C., and Walters, C.J. 2017. When does fishing forage species
affect their predators? Fisheries Research \textbf{191}: 211--221.
doi:\href{https://doi.org/10.1016/j.fishres.2017.01.008}{10.1016/j.fishres.2017.01.008}.

\hypertarget{ref-hilborn_quantitative_2003}{}
Hilborn, R., and Walters, C.J. 2003. Quantitative Fisheries Stock
Assessment: Choice, Dynamics and Uncertainty. Springer Science \&
Business Media.

\hypertarget{ref-holland1999empirical}{}
Holland, D.S., and Sutinen, J.G. 1999. An empirical model of fleet
dynamics in new england trawl fisheries. Canadian Journal of Fisheries
and Aquatic Sciences \textbf{56}(2): 253--264. NRC Research Press.

\hypertarget{ref-hutniczak2014increasing}{}
Hutniczak, B. 2014. Increasing pressure on unregulated species due to
changes in individual vessel quotas: An empirical application to trawler
fishing in the baltic sea. Marine Resource Economics \textbf{29}(3):
201--217. University of Chicago Press Chicago, IL.

\hypertarget{ref-iccat_report_2015}{}
ICCAT. 2015. Report of the 2014 Atlantic bluefin tuna stock assessment
session (Madrid, Spain, 22--27 September 2014). Collected Volume
Scientific Papers \textbf{71}(2): 692--945. Available from
\url{https://www.iccat.int/Documents/Meetings/Docs/2014_BFT_ASSESS-ENG.pdf}.

\hypertarget{ref-irwin2008evaluating}{}
Irwin, B.J., Wilberg, M.J., Bence, J.R., and Jones, M.L. 2008.
Evaluating alternative harvest policies for yellow perch in southern
lake michigan. Fisheries Research \textbf{94}(3): 267--281. Elsevier.

\hypertarget{ref-Jin2016ApplyingEBFM}{}
Jin, D., DePiper, G., and Hoagland, P. 2016. Applying Portfolio
Management to Implement Ecosystem-Based Fishery Management (EBFM). North
American Journal of Fisheries Management \textbf{36}: 652--669.
doi:\href{https://doi.org/10.1080/02755947.2016.1146180}{10.1080/02755947.2016.1146180}.

\hypertarget{ref-Jin2003LinkingEcosystem}{}
Jin, D., Hoagland, P., and Dalton, T.M. 2003. Linking economic and
ecological models for a marine ecosystem. Ecological Economics.
doi:\href{https://doi.org/10.1016/j.ecolecon.2003.06.001}{10.1016/j.ecolecon.2003.06.001}.

\hypertarget{ref-Jin2012DevelopmentEngland}{}
Jin, D., Hoagland, P., Dalton, T.M., and Thunberg, E.M. 2012.
Development of an integrated economic and ecological framework for
ecosystem-based fisheries management in New England. Progress in
Oceanography \textbf{102}: 93--101. Elsevier.
doi:\href{https://doi.org/10.1016/j.pocean.2012.03.007}{10.1016/j.pocean.2012.03.007}.

\hypertarget{ref-johnston2010methods}{}
Johnston, R.J., and Rosenberger, R.S. 2010. Methods, trends and
controversies in contemporary benefit transfer. Journal of Economic
Surveys \textbf{24}(3): 479--510. Wiley Online Library.

\hypertarget{ref-katsukawa2004numerical}{}
Katsukawa, T. 2004. Numerical investigation of the optimal control rule
for decision-making in fisheries management. Fisheries science
\textbf{70}(1): 123--131. Wiley Online Library.

\hypertarget{ref-Kell2006}{}
Kell, L., Pilling, G., Kirkwood, G., Pastoors, M., Mesnil, B.,
Korsbrekke, K., Abaunza, P., Aps, R., Biseau, A., Kunzlik, P., Needle,
C., Roel, B., and Ulrich, C. 2006. An evaluation of multi-annual
management strategies for ices roundfish stocks. ICES Journal of Marine
Science \textbf{63}: 12--24. Elsevier.

\hypertarget{ref-Kirkley2011AConsiderations}{}
Kirkley, J.E., Walden, J., and Färe, R. 2011. A general equilibrium
model for Atlantic herring (Clupea harengus) with ecosystem
considerations. ICES Journal of Marine Science: Journal du Conseil
\textbf{68}(5): 860--866. Oxford University Press.

\hypertarget{ref-Larkin1999IntrinsicFishery}{}
Larkin, S.L., and Sylvia, G. 1999. Intrinsic fish characteristics and
intraseason production efficiency: a management-level bioeconomic
analysis of a commercial fishery. American Journal of Agricultural
Economics \textbf{81}(1): 29--43.

\hypertarget{ref-Larson2004RevealingData}{}
Larson, D.M., Shaikh, S.L., and Layton, D.F. 2004. Revealing Preferences
for Leisure Time from Stated Preference Data. American Journal of
Agricultural Economics \textbf{86}(2): 307--320. Blackwell Synergy.

\hypertarget{ref-legault2015direction}{}
Legault, C.M., and Palmer, M.C. 2015. In what direction should the
fishing mortality target change when natural mortality increases within
an assessment? Canadian journal of fisheries and aquatic sciences
\textbf{73}(3): 349--357. NRC Research Press.

\hypertarget{ref-Lehuta2013InvestigatingEngland}{}
Lehuta, S., Holland, D.S., and Pershing, A.J. 2013. Investigating
interconnected fisheries: a coupled model of the lobster and herring
fisheries in New England. Canadian Journal of Fisheries and Aquatic
Sciences \textbf{71}(2): 272--289. NRC Research Press.

\hypertarget{ref-lew2015willingness}{}
Lew, D.K. 2015. Willingness to pay for threatened and endangered marine
species: A review of the literature and prospects for policy use.
Frontiers in Marine Science \textbf{2}: 96. Frontiers.

\hypertarget{ref-lew2017temporal}{}
Lew, D.K., and Wallmo, K. 2017. Temporal stability of stated preferences
for endangered species protection from choice experiments. Ecological
Economics \textbf{131}: 87--97. Elsevier.

\hypertarget{ref-lew2010valuing}{}
Lew, D.K., Layton, D.F., and Rowe, R.D. 2010. Valuing enhancements to
endangered species protection under alternative baseline futures: The
case of the steller sea lion. Marine Resource Economics \textbf{25}(2):
133--154. University of Chicago Press Chicago, IL.

\hypertarget{ref-li2016influence}{}
Li, Y., Bence, J.R., and Brenden, T.O. 2016. The influence of stock
assessment frequency on the achievement of fishery management
objectives. North American Journal of Fisheries Management
\textbf{36}(4): 793--812. Wiley Online Library.

\hypertarget{ref-link_does_2002}{}
Link, J. 2002. Does food web theory work for marine ecosystems? Marine
ecology progress series \textbf{230}: 1--9. Available from
\url{http://www.int-res.com/abstracts/meps/v230/p1-9/} {[}accessed 20
February 2017{]}.

\hypertarget{ref-link_response_2009}{}
Link, J., Col, L., Guida, V., Dow, D., O'Reilly, J., Green, J.,
Overholtz, W., Palka, D., Legault, C., Vitaliano, J., Griswold, C.,
Fogarty, M., and Friedland, K. 2009. Response of balanced network models
to large-scale perturbation: Implications for evaluating the role of
small pelagics in the Gulf of Maine. Ecological Modelling
\textbf{220}(3): 351--369.
doi:\href{https://doi.org/10.1016/j.ecolmodel.2008.10.009}{10.1016/j.ecolmodel.2008.10.009}.

\hypertarget{ref-link_documentation_2006}{}
Link, J., Griswold, C., Methratta, E., and Gunnard, J. (\emph{Editors}).
2006. Documentation for the Energy Modeling and Analysis eXercise
(EMAX). Northeast Fish. Sci. Cent. Ref. Doc. 06-15. US Dep. Commer.,
National Marine Fisheries Service, Woods Hole, MA.

\hypertarget{ref-link_northeast_2008}{}
Link, J., Overholtz, W., O'Reilly, J., Green, J., Dow, D., Palka, D.,
Legault, C., Vitaliano, J., Guida, V., Fogarty, M., Brodziak, J.,
Methratta, L., Stockhausen, W., Col, L., and Griswold, C. 2008. The
Northeast U.S. continental shelf Energy Modeling and Analysis exercise
(EMAX): Ecological network model development and basic ecosystem
metrics. Journal of Marine Systems \textbf{74}(1--2): 453--474.
doi:\href{https://doi.org/10.1016/j.jmarsys.2008.03.007}{10.1016/j.jmarsys.2008.03.007}.

\hypertarget{ref-logan_diet_2015}{}
Logan, J.M., Golet, W.J., and Lutcavage, M.E. 2015. Diet and condition
of Atlantic bluefin tuna (Thunnus thynnus) in the Gulf of Maine,
2004--2008. Environmental Biology of Fishes \textbf{98}(5): 1411--1430.
doi:\href{https://doi.org/10.1007/s10641-014-0368-y}{10.1007/s10641-014-0368-y}.

\hypertarget{ref-loomis1996economic}{}
Loomis, J.B., and White, D.S. 1996. Economic benefits of rare and
endangered species: Summary and meta-analysis. Ecological Economics
\textbf{18}(3): 197--206. Elsevier.

\hypertarget{ref-lovvorn_niche_2013}{}
Lovvorn, J.R., Cruz, S.E.W.D.L., Takekawa, J.Y., Shaskey, L.E., and
Richman, S.E. 2013. Niche overlap, threshold food densities, and limits
to prey depletion for a diving duck assemblage in an estuarine bay.
Marine Ecology Progress Series \textbf{476}: 251--268.
doi:\href{https://doi.org/10.3354/meps10104}{10.3354/meps10104}.

\hypertarget{ref-McConnell2000HedonicHawaii}{}
McConnell, K.E., and Strand, I.E. 2000. Hedonic Prices for Fish: Tuna
Prices in Hawaii. American Journal of Agricultural Economics
\textbf{82}(1): 133--144. American Agricultural Economics Association.

\hypertarget{ref-navrud2007environmental}{}
Navrud, S., and Ready, R.C. 2007. Environmental value transfer: Issues
and methods. Springer.

\hypertarget{ref-nisbet_common_2002}{}
Nisbet, I.C.T. 2002. Common Tern (Sterna hirundo). The Birds of North
America Online.
doi:\href{https://doi.org/10.2173/bna.618}{10.2173/bna.618}.

\hypertarget{ref-NEFSC2012Assessment}{}
Northeast Fisheries Science Center. 2012. 54\(^{th}\) Northeast Regional
Stock Assessment Workshop (54th SAW) Assessment Report. US Dept of
Commerce, Northeast Fisheries Science Center,166 Water Street, Woods
Hole, MA 02543.

\hypertarget{ref-overholtz_consumption_2007}{}
Overholtz, W.J., and Link, J.S. 2007. Consumption impacts by marine
mammals, fish, and seabirds on the Gulf of Maine--Georges Bank Atlantic
herring (Clupea harengus) complex during the years 1977--2002. ICES
Journal of Marine Science: Journal du Conseil \textbf{64}(1): 83--96.
Available from
\url{http://icesjms.oxfordjournals.org/content/64/1/83.short}
{[}accessed 13 October 2016{]}.

\hypertarget{ref-overholtz_consumption_2000}{}
Overholtz, W.J., Link, J.S., and Suslowicz, L.E. 2000. Consumption of
important pelagic fish and squid by predatory fish in the northeastern
USA shelf ecosystem with some fishery comparisons. ICES Journal of
Marine Science: Journal du Conseil \textbf{57}(4): 1147--1159.
doi:\href{https://doi.org/10.1006/jmsc.2000.0802}{10.1006/jmsc.2000.0802}.

\hypertarget{ref-Pesaran2001BoundsRelationships}{}
Pesaran, M.H., Shin, Y., and Smith, R.J. 2001. Bounds testing approaches
to the analysis of level relationships. Journal of Applied Econometrics
\textbf{16}(3): 289--326.
doi:\href{https://doi.org/10.1002/jae.616}{10.1002/jae.616}.

\hypertarget{ref-pikitch_lenfest_2012}{}
Pikitch, E., Boersma, P., Boyd, I., Conover, D., Cury, P., Essington,
T., Heppell, S., Houde, E., Mangel, M., Pauly, D., Plagányi, É.,
Sainsbury, K., and Steneck, R. 2012. Little fish, big impact: Managing a
crucial link in ocean food webs. Lenfest Oceans Program, Washington, DC.

\hypertarget{ref-plaganyi_scotia_2012}{}
Plagányi, É.E., and Butterworth, D.S. 2012. The Scotia Sea krill fishery
and its possible impacts on dependent predators: Modeling localized
depletion of prey. Ecological Applications \textbf{22}(3): 748--761.
doi:\href{https://doi.org/10.1890/11-0441.1}{10.1890/11-0441.1}.

\hypertarget{ref-plaganyi_multispecies_2014}{}
Plagányi, É.E., Punt, A.E., Hillary, R., Morello, E.B., Thébaud, O.,
Hutton, T., Pillans, R.D., Thorson, J.T., Fulton, E.A., Smith, A.D.M.,
Smith, F., Bayliss, P., Haywood, M., Lyne, V., and Rothlisberg, P.C.
2014. Multispecies fisheries management and conservation: Tactical
applications using models of intermediate complexity. Fish and Fisheries
\textbf{15}(1): 1--22.
doi:\href{https://doi.org/10.1111/j.1467-2979.2012.00488.x}{10.1111/j.1467-2979.2012.00488.x}.

\hypertarget{ref-porch_making_2016}{}
Porch, C.E., and Lauretta, M.V. 2016. On Making Statistical Inferences
Regarding the Relationship between Spawners and Recruits and the
Irresolute Case of Western Atlantic Bluefin Tuna ( Thunnus thynnus ).
PLOS ONE \textbf{11}(6): e0156767.
doi:\href{https://doi.org/10.1371/journal.pone.0156767}{10.1371/journal.pone.0156767}.

\hypertarget{ref-punt_management_2016}{}
Punt, A.E., Butterworth, D.S., Moor, C.L. de, De Oliveira, J.A.A., and
Haddon, M. 2016a. Management strategy evaluation: Best practices. Fish
and Fisheries \textbf{17}(2): 303--334.
doi:\href{https://doi.org/10.1111/faf.12104}{10.1111/faf.12104}.

\hypertarget{ref-punt2008evaluation}{}
Punt, A.E., Dorn, M.W., and Haltuch, M.A. 2008. Evaluation of threshold
management strategies for groundfish off the us west coast. Fisheries
Research \textbf{94}(3): 251--266. Elsevier.

\hypertarget{ref-punt_exploring_2016}{}
Punt, A.E., MacCall, A.D., Essington, T.E., Francis, T.B.,
Hurtado-Ferro, F., Johnson, K.F., Kaplan, I.C., Koehn, L.E., Levin,
P.S., and Sydeman, W.J. 2016b. Exploring the implications of the harvest
control rule for Pacific sardine, accounting for predator dynamics: A
MICE model. Ecological Modelling \textbf{337}: 79--95.
doi:\href{https://doi.org/10.1016/j.ecolmodel.2016.06.004}{10.1016/j.ecolmodel.2016.06.004}.

\hypertarget{ref-punt2002evaluation}{}
Punt, A.E., Smith, A.D., and Cui, G. 2002. Evaluation of management
tools for australia's south east fishery. 3. towards selecting
appropriate harvest strategies. Marine and Freshwater Research
\textbf{53}(3): 645--660. CSIRO.

\hypertarget{ref-Rcite}{}
R Core Team. 2016. R: A language and environment for statistical
computing. R Foundation for Statistical Computing, Vienna, Austria.
Available from \url{https://www.R-project.org/}.

\hypertarget{ref-rago_biological_2010}{}
Rago, P., and Sosebee, K. 2010. Biological Reference Points for Spiny
Dogfish. Northeast Fish Sci Cent Ref Doc. 10-06: 52. Available from
\url{http://docs.lib.noaa.gov/noaa_documents/NMFS/NEFSC/NEFSC_reference_documnet/NEFSC_RD_10_12.pdf}
{[}accessed 19 November 2016{]}.

\hypertarget{ref-rago_update_2013}{}
Rago, P., and Sosebee, K. 2013. Update on the status of spiny dogfish in
2013 and projected harvests at the Fmsy proxy and Pstar of 40\%. Mid
Atlantic Fishery Management Council Scientific and Statistical
Committee. Available from
\url{https://www.mafmc.org/s/2015-Status-Report-and-Projections_final.pdf}
{[}accessed 15 November 2016{]}.

\hypertarget{ref-rago_implications_1998}{}
Rago, P.J., Sosebee, K.A., Brodziak, J.K.T., Murawski, S.A., and
Anderson, E.D. 1998. Implications of recent increases in catches on the
dynamics of Northwest Atlantic spiny dogfish (Squalus acanthias).
Fisheries Research \textbf{39}(2): 165--181. Available from
\url{http://www.sciencedirect.com/science/article/pii/S0165783698001817}
{[}accessed 15 November 2016{]}.

\hypertarget{ref-reimer2017fisheries}{}
Reimer, M.N., Abbott, J.K., and Wilen, J.E. 2017. Fisheries production:
Management institutions, spatial choice, and the quest for policy
invariance. Marine Resource Economics \textbf{32}(2): 143--168.
University of Chicago Press Chicago, IL.

\hypertarget{ref-restrepo_updated_2010}{}
Restrepo, V.R., Diaz, G.A., Walter, J.F., Neilson, J.D., Campana, S.E.,
Secor, D., and Wingate, R.L. 2010. Updated estimate of the growth curve
of Western Atlantic bluefin tuna. Aquatic Living Resources
\textbf{23}(4): 335--342.
doi:\href{https://doi.org/10.1051/alr/2011004}{10.1051/alr/2011004}.

\hypertarget{ref-richardson2009total}{}
Richardson, L., and Loomis, J. 2009. The total economic value of
threatened, endangered and rare species: An updated meta-analysis.
Ecological Economics \textbf{68}(5): 1535--1548. Elsevier.

\hypertarget{ref-scopel_predation_2017}{}
Scopel, L., and Diamond, A. 2017a. Predation and food--weather
interactions drive colony collapse in a managed metapopulation of Arctic
Terns (Sterna paradisaea). Canadian Journal of Zoology \textbf{96}(1):
13--22.
doi:\href{https://doi.org/10.1139/cjz-2016-0281}{10.1139/cjz-2016-0281}.

\hypertarget{ref-scopel_lauren_case_2017}{}
Scopel, L.C., and Diamond, A.W. 2017b. The case for lethal control of
gulls on seabird colonies. The Journal of Wildlife Management
\textbf{81}(4): 572--580.
doi:\href{https://doi.org/10.1002/jwmg.21233}{10.1002/jwmg.21233}.

\hypertarget{ref-scopel_seabird_2017}{}
Scopel, L.C., Diamond, A.W., Kress, S.W., Hards, A.R., and Shannon, P.
2017. Seabird diets as bioindicators of Atlantic herring recruitment and
stock size: A new tool for ecosystem-based fisheries management.
Canadian Journal of Fisheries and Aquatic Sciences: 1--15.
doi:\href{https://doi.org/10.1139/cjfas-2017-0140}{10.1139/cjfas-2017-0140}.

\hypertarget{ref-smith_trophic_2010}{}
Smith, B.E., and Link, J.S. 2010. The Trophic Dynamics of 50 Finfish and
2 Squid Species on the Northeast US Continental Shelf. NOAA Technical
Memorandum NMFS-NE-216. National Marine Fisheries Service, 166 Water
Street, Woods Hole, MA 02543-1026. Available from
\url{http://www.nefsc.noaa.gov/publications/tm/tm216/} {[}accessed 26
April 2016{]}.

\hypertarget{ref-smith_consumption_2015}{}
Smith, L.A., Link, J.S., Cadrin, S.X., and Palka, D.L. 2015. Consumption
by marine mammals on the Northeast U.S. continental shelf. Ecological
Applications \textbf{25}(2): 373--389.
doi:\href{https://doi.org/10.1890/13-1656.1}{10.1890/13-1656.1}.

\hypertarget{ref-stouffer1949american}{}
Stouffer, S.A., Suchman, E.A., DeVinney, L.C., Star, S.A., and Williams
Jr, R.M. 1949. The american soldier: Adjustment during army
life.(Studies in social psychology in World War II), vol. 1. Princeton
Univ. Press.

\hypertarget{ref-sydeman_best_2017}{}
Sydeman, W.J., Thompson, S.A., Anker-Nilssen, T., Arimitsu, M.,
Bennison, A., Bertrand, S., Boersch-Supan, P., Boyd, C., Bransome, N.C.,
Crawford, R.J., Daunt, F., Furness, R.W., Gianuca, D., Gladics, A.,
Koehn, L., Lang, J.W., Logerwell, E., Morris, T.L., Phillips, E.M.,
Provencher, J., Punt, A.E., Saraux, C., Shannon, L., Sherley, R.B.,
Simeone, A., Wanless, R.M., Wanless, S., and Zador, S. 2017. Best
practices for assessing forage fish fisheries-seabird resource
competition. Fisheries Research \textbf{194}: 209--221.
doi:\href{https://doi.org/10.1016/j.fishres.2017.05.018}{10.1016/j.fishres.2017.05.018}.

\hypertarget{ref-sydeman_climateecosystem_2015}{}
Sydeman, W.J., Thompson, S.A., Santora, J.A., Koslow, J.A., Goericke,
R., and Ohman, M.D. 2015. Climate--ecosystem change off southern
California: Time-dependent seabird predator--prey numerical responses.
Deep Sea Research Part II: Topical Studies in Oceanography \textbf{112}:
158--170.
doi:\href{https://doi.org/10.1016/j.dsr2.2014.03.008}{10.1016/j.dsr2.2014.03.008}.

\hypertarget{ref-Tschirhart2000GeneralEcosystem}{}
Tschirhart, J. 2000. General Equilibrium of an Ecosystem. Journal of
Theoretical Biology \textbf{203}(1): 13--32.

\hypertarget{ref-Whitlock2005CombiningApproach}{}
Whitlock, M.C. 2005. Combining probability from independent tests: The
weighted Z-method is superior to Fisher's approach. Journal of
Evolutionary Biology.
doi:\href{https://doi.org/10.1111/j.1420-9101.2005.00917.x}{10.1111/j.1420-9101.2005.00917.x}.

\hypertarget{ref-Wiedenmann2017}{}
Wiedenmann, J., Wilberg, M., Sylvia, A., and Miller, T. 2017. An
evaluation of acceptable biological catch harvest control rules designed
to limit overfishing. Canadian Journal of Fisheries and Aquatic Sciences
\textbf{74}: 1028--1040. NRC Research Press.

\hypertarget{ref-zador2017linking}{}
Zador, S.G., Gaichas, S.K., Kasperski, S., Ward, C.L., Blake, R.E., Ban,
N.C., Himes-Cornell, A., and Koehn, J.Z. 2017. Linking ecosystem
processes to communities of practice through commercially fished species
in the gulf of alaska. ICES Journal of Marine Science \textbf{74}(7):
2024--2033. Oxford University Press.

\newpage

\section{Tables}\label{tables}

\begin{longtable}[]{@{}ll@{}}
\caption{Table of Symbols \label{symtab}}\tabularnewline
\toprule
\begin{minipage}[b]{0.13\columnwidth}\raggedright\strut
Symbol\strut
\end{minipage} & \begin{minipage}[b]{0.76\columnwidth}\raggedright\strut
Definition\strut
\end{minipage}\tabularnewline
\midrule
\endfirsthead
\toprule
\begin{minipage}[b]{0.13\columnwidth}\raggedright\strut
Symbol\strut
\end{minipage} & \begin{minipage}[b]{0.76\columnwidth}\raggedright\strut
Definition\strut
\end{minipage}\tabularnewline
\midrule
\endhead
\begin{minipage}[t]{0.13\columnwidth}\raggedright\strut
\(y\)\strut
\end{minipage} & \begin{minipage}[t]{0.76\columnwidth}\raggedright\strut
year\strut
\end{minipage}\tabularnewline
\begin{minipage}[t]{0.13\columnwidth}\raggedright\strut
\(a\)\strut
\end{minipage} & \begin{minipage}[t]{0.76\columnwidth}\raggedright\strut
age\strut
\end{minipage}\tabularnewline
\begin{minipage}[t]{0.13\columnwidth}\raggedright\strut
\(R\)\strut
\end{minipage} & \begin{minipage}[t]{0.76\columnwidth}\raggedright\strut
herring recruitment\strut
\end{minipage}\tabularnewline
\begin{minipage}[t]{0.13\columnwidth}\raggedright\strut
\(SSB\)\strut
\end{minipage} & \begin{minipage}[t]{0.76\columnwidth}\raggedright\strut
herring spawning stock biomass\strut
\end{minipage}\tabularnewline
\begin{minipage}[t]{0.13\columnwidth}\raggedright\strut
\(F\)\strut
\end{minipage} & \begin{minipage}[t]{0.76\columnwidth}\raggedright\strut
fishing mortality\strut
\end{minipage}\tabularnewline
\begin{minipage}[t]{0.13\columnwidth}\raggedright\strut
\(h\)\strut
\end{minipage} & \begin{minipage}[t]{0.76\columnwidth}\raggedright\strut
steepness\strut
\end{minipage}\tabularnewline
\begin{minipage}[t]{0.13\columnwidth}\raggedright\strut
\(\varepsilon_R\)\strut
\end{minipage} & \begin{minipage}[t]{0.76\columnwidth}\raggedright\strut
recrutment process error\strut
\end{minipage}\tabularnewline
\begin{minipage}[t]{0.13\columnwidth}\raggedright\strut
\(\sigma_R^2\)\strut
\end{minipage} & \begin{minipage}[t]{0.76\columnwidth}\raggedright\strut
variance of recruitment process error\strut
\end{minipage}\tabularnewline
\begin{minipage}[t]{0.13\columnwidth}\raggedright\strut
\(\omega\)\strut
\end{minipage} & \begin{minipage}[t]{0.76\columnwidth}\raggedright\strut
autocorrelation of recruitment error\strut
\end{minipage}\tabularnewline
\begin{minipage}[t]{0.13\columnwidth}\raggedright\strut
\(N\)\strut
\end{minipage} & \begin{minipage}[t]{0.76\columnwidth}\raggedright\strut
herring abundance\strut
\end{minipage}\tabularnewline
\begin{minipage}[t]{0.13\columnwidth}\raggedright\strut
\(m\)\strut
\end{minipage} & \begin{minipage}[t]{0.76\columnwidth}\raggedright\strut
herring maturity\strut
\end{minipage}\tabularnewline
\begin{minipage}[t]{0.13\columnwidth}\raggedright\strut
\(W\)\strut
\end{minipage} & \begin{minipage}[t]{0.76\columnwidth}\raggedright\strut
herring weight\strut
\end{minipage}\tabularnewline
\begin{minipage}[t]{0.13\columnwidth}\raggedright\strut
\(\widehat{N}\)\strut
\end{minipage} & \begin{minipage}[t]{0.76\columnwidth}\raggedright\strut
assessed herring abundance\strut
\end{minipage}\tabularnewline
\begin{minipage}[t]{0.13\columnwidth}\raggedright\strut
\(\rho\)\strut
\end{minipage} & \begin{minipage}[t]{0.76\columnwidth}\raggedright\strut
bias in assessed herring abundance\strut
\end{minipage}\tabularnewline
\begin{minipage}[t]{0.13\columnwidth}\raggedright\strut
\(\varepsilon_{\phi}\)\strut
\end{minipage} & \begin{minipage}[t]{0.76\columnwidth}\raggedright\strut
assessment error\strut
\end{minipage}\tabularnewline
\begin{minipage}[t]{0.13\columnwidth}\raggedright\strut
\(\tau\)\strut
\end{minipage} & \begin{minipage}[t]{0.76\columnwidth}\raggedright\strut
autocorrelation of assessment error\strut
\end{minipage}\tabularnewline
\begin{minipage}[t]{0.13\columnwidth}\raggedright\strut
\(\sigma_{\phi}^2\)\strut
\end{minipage} & \begin{minipage}[t]{0.76\columnwidth}\raggedright\strut
variance of assessment error\strut
\end{minipage}\tabularnewline
\begin{minipage}[t]{0.13\columnwidth}\raggedright\strut
\(B\)\strut
\end{minipage} & \begin{minipage}[t]{0.76\columnwidth}\raggedright\strut
herring total biomass\strut
\end{minipage}\tabularnewline
\begin{minipage}[t]{0.13\columnwidth}\raggedright\strut
\(\widehat{SSB}\)\strut
\end{minipage} & \begin{minipage}[t]{0.76\columnwidth}\raggedright\strut
assessed herring spawning stock biomass\strut
\end{minipage}\tabularnewline
\begin{minipage}[t]{0.13\columnwidth}\raggedright\strut
\(\widehat{B}\)\strut
\end{minipage} & \begin{minipage}[t]{0.76\columnwidth}\raggedright\strut
assessed herring total biomass\strut
\end{minipage}\tabularnewline
\begin{minipage}[t]{0.13\columnwidth}\raggedright\strut
\(M\)\strut
\end{minipage} & \begin{minipage}[t]{0.76\columnwidth}\raggedright\strut
herring natural mortality\strut
\end{minipage}\tabularnewline
\begin{minipage}[t]{0.13\columnwidth}\raggedright\strut
\(Y\)\strut
\end{minipage} & \begin{minipage}[t]{0.76\columnwidth}\raggedright\strut
herring yield\strut
\end{minipage}\tabularnewline
\begin{minipage}[t]{0.13\columnwidth}\raggedright\strut
\(MSY\)\strut
\end{minipage} & \begin{minipage}[t]{0.76\columnwidth}\raggedright\strut
herring maximum sustainable yield\strut
\end{minipage}\tabularnewline
\begin{minipage}[t]{0.13\columnwidth}\raggedright\strut
\(SSB_{up}\)\strut
\end{minipage} & \begin{minipage}[t]{0.76\columnwidth}\raggedright\strut
upper biomass parameter for biomass based control rule\strut
\end{minipage}\tabularnewline
\begin{minipage}[t]{0.13\columnwidth}\raggedright\strut
\(SSB_{low}\)\strut
\end{minipage} & \begin{minipage}[t]{0.76\columnwidth}\raggedright\strut
lower biomass parameter for biomass based control rule\strut
\end{minipage}\tabularnewline
\begin{minipage}[t]{0.13\columnwidth}\raggedright\strut
\(\psi\)\strut
\end{minipage} & \begin{minipage}[t]{0.76\columnwidth}\raggedright\strut
proportion of \(F_{MSY}\) for biomass based control rule\strut
\end{minipage}\tabularnewline
\begin{minipage}[t]{0.13\columnwidth}\raggedright\strut
\(F_{MSY}\)\strut
\end{minipage} & \begin{minipage}[t]{0.76\columnwidth}\raggedright\strut
fishing mortality at herring maximum sustainable yield\strut
\end{minipage}\tabularnewline
\begin{minipage}[t]{0.13\columnwidth}\raggedright\strut
\(\tilde{F}\)\strut
\end{minipage} & \begin{minipage}[t]{0.76\columnwidth}\raggedright\strut
target herring fishing mortality\strut
\end{minipage}\tabularnewline
\begin{minipage}[t]{0.13\columnwidth}\raggedright\strut
\(S\)\strut
\end{minipage} & \begin{minipage}[t]{0.76\columnwidth}\raggedright\strut
herring fishery selectivity\strut
\end{minipage}\tabularnewline
\begin{minipage}[t]{0.13\columnwidth}\raggedright\strut
\(\bar{F}\)\strut
\end{minipage} & \begin{minipage}[t]{0.76\columnwidth}\raggedright\strut
fishing mortality that would remove quota from herring population\strut
\end{minipage}\tabularnewline
\begin{minipage}[t]{0.13\columnwidth}\raggedright\strut
\(Q\)\strut
\end{minipage} & \begin{minipage}[t]{0.76\columnwidth}\raggedright\strut
quota for herring fishery\strut
\end{minipage}\tabularnewline
\begin{minipage}[t]{0.13\columnwidth}\raggedright\strut
\(F\)\strut
\end{minipage} & \begin{minipage}[t]{0.76\columnwidth}\raggedright\strut
realized herring fishery mortality\strut
\end{minipage}\tabularnewline
\begin{minipage}[t]{0.13\columnwidth}\raggedright\strut
\(\varepsilon_{\theta}\)\strut
\end{minipage} & \begin{minipage}[t]{0.76\columnwidth}\raggedright\strut
implementation error\strut
\end{minipage}\tabularnewline
\begin{minipage}[t]{0.13\columnwidth}\raggedright\strut
\(\sigma_{\theta}^2\)\strut
\end{minipage} & \begin{minipage}[t]{0.76\columnwidth}\raggedright\strut
variance of implementation error\strut
\end{minipage}\tabularnewline
\begin{minipage}[t]{0.13\columnwidth}\raggedright\strut
\(P\)\strut
\end{minipage} & \begin{minipage}[t]{0.76\columnwidth}\raggedright\strut
denotes a quantity that applies to a predator of herring\strut
\end{minipage}\tabularnewline
\begin{minipage}[t]{0.13\columnwidth}\raggedright\strut
\(N^P\)\strut
\end{minipage} & \begin{minipage}[t]{0.76\columnwidth}\raggedright\strut
predator abundance\strut
\end{minipage}\tabularnewline
\begin{minipage}[t]{0.13\columnwidth}\raggedright\strut
\(S^P\)\strut
\end{minipage} & \begin{minipage}[t]{0.76\columnwidth}\raggedright\strut
annual predator survival\strut
\end{minipage}\tabularnewline
\begin{minipage}[t]{0.13\columnwidth}\raggedright\strut
\(R^P\)\strut
\end{minipage} & \begin{minipage}[t]{0.76\columnwidth}\raggedright\strut
predator recruitment from Beverton-Holt relationship\strut
\end{minipage}\tabularnewline
\begin{minipage}[t]{0.13\columnwidth}\raggedright\strut
\(\bar{R}^P\)\strut
\end{minipage} & \begin{minipage}[t]{0.76\columnwidth}\raggedright\strut
predator recruitment as modified by a relationship with herring\strut
\end{minipage}\tabularnewline
\begin{minipage}[t]{0.13\columnwidth}\raggedright\strut
\(v\)\strut
\end{minipage} & \begin{minipage}[t]{0.76\columnwidth}\raggedright\strut
annual predator natural mortality\strut
\end{minipage}\tabularnewline
\begin{minipage}[t]{0.13\columnwidth}\raggedright\strut
\(u\)\strut
\end{minipage} & \begin{minipage}[t]{0.76\columnwidth}\raggedright\strut
annual predator exploitation mortality\strut
\end{minipage}\tabularnewline
\begin{minipage}[t]{0.13\columnwidth}\raggedright\strut
\(Fwint\)\strut
\end{minipage} & \begin{minipage}[t]{0.76\columnwidth}\raggedright\strut
intercept of Ford-Walford relationship\strut
\end{minipage}\tabularnewline
\begin{minipage}[t]{0.13\columnwidth}\raggedright\strut
\(Fwslope\)\strut
\end{minipage} & \begin{minipage}[t]{0.76\columnwidth}\raggedright\strut
slope of Ford-Walford relationship\strut
\end{minipage}\tabularnewline
\begin{minipage}[t]{0.13\columnwidth}\raggedright\strut
\(B^P\)\strut
\end{minipage} & \begin{minipage}[t]{0.76\columnwidth}\raggedright\strut
predator biomass\strut
\end{minipage}\tabularnewline
\begin{minipage}[t]{0.13\columnwidth}\raggedright\strut
\(N_{thresh}\)\strut
\end{minipage} & \begin{minipage}[t]{0.76\columnwidth}\raggedright\strut
threshold in herring abundance that induces a reaction from a
predator\strut
\end{minipage}\tabularnewline
\begin{minipage}[t]{0.13\columnwidth}\raggedright\strut
\(AnnAlpha\)\strut
\end{minipage} & \begin{minipage}[t]{0.76\columnwidth}\raggedright\strut
intercept of Ford-Walford relationship as modified by a relationship
with herring\strut
\end{minipage}\tabularnewline
\begin{minipage}[t]{0.13\columnwidth}\raggedright\strut
\(AnnRho\)\strut
\end{minipage} & \begin{minipage}[t]{0.76\columnwidth}\raggedright\strut
slope of Ford-Walford relationship as modified by a relationship with
herring\strut
\end{minipage}\tabularnewline
\begin{minipage}[t]{0.13\columnwidth}\raggedright\strut
\(\delta\)\strut
\end{minipage} & \begin{minipage}[t]{0.76\columnwidth}\raggedright\strut
exponent relating herring abundance to predator survival\strut
\end{minipage}\tabularnewline
\begin{minipage}[t]{0.13\columnwidth}\raggedright\strut
\(Havgwt\)\strut
\end{minipage} & \begin{minipage}[t]{0.76\columnwidth}\raggedright\strut
average weight of herring\strut
\end{minipage}\tabularnewline
\begin{minipage}[t]{0.13\columnwidth}\raggedright\strut
\(\lambda\)\strut
\end{minipage} & \begin{minipage}[t]{0.76\columnwidth}\raggedright\strut
exponent relating herring average weight to tuna growth\strut
\end{minipage}\tabularnewline
\begin{minipage}[t]{0.13\columnwidth}\raggedright\strut
\(T\)\strut
\end{minipage} & \begin{minipage}[t]{0.76\columnwidth}\raggedright\strut
inflection point of the relationship between herring average weight and
tuna growth\strut
\end{minipage}\tabularnewline
\begin{minipage}[t]{0.13\columnwidth}\raggedright\strut
\(SSB_{MSY}\)\strut
\end{minipage} & \begin{minipage}[t]{0.76\columnwidth}\raggedright\strut
herring spawning stock biomass at maximum sustainable yield\strut
\end{minipage}\tabularnewline
\begin{minipage}[t]{0.13\columnwidth}\raggedright\strut
\(IAV\)\strut
\end{minipage} & \begin{minipage}[t]{0.76\columnwidth}\raggedright\strut
interannual variation in herring yield\strut
\end{minipage}\tabularnewline
\begin{minipage}[t]{0.13\columnwidth}\raggedright\strut
\(SSB_{MSY}^P\)\strut
\end{minipage} & \begin{minipage}[t]{0.76\columnwidth}\raggedright\strut
predator spawning stock biomass at maximum sustainable yield\strut
\end{minipage}\tabularnewline
\begin{minipage}[t]{0.13\columnwidth}\raggedright\strut
GR\strut
\end{minipage} & \begin{minipage}[t]{0.76\columnwidth}\raggedright\strut
gross revenue\strut
\end{minipage}\tabularnewline
\begin{minipage}[t]{0.13\columnwidth}\raggedright\strut
NR\strut
\end{minipage} & \begin{minipage}[t]{0.76\columnwidth}\raggedright\strut
net operating revenues\strut
\end{minipage}\tabularnewline
\begin{minipage}[t]{0.13\columnwidth}\raggedright\strut
i\strut
\end{minipage} & \begin{minipage}[t]{0.76\columnwidth}\raggedright\strut
fleet (trawl, t, or purse seine, s)\strut
\end{minipage}\tabularnewline
\begin{minipage}[t]{0.13\columnwidth}\raggedright\strut
q\strut
\end{minipage} & \begin{minipage}[t]{0.76\columnwidth}\raggedright\strut
quantity landed\strut
\end{minipage}\tabularnewline
\begin{minipage}[t]{0.13\columnwidth}\raggedright\strut
c()\strut
\end{minipage} & \begin{minipage}[t]{0.76\columnwidth}\raggedright\strut
cost function\strut
\end{minipage}\tabularnewline
\begin{minipage}[t]{0.13\columnwidth}\raggedright\strut
p\strut
\end{minipage} & \begin{minipage}[t]{0.76\columnwidth}\raggedright\strut
function relating landings to prices\strut
\end{minipage}\tabularnewline
\begin{minipage}[t]{0.13\columnwidth}\raggedright\strut
t\strut
\end{minipage} & \begin{minipage}[t]{0.76\columnwidth}\raggedright\strut
denotes trawl fishery in economic model\strut
\end{minipage}\tabularnewline
\begin{minipage}[t]{0.13\columnwidth}\raggedright\strut
s\strut
\end{minipage} & \begin{minipage}[t]{0.76\columnwidth}\raggedright\strut
denotes purse seine fishery in economic model\strut
\end{minipage}\tabularnewline
\begin{minipage}[t]{0.13\columnwidth}\raggedright\strut
\(a_i\)\strut
\end{minipage} & \begin{minipage}[t]{0.76\columnwidth}\raggedright\strut
parameter of price and landing model\strut
\end{minipage}\tabularnewline
\begin{minipage}[t]{0.13\columnwidth}\raggedright\strut
b\strut
\end{minipage} & \begin{minipage}[t]{0.76\columnwidth}\raggedright\strut
parameter of price and landing model\strut
\end{minipage}\tabularnewline
\begin{minipage}[t]{0.13\columnwidth}\raggedright\strut
\(\gamma\)\strut
\end{minipage} & \begin{minipage}[t]{0.76\columnwidth}\raggedright\strut
parameter of price and landing model\strut
\end{minipage}\tabularnewline
\begin{minipage}[t]{0.13\columnwidth}\raggedright\strut
\(\alpha\)\strut
\end{minipage} & \begin{minipage}[t]{0.76\columnwidth}\raggedright\strut
parameter of price and landing model\strut
\end{minipage}\tabularnewline
\begin{minipage}[t]{0.13\columnwidth}\raggedright\strut
\(\theta_i\)\strut
\end{minipage} & \begin{minipage}[t]{0.76\columnwidth}\raggedright\strut
parameter of price and landing model\strut
\end{minipage}\tabularnewline
\begin{minipage}[t]{0.13\columnwidth}\raggedright\strut
\(\beta\)\strut
\end{minipage} & \begin{minipage}[t]{0.76\columnwidth}\raggedright\strut
parameter of price and landing model\strut
\end{minipage}\tabularnewline
\begin{minipage}[t]{0.13\columnwidth}\raggedright\strut
\(\xi\)\strut
\end{minipage} & \begin{minipage}[t]{0.76\columnwidth}\raggedright\strut
parameter of economic stationarity metric\strut
\end{minipage}\tabularnewline
\bottomrule
\end{longtable}

\newpage

\begin{longtable}[]{@{}llll@{}}
\caption{Operating model uncertainties
addressed\label{OMs}}\tabularnewline
\toprule
Operating Model Name & Herring productivity & Herring growth &
Assessment bias\tabularnewline
\midrule
\endfirsthead
\toprule
Operating Model Name & Herring productivity & Herring growth &
Assessment bias\tabularnewline
\midrule
\endhead
LowFastBiased & Low: high \(M\), low \(h\) (0.44) & 1976-1985: fast &
60\% overestimate\tabularnewline
LowSlowBiased & Low: high \(M\), low \(h\) (0.44) & 2005-2014: slow &
60\% overestimate\tabularnewline
LowFastCorrect & Low: high \(M\), low \(h\) (0.44) & 1976-1985: fast &
None\tabularnewline
LowSlowCorrect & Low: high \(M\), low \(h\) (0.44) & 2005-2014: slow &
None\tabularnewline
HighFastBiased & High: low \(M\), high \(h\) (0.79) & 1976-1985: fast &
60\% overestimate\tabularnewline
HighSlowBiased & High: low \(M\), high \(h\) (0.79) & 2005-2014: slow &
60\% overestimate\tabularnewline
HighFastCorrect & High: low \(M\), high \(h\) (0.79) & 1976-1985: fast &
None\tabularnewline
HighSlowCorrect & High: low \(M\), high \(h\) (0.79) & 2005-2014: slow &
None\tabularnewline
\bottomrule
\end{longtable}

\newpage

\begin{longtable}[]{@{}lllll@{}}
\caption{Herring natural mortality and mean weights-at-age (kg)
\label{MWttab}}\tabularnewline
\toprule
Age & High \(M\) & Low \(M\) & Fast Growth & Slow Growth\tabularnewline
\midrule
\endfirsthead
\toprule
Age & High \(M\) & Low \(M\) & Fast Growth & Slow Growth\tabularnewline
\midrule
\endhead
1 & 0.81 & 0.54 & 0.01 & 0.02\tabularnewline
2 & 0.65 & 0.43 & 0.03 & 0.04\tabularnewline
3 & 0.54 & 0.36 & 0.09 & 0.07\tabularnewline
4 & 0.48 & 0.32 & 0.16 & 0.11\tabularnewline
5 & 0.45 & 0.3 & 0.21 & 0.14\tabularnewline
6 & 0.42 & 0.28 & 0.25 & 0.16\tabularnewline
7 & 0.41 & 0.27 & 0.28 & 0.18\tabularnewline
8+ & 0.40 & 0.26 & 0.31 & 0.21\tabularnewline
\bottomrule
\end{longtable}

\newpage

\begin{longtable}[]{@{}lllllll@{}}
\caption{Life history traits and reference points including unfished SSB
(i.e., virgin SSB), \(SSB_{MSY}\), MSY, and \(F_{MSY}\) reference points
for the Atlantic herring operating models \label{RefPts}}\tabularnewline
\toprule
\begin{minipage}[b]{0.07\columnwidth}\raggedright\strut
Steepness\strut
\end{minipage} & \begin{minipage}[b]{0.07\columnwidth}\raggedright\strut
Natural Mortality\strut
\end{minipage} & \begin{minipage}[b]{0.07\columnwidth}\raggedright\strut
Growth\strut
\end{minipage} & \begin{minipage}[b]{0.07\columnwidth}\raggedright\strut
Unfished SSB (mt)\strut
\end{minipage} & \begin{minipage}[b]{0.07\columnwidth}\raggedright\strut
SSB MSY (mt)\strut
\end{minipage} & \begin{minipage}[b]{0.07\columnwidth}\raggedright\strut
MSY (mt)\strut
\end{minipage} & \begin{minipage}[b]{0.07\columnwidth}\raggedright\strut
F MSY\strut
\end{minipage}\tabularnewline
\midrule
\endfirsthead
\toprule
\begin{minipage}[b]{0.07\columnwidth}\raggedright\strut
Steepness\strut
\end{minipage} & \begin{minipage}[b]{0.07\columnwidth}\raggedright\strut
Natural Mortality\strut
\end{minipage} & \begin{minipage}[b]{0.07\columnwidth}\raggedright\strut
Growth\strut
\end{minipage} & \begin{minipage}[b]{0.07\columnwidth}\raggedright\strut
Unfished SSB (mt)\strut
\end{minipage} & \begin{minipage}[b]{0.07\columnwidth}\raggedright\strut
SSB MSY (mt)\strut
\end{minipage} & \begin{minipage}[b]{0.07\columnwidth}\raggedright\strut
MSY (mt)\strut
\end{minipage} & \begin{minipage}[b]{0.07\columnwidth}\raggedright\strut
F MSY\strut
\end{minipage}\tabularnewline
\midrule
\endhead
\begin{minipage}[t]{0.07\columnwidth}\raggedright\strut
0.44\strut
\end{minipage} & \begin{minipage}[t]{0.07\columnwidth}\raggedright\strut
High\strut
\end{minipage} & \begin{minipage}[t]{0.07\columnwidth}\raggedright\strut
Slow\strut
\end{minipage} & \begin{minipage}[t]{0.07\columnwidth}\raggedright\strut
845176\strut
\end{minipage} & \begin{minipage}[t]{0.07\columnwidth}\raggedright\strut
324977\strut
\end{minipage} & \begin{minipage}[t]{0.07\columnwidth}\raggedright\strut
66061\strut
\end{minipage} & \begin{minipage}[t]{0.07\columnwidth}\raggedright\strut
0.31\strut
\end{minipage}\tabularnewline
\begin{minipage}[t]{0.07\columnwidth}\raggedright\strut
0.44\strut
\end{minipage} & \begin{minipage}[t]{0.07\columnwidth}\raggedright\strut
High\strut
\end{minipage} & \begin{minipage}[t]{0.07\columnwidth}\raggedright\strut
Fast\strut
\end{minipage} & \begin{minipage}[t]{0.07\columnwidth}\raggedright\strut
845176\strut
\end{minipage} & \begin{minipage}[t]{0.07\columnwidth}\raggedright\strut
335849\strut
\end{minipage} & \begin{minipage}[t]{0.07\columnwidth}\raggedright\strut
60969\strut
\end{minipage} & \begin{minipage}[t]{0.07\columnwidth}\raggedright\strut
0.28\strut
\end{minipage}\tabularnewline
\begin{minipage}[t]{0.07\columnwidth}\raggedright\strut
0.79\strut
\end{minipage} & \begin{minipage}[t]{0.07\columnwidth}\raggedright\strut
Low\strut
\end{minipage} & \begin{minipage}[t]{0.07\columnwidth}\raggedright\strut
Slow\strut
\end{minipage} & \begin{minipage}[t]{0.07\columnwidth}\raggedright\strut
1347080\strut
\end{minipage} & \begin{minipage}[t]{0.07\columnwidth}\raggedright\strut
369089\strut
\end{minipage} & \begin{minipage}[t]{0.07\columnwidth}\raggedright\strut
129171\strut
\end{minipage} & \begin{minipage}[t]{0.07\columnwidth}\raggedright\strut
0.54\strut
\end{minipage}\tabularnewline
\begin{minipage}[t]{0.07\columnwidth}\raggedright\strut
0.79\strut
\end{minipage} & \begin{minipage}[t]{0.07\columnwidth}\raggedright\strut
Low\strut
\end{minipage} & \begin{minipage}[t]{0.07\columnwidth}\raggedright\strut
Fast\strut
\end{minipage} & \begin{minipage}[t]{0.07\columnwidth}\raggedright\strut
1347080\strut
\end{minipage} & \begin{minipage}[t]{0.07\columnwidth}\raggedright\strut
405485\strut
\end{minipage} & \begin{minipage}[t]{0.07\columnwidth}\raggedright\strut
120360\strut
\end{minipage} & \begin{minipage}[t]{0.07\columnwidth}\raggedright\strut
0.45\strut
\end{minipage}\tabularnewline
\bottomrule
\end{longtable}

\newpage

\begin{longtable}[]{@{}lllll@{}}
\caption{Predator population model specification and parameter sources.
See methods for details of Gulf of Maine Seabird Working Group (GOMSWG)
data. \label{predsource}}\tabularnewline
\toprule
\begin{minipage}[b]{0.17\columnwidth}\raggedright\strut
\strut
\end{minipage} & \begin{minipage}[b]{0.17\columnwidth}\raggedright\strut
Highly migratory\strut
\end{minipage} & \begin{minipage}[b]{0.17\columnwidth}\raggedright\strut
Seabird\strut
\end{minipage} & \begin{minipage}[b]{0.17\columnwidth}\raggedright\strut
Groundfish\strut
\end{minipage} & \begin{minipage}[b]{0.17\columnwidth}\raggedright\strut
Marine mammal\strut
\end{minipage}\tabularnewline
\midrule
\endfirsthead
\toprule
\begin{minipage}[b]{0.17\columnwidth}\raggedright\strut
\strut
\end{minipage} & \begin{minipage}[b]{0.17\columnwidth}\raggedright\strut
Highly migratory\strut
\end{minipage} & \begin{minipage}[b]{0.17\columnwidth}\raggedright\strut
Seabird\strut
\end{minipage} & \begin{minipage}[b]{0.17\columnwidth}\raggedright\strut
Groundfish\strut
\end{minipage} & \begin{minipage}[b]{0.17\columnwidth}\raggedright\strut
Marine mammal\strut
\end{minipage}\tabularnewline
\midrule
\endhead
\begin{minipage}[t]{0.17\columnwidth}\raggedright\strut
Stakeholder preferred species\strut
\end{minipage} & \begin{minipage}[t]{0.17\columnwidth}\raggedright\strut
Bluefin tuna\strut
\end{minipage} & \begin{minipage}[t]{0.17\columnwidth}\raggedright\strut
Common tern\strut
\end{minipage} & \begin{minipage}[t]{0.17\columnwidth}\raggedright\strut
not specified\strut
\end{minipage} & \begin{minipage}[t]{0.17\columnwidth}\raggedright\strut
not specified\strut
\end{minipage}\tabularnewline
\begin{minipage}[t]{0.17\columnwidth}\raggedright\strut
Species modeled\strut
\end{minipage} & \begin{minipage}[t]{0.17\columnwidth}\raggedright\strut
Bluefin tuna (western Atlantic stock)\strut
\end{minipage} & \begin{minipage}[t]{0.17\columnwidth}\raggedright\strut
Common tern (Gulf of Maine colonies as defined by the GOM Seabird
Working Group)\strut
\end{minipage} & \begin{minipage}[t]{0.17\columnwidth}\raggedright\strut
Spiny dogfish (GOM and GB cod stocks also examined)\strut
\end{minipage} & \begin{minipage}[t]{0.17\columnwidth}\raggedright\strut
none, data limited (Minke \& humpback whales, harbor porpoise, harbor
seal examined)\strut
\end{minipage}\tabularnewline
\begin{minipage}[t]{0.17\columnwidth}\raggedright\strut
Stock-recruitment\strut
\end{minipage} & \begin{minipage}[t]{0.17\columnwidth}\raggedright\strut
ICCAT (2015) and Porch and Lauretta (2016)\strut
\end{minipage} & \begin{minipage}[t]{0.17\columnwidth}\raggedright\strut
Derived from GOMSWG data\strut
\end{minipage} & \begin{minipage}[t]{0.17\columnwidth}\raggedright\strut
Rago et al. (1998) and Rago and Sosebee (2010)\strut
\end{minipage} & \begin{minipage}[t]{0.17\columnwidth}\raggedright\strut
No time series data for our region\strut
\end{minipage}\tabularnewline
\begin{minipage}[t]{0.17\columnwidth}\raggedright\strut
Natural mortality\strut
\end{minipage} & \begin{minipage}[t]{0.17\columnwidth}\raggedright\strut
ICCAT (2015)\strut
\end{minipage} & \begin{minipage}[t]{0.17\columnwidth}\raggedright\strut
Nisbet (2002)\strut
\end{minipage} & \begin{minipage}[t]{0.17\columnwidth}\raggedright\strut
Rago and Sosebee (2013) and Rago pers. comm. 2016\strut
\end{minipage} & \begin{minipage}[t]{0.17\columnwidth}\raggedright\strut
Derivable from Hayes et al. (2017)?\strut
\end{minipage}\tabularnewline
\begin{minipage}[t]{0.17\columnwidth}\raggedright\strut
Fishing mortality\strut
\end{minipage} & \begin{minipage}[t]{0.17\columnwidth}\raggedright\strut
ICCAT (2015)\strut
\end{minipage} & \begin{minipage}[t]{0.17\columnwidth}\raggedright\strut
n/a\strut
\end{minipage} & \begin{minipage}[t]{0.17\columnwidth}\raggedright\strut
Rago and Sosebee (2013) and Rago pers. comm. 2016\strut
\end{minipage} & \begin{minipage}[t]{0.17\columnwidth}\raggedright\strut
Derivable from Hayes et al. (2017)?\strut
\end{minipage}\tabularnewline
\begin{minipage}[t]{0.17\columnwidth}\raggedright\strut
Initial population\strut
\end{minipage} & \begin{minipage}[t]{0.17\columnwidth}\raggedright\strut
ICCAT (2015)\strut
\end{minipage} & \begin{minipage}[t]{0.17\columnwidth}\raggedright\strut
Derived from GOMSWG data\strut
\end{minipage} & \begin{minipage}[t]{0.17\columnwidth}\raggedright\strut
Rago and Sosebee (2013) and Rago pers. comm. 2016\strut
\end{minipage} & \begin{minipage}[t]{0.17\columnwidth}\raggedright\strut
Derivable from Hayes et al. (2017)?\strut
\end{minipage}\tabularnewline
\begin{minipage}[t]{0.17\columnwidth}\raggedright\strut
Weight at age\strut
\end{minipage} & \begin{minipage}[t]{0.17\columnwidth}\raggedright\strut
Restrepo et al. (2010)\strut
\end{minipage} & \begin{minipage}[t]{0.17\columnwidth}\raggedright\strut
Nisbet (2002)\strut
\end{minipage} & \begin{minipage}[t]{0.17\columnwidth}\raggedright\strut
Rago et al. (1998)\strut
\end{minipage} & \begin{minipage}[t]{0.17\columnwidth}\raggedright\strut
Depending on species, derivable from literature\strut
\end{minipage}\tabularnewline
\bottomrule
\end{longtable}

\newpage

\begin{longtable}[]{@{}llll@{}}
\caption{Predator model input parameters\label{predpars}}\tabularnewline
\toprule
\begin{minipage}[b]{0.56\columnwidth}\raggedright\strut
Parameter\strut
\end{minipage} & \begin{minipage}[b]{0.11\columnwidth}\raggedright\strut
Tuna\strut
\end{minipage} & \begin{minipage}[b]{0.11\columnwidth}\raggedright\strut
Tern\strut
\end{minipage} & \begin{minipage}[b]{0.11\columnwidth}\raggedright\strut
Dogfish\strut
\end{minipage}\tabularnewline
\midrule
\endfirsthead
\toprule
\begin{minipage}[b]{0.56\columnwidth}\raggedright\strut
Parameter\strut
\end{minipage} & \begin{minipage}[b]{0.11\columnwidth}\raggedright\strut
Tuna\strut
\end{minipage} & \begin{minipage}[b]{0.11\columnwidth}\raggedright\strut
Tern\strut
\end{minipage} & \begin{minipage}[b]{0.11\columnwidth}\raggedright\strut
Dogfish\strut
\end{minipage}\tabularnewline
\midrule
\endhead
\begin{minipage}[t]{0.56\columnwidth}\raggedright\strut
Numbers or Weight?\strut
\end{minipage} & \begin{minipage}[t]{0.11\columnwidth}\raggedright\strut
Weight\strut
\end{minipage} & \begin{minipage}[t]{0.11\columnwidth}\raggedright\strut
Numbers\strut
\end{minipage} & \begin{minipage}[t]{0.11\columnwidth}\raggedright\strut
Weight\strut
\end{minipage}\tabularnewline
\begin{minipage}[t]{0.56\columnwidth}\raggedright\strut
Unfished spawning pop (metric tons for fish, n of nesting pairs for
tern)\strut
\end{minipage} & \begin{minipage}[t]{0.11\columnwidth}\raggedright\strut
6.69E+04\strut
\end{minipage} & \begin{minipage}[t]{0.11\columnwidth}\raggedright\strut
45000\strut
\end{minipage} & \begin{minipage}[t]{0.11\columnwidth}\raggedright\strut
300000\strut
\end{minipage}\tabularnewline
\begin{minipage}[t]{0.56\columnwidth}\raggedright\strut
Steepness \emph{h}\strut
\end{minipage} & \begin{minipage}[t]{0.11\columnwidth}\raggedright\strut
1.0\strut
\end{minipage} & \begin{minipage}[t]{0.11\columnwidth}\raggedright\strut
0.26\strut
\end{minipage} & \begin{minipage}[t]{0.11\columnwidth}\raggedright\strut
0.97\strut
\end{minipage}\tabularnewline
\begin{minipage}[t]{0.56\columnwidth}\raggedright\strut
Annual nat. mortality rate \emph{v}\strut
\end{minipage} & \begin{minipage}[t]{0.11\columnwidth}\raggedright\strut
0.14\strut
\end{minipage} & \begin{minipage}[t]{0.11\columnwidth}\raggedright\strut
0.1\strut
\end{minipage} & \begin{minipage}[t]{0.11\columnwidth}\raggedright\strut
0.092\strut
\end{minipage}\tabularnewline
\begin{minipage}[t]{0.56\columnwidth}\raggedright\strut
Annual exploitation rate \emph{u}\strut
\end{minipage} & \begin{minipage}[t]{0.11\columnwidth}\raggedright\strut
0.079\strut
\end{minipage} & \begin{minipage}[t]{0.11\columnwidth}\raggedright\strut
0.00\strut
\end{minipage} & \begin{minipage}[t]{0.11\columnwidth}\raggedright\strut
0.092\strut
\end{minipage}\tabularnewline
\begin{minipage}[t]{0.56\columnwidth}\raggedright\strut
Growth intercept \emph{FWint}\strut
\end{minipage} & \begin{minipage}[t]{0.11\columnwidth}\raggedright\strut
0.020605\strut
\end{minipage} & \begin{minipage}[t]{0.11\columnwidth}\raggedright\strut
0.00015\strut
\end{minipage} & \begin{minipage}[t]{0.11\columnwidth}\raggedright\strut
0.000278\strut
\end{minipage}\tabularnewline
\begin{minipage}[t]{0.56\columnwidth}\raggedright\strut
Growth slope \emph{FWslope }\strut
\end{minipage} & \begin{minipage}[t]{0.11\columnwidth}\raggedright\strut
0.9675\strut
\end{minipage} & \begin{minipage}[t]{0.11\columnwidth}\raggedright\strut
0.0\strut
\end{minipage} & \begin{minipage}[t]{0.11\columnwidth}\raggedright\strut
0.9577\strut
\end{minipage}\tabularnewline
\begin{minipage}[t]{0.56\columnwidth}\raggedright\strut
Initial abundance (n for fish, n nesting pairs for tern)\strut
\end{minipage} & \begin{minipage}[t]{0.11\columnwidth}\raggedright\strut
111864\strut
\end{minipage} & \begin{minipage}[t]{0.11\columnwidth}\raggedright\strut
3000\strut
\end{minipage} & \begin{minipage}[t]{0.11\columnwidth}\raggedright\strut
49629630\strut
\end{minipage}\tabularnewline
\begin{minipage}[t]{0.56\columnwidth}\raggedright\strut
Initial biomass (metric tons for fish, kg for tern but not used in
model)\strut
\end{minipage} & \begin{minipage}[t]{0.11\columnwidth}\raggedright\strut
27966\strut
\end{minipage} & \begin{minipage}[t]{0.11\columnwidth}\raggedright\strut
1.5\strut
\end{minipage} & \begin{minipage}[t]{0.11\columnwidth}\raggedright\strut
134000\strut
\end{minipage}\tabularnewline
\begin{minipage}[t]{0.56\columnwidth}\raggedright\strut
Recruit delay (age, years) \emph{a}\strut
\end{minipage} & \begin{minipage}[t]{0.11\columnwidth}\raggedright\strut
1\strut
\end{minipage} & \begin{minipage}[t]{0.11\columnwidth}\raggedright\strut
4\strut
\end{minipage} & \begin{minipage}[t]{0.11\columnwidth}\raggedright\strut
10\strut
\end{minipage}\tabularnewline
\begin{minipage}[t]{0.56\columnwidth}\raggedright\strut
Prey-recruitment link\strut
\end{minipage} & \begin{minipage}[t]{0.11\columnwidth}\raggedright\strut
1 (off)\strut
\end{minipage} & \begin{minipage}[t]{0.11\columnwidth}\raggedright\strut
1.09\strut
\end{minipage} & \begin{minipage}[t]{0.11\columnwidth}\raggedright\strut
1 (off)\strut
\end{minipage}\tabularnewline
\begin{minipage}[t]{0.56\columnwidth}\raggedright\strut
Prey-mortality link\strut
\end{minipage} & \begin{minipage}[t]{0.11\columnwidth}\raggedright\strut
0 (off)\strut
\end{minipage} & \begin{minipage}[t]{0.11\columnwidth}\raggedright\strut
0 (off)\strut
\end{minipage} & \begin{minipage}[t]{0.11\columnwidth}\raggedright\strut
0.2\strut
\end{minipage}\tabularnewline
\begin{minipage}[t]{0.56\columnwidth}\raggedright\strut
Prey-growth link\strut
\end{minipage} & \begin{minipage}[t]{0.11\columnwidth}\raggedright\strut
1.1\strut
\end{minipage} & \begin{minipage}[t]{0.11\columnwidth}\raggedright\strut
1 (off)\strut
\end{minipage} & \begin{minipage}[t]{0.11\columnwidth}\raggedright\strut
1 (off)\strut
\end{minipage}\tabularnewline
\bottomrule
\end{longtable}

\newpage

\begin{longtable}[]{@{}lrrrrrr@{}}
\caption{Marginal cost of 1mt of herring, daily catch, trip length, and
adjusted cost per day for the purse seine and trawl fleets averaged over
2011-2015.\label{costs_combined}}\tabularnewline
\toprule
\begin{minipage}[b]{0.07\columnwidth}\raggedright\strut
Year\strut
\end{minipage} & \begin{minipage}[b]{0.17\columnwidth}\raggedleft\strut
Marginal Cost (\$/mt)\strut
\end{minipage} & \begin{minipage}[b]{0.11\columnwidth}\raggedleft\strut
Catch (mt/day)\strut
\end{minipage} & \begin{minipage}[b]{0.14\columnwidth}\raggedleft\strut
Trip length (days)\strut
\end{minipage} & \begin{minipage}[b]{0.15\columnwidth}\raggedleft\strut
Adjusted Cost (\$/day)\strut
\end{minipage} & \begin{minipage}[b]{0.09\columnwidth}\raggedleft\strut
Observed trips\strut
\end{minipage} & \begin{minipage}[b]{0.08\columnwidth}\raggedleft\strut
VTR trips\strut
\end{minipage}\tabularnewline
\midrule
\endfirsthead
\toprule
\begin{minipage}[b]{0.07\columnwidth}\raggedright\strut
Year\strut
\end{minipage} & \begin{minipage}[b]{0.17\columnwidth}\raggedleft\strut
Marginal Cost (\$/mt)\strut
\end{minipage} & \begin{minipage}[b]{0.11\columnwidth}\raggedleft\strut
Catch (mt/day)\strut
\end{minipage} & \begin{minipage}[b]{0.14\columnwidth}\raggedleft\strut
Trip length (days)\strut
\end{minipage} & \begin{minipage}[b]{0.15\columnwidth}\raggedleft\strut
Adjusted Cost (\$/day)\strut
\end{minipage} & \begin{minipage}[b]{0.09\columnwidth}\raggedleft\strut
Observed trips\strut
\end{minipage} & \begin{minipage}[b]{0.08\columnwidth}\raggedleft\strut
VTR trips\strut
\end{minipage}\tabularnewline
\midrule
\endhead
\begin{minipage}[t]{0.07\columnwidth}\raggedright\strut
Seine\strut
\end{minipage} & \begin{minipage}[t]{0.17\columnwidth}\raggedleft\strut
14.27\strut
\end{minipage} & \begin{minipage}[t]{0.11\columnwidth}\raggedleft\strut
87.5\strut
\end{minipage} & \begin{minipage}[t]{0.14\columnwidth}\raggedleft\strut
1.0\strut
\end{minipage} & \begin{minipage}[t]{0.15\columnwidth}\raggedleft\strut
1,249\strut
\end{minipage} & \begin{minipage}[t]{0.09\columnwidth}\raggedleft\strut
207\strut
\end{minipage} & \begin{minipage}[t]{0.08\columnwidth}\raggedleft\strut
1,413\strut
\end{minipage}\tabularnewline
\begin{minipage}[t]{0.07\columnwidth}\raggedright\strut
Trawl\strut
\end{minipage} & \begin{minipage}[t]{0.17\columnwidth}\raggedleft\strut
62.43\strut
\end{minipage} & \begin{minipage}[t]{0.11\columnwidth}\raggedleft\strut
61.4\strut
\end{minipage} & \begin{minipage}[t]{0.14\columnwidth}\raggedleft\strut
2.9\strut
\end{minipage} & \begin{minipage}[t]{0.15\columnwidth}\raggedleft\strut
3,833\strut
\end{minipage} & \begin{minipage}[t]{0.09\columnwidth}\raggedleft\strut
573\strut
\end{minipage} & \begin{minipage}[t]{0.08\columnwidth}\raggedleft\strut
2,005\strut
\end{minipage}\tabularnewline
\bottomrule
\end{longtable}

\newpage

\begin{table}[htbp]
\begin{center}
\caption{Regression results from the ``ARDL'' specification (Equation \ref{ARDL}), PSS Bounds test and associated critical values.  1995-2015 data only. The explanatory variables enter in levels in the first column and natural logarithms in the second column.  \label{ardl_regression}}
\begin{tabular}{lcc} \hline
Model       &   (1) &   (2)     \\
    &   price & \textit{ln} price   \\\hline
Price$_{t-1}$       &   0.646***    &   0.666***    \\
        &   -0.0937 &   -0.0935 \\
Quantity        &   -1.194***   &   -0.395***   \\
        &   -0.277  &   -0.0956 \\
Constant        &   217.8***    &   6.423***    \\
        &   -48.16  &   -1.484  \\
Observations        &   21  &   21  \\
R-squared       &   0.906   &   0.906   \\
BGp     &   0.34    &   0.547   \\
BGF     &   1   &   0.4 \\
PSS F-statistic &   10.2    &   9.34    \\\hline
1\% Critical values &\multicolumn{2}{c}{[6.84, 7.84]} \\\hline
\multicolumn{3}{c}{ Standard errors in parentheses} \\
\multicolumn{3}{c}{ *** p$<$0.01, ** p$<$0.05, * p$<$0.1} \\\hline
\end{tabular}
\end{center}
\end{table}

\newpage

\section{Figures}\label{figures}

\listoffigures

\newpage

\begin{figure}

{\centering \includegraphics{HerringTechnicalManu_files/figure-latex/unnamed-chunk-1-1} 

}

\caption{Modeled herring average weight (population >180mm in length) to tuna growth relationship. See text (Herring-tuna relationship model section) for derivation. \label{herringtuna}}\label{fig:unnamed-chunk-1}
\end{figure}\newpage

\begin{figure}

{\centering \includegraphics{HerringTechnicalManu_files/figure-latex/unnamed-chunk-2-1} 

}

\caption{Stock-recruitment function for Gulf of Maine common terns assuming 10\% fledgling to adult survival. Fitted parameters with all years of the common tern dataset included a non-significant beta parameter (dashed line), while fits to a truncated dataset resulted in low population production rates inconsistent with currently observed common tern trends (dotted line). Therefore, steepness was estimated to give a relationship (solid black line) falling between these two lines. \label{ternSR}}\label{fig:unnamed-chunk-2}
\end{figure}

\newpage

\begin{figure}

{\centering \includegraphics{HerringTechnicalManu_files/figure-latex/unnamed-chunk-3-1} 

}

\caption{Gulf of Maine tern annual productivity distributions by majority diet item offered to fledglings. Boxplots represent the median (wide line within the box) and 25th and 75th percentiles (box) of annual productivity measured across all nesting colonies in the Gulf of Maine where that prey species was the majority in the diet. Boxplot whiskers include the highest and lowest observations within 1.5 box lengths from the box. Observations further from the box (outliers) are represented by points. The black line represents the target tern productivity of 1.0 fledgling per nest. \label{chickdiet}}\label{fig:unnamed-chunk-3}
\end{figure}

\newpage

\begin{figure}

{\centering \includegraphics[width=\linewidth]{HerringTechnicalManu_files/figure-latex/unnamed-chunk-4-1} 

}

\caption{Herring proporiton in diet and annual tern productivity (fledglings/nest) by Gulf of Maine colony. Colony names are at the top of each subplot; I = island. Two colonies, Machias Seal and Stratton above, have significant positive Spearman's rank correlations between herring proportion in diet and annual productivity for common terns. No linear model slopes were significant, so none are shown in the plot. The black line represents the target tern productivity of 1.0 fledgling per nest. \label{proddiet}}\label{fig:unnamed-chunk-4}
\end{figure}\newpage

\begin{figure}

{\centering \includegraphics{HerringTechnicalManu_files/figure-latex/unnamed-chunk-5-1} 

}

\caption{Modeled influence of herring total biomass on tern reproductive success. Total annual productivity (fledglings per nest) for both tern species relative to assessed herring total biomass is shown, but the modeled relationship (curve) is based only on common terns. A linear relationship between herring total biomass and common tern productivity crosses tern productivity=1 (black dotted line) at ~400,000 tons herring total biomass. This linear relationship does not have a statistically significant slope; the curve was fit to represent a level but positive contribution of herring total biomass to common tern productivity above the threshold. The black horizontal line represents the target tern productivity of 1.0 fledgling per nest. \label{herrternmod}}\label{fig:unnamed-chunk-5}
\end{figure}

\newpage

\begin{figure}

{\centering \includegraphics{HerringTechnicalManu_files/figure-latex/unnamed-chunk-7-1} 

}

\caption{Population trends for Gulf of Maine terns with and without the simulated herring-common tern productivity relationship. Linear model fit (line) with 95\% CI band (gray shading) is shown for significant relationships. \label{terntrendwherring}}\label{fig:unnamed-chunk-7}
\end{figure}

\newpage

\begin{figure}

{\centering \includegraphics[width=\linewidth]{HerringTechnicalManu_files/figure-latex/unnamed-chunk-8-1} 

}

\caption{Annual percent of herring in diet compositions for major groundfish predators of herring (dogfish and two cod stocks) estimated from NEFSC food habits database. Prey identified to herring family (Clupeidae, open bars) as well as Atlantic herring (filled bars) are included. \label{gfishdiets}}\label{fig:unnamed-chunk-8}
\end{figure}

\newpage

\begin{figure}

{\centering \includegraphics{HerringTechnicalManu_files/figure-latex/unnamed-chunk-9-1} 

}

\caption{Relationship of groundfish predator (dogfish, GBcod = Georges Bank cod, and GOMcod = Gulf of Maine cod) spawning stock biomass (SSB) with the percentage of herring in diet: both all clupeids including Atlantic herring (All Clupeids) and Atlantic herring only (A. herring). Linear model fit (line) with 95\% CI band (gray shading) is shown for significant relationships. \label{gfishBherrdiet}}\label{fig:unnamed-chunk-9}
\end{figure}\newpage

\begin{figure}

{\centering \includegraphics{HerringTechnicalManu_files/figure-latex/unnamed-chunk-11-1} 

}

\caption{Modeled herring relative population size-dogfish natural mortality relationship \label{herringdogfish}}\label{fig:unnamed-chunk-11}
\end{figure}

\newpage

\begin{figure}

{\centering \includegraphics[width=\linewidth]{HerringTechnicalManu_files/figure-latex/YAY-1} 

}

\caption{Differences between operating models (OM) for key metrics. Boxplots represent the median (wide line within the box) and 25th and 75th percentiles (box) of the distribution of medians for final 50 years of each simulation for each performance metric (x-axis label, please see text for definitions) and operating model (see Table 2 for definitions) across all control rule types. Boxplot whiskers include the highest and lowest observations within 1.5 box lengths from the box. Observations further than 1.5 box lengths from the box (outliers) are represented by points. \label{YAY}}\label{fig:YAY}
\end{figure}

\newpage

\begin{figure}

{\centering \includegraphics{HerringTechnicalManu_files/figure-latex/ZZZ-1} 

}

\caption{Differences between control rule types for key metrics. Boxplots represent the median (wide line within the box) and 25th and 75th percentiles (box) of the distribution of medians for final 50 years of each simulation for each performance metric (x-axis label, please see text for definitions) and control rule type (CR; described in the text Harvest control rules section: biomass based with 1 (BB), 3 (BB3yr), and 5 (BB5yr) quota blocks, biomass based 3 year quota block with a 15\% restriction on interannual quota change (BB3yrPerc), constant catch (CC), and conditional constant catch(CCC)) across all operating models. Boxplot whiskers include the highest and lowest observations within 1.5 box lengths from the box. Observations further than 1.5 box lengths from the box (outliers) are represented by points.\label{ZZZ}}\label{fig:ZZZ}
\end{figure}

\newpage

\begin{figure}

{\centering \includegraphics{HerringTechnicalManu_files/figure-latex/yieldSSB-1} 

}

\caption{Tradeoff between herring relative yield and relative SSB for HighSlowCorrect operating model and biomass based control rules with 1 year (BB), 3 year (BB3yr), or 5 year (BB5yr) quota blocks. Each point represents the median of 100 medians taken over the final 50 years of each simulation. \label{yieldSSB}}\label{fig:yieldSSB}
\end{figure}

\newpage

\begin{figure}

{\centering \includegraphics{HerringTechnicalManu_files/figure-latex/yieldIAV-1} 

}

\caption{Tradeoff between herring relative yield and variation in yield for HighSlowCorrect operating model and biomass based control rules with 1 year (BB), 3 year (BB3yr), or 5 year (BB5yr) quota blocks. Each point represents the median of 100 medians taken over the final 50 years of each simulation.\label{yieldIAV}}\label{fig:yieldIAV}
\end{figure}

\newpage

\begin{figure}

{\centering \includegraphics{HerringTechnicalManu_files/figure-latex/yieldclose-1} 

}

\caption{Tradeoff between herring relative yield and frequency of fishery closure for HighSlowCorrect operating model and biomass based control rules with 1 year (BB), 3 year (BB3yr), or 5 year (BB5yr) quota blocks. Each point represents the median of 100 medians taken over the final 50 years of each simulation.\label{yieldPropClosure}}\label{fig:yieldclose}
\end{figure}

\newpage

\begin{figure}

{\centering \includegraphics{HerringTechnicalManu_files/figure-latex/yieldOF-1} 

}

\caption{Tradeoff between herring relative yield and probability of the stock is overfished for HighSlowCorrect operating model and biomass based control rules with 1 year (BB), 3 year (BB3yr), or 5 year (BB5yr) quota blocks. Each point represents the median of 100 medians taken over the final 50 years of each simulation.\label{yieldProbOF}}\label{fig:yieldOF}
\end{figure}

\newpage

\begin{figure}

{\centering \includegraphics{HerringTechnicalManu_files/figure-latex/yieldbirds-1} 

}

\caption{Tradeoff between herring relative yield and probability of good tern productivity for HighSlowCorrect operating model and biomass based control rules with 1 year (BB), 3 year (BB3yr), or 5 year (BB5yr) quota blocks. Each point represents the median of 100 medianstaken over the final 50 years of each simulation.\label{yieldbirds}}\label{fig:yieldbirds}
\end{figure}

\newpage

\begin{figure}

{\centering \includegraphics{HerringTechnicalManu_files/figure-latex/revenuestability-1} 

}

\caption{Tradeoff between herring fleet Net Revenue and Equil1 for HighSlowCorrect operating model and biomass based control rules with 1 year (BB), 3 year (BB3yr), or 5 year (BB5yr) quota blocks compared with 3 year quota block with a 15\% restriction on annual quota changes (BB3yrPerc). Each point represents the median of 100 medians taken over the final 50 years of each simulation.\label{revenuestability}}\label{fig:revenuestability}
\end{figure}


\end{document}
